\chapter{Extensions of {\sf Gallina}}
\label{Gallina-extension}\index{Gallina}

{\gallina} is the kernel language of {\Coq}. We describe here extensions of
the Gallina's syntax.

\section{Record types}\comindex{Record}
\label{Record}

The \verb+Record+ construction is a macro allowing the definition of
records as is done in many programming languages.  Its syntax is
described on figure \ref{record-syntax}.  In fact, the \verb+Record+
macro is more general than the usual record types, since it allows
also for ``manifest'' expressions. In this sense, the \verb+Record+
construction allows to define ``signatures''.

\begin{figure}
\begin{tabular}{|lcl|}
\hline
{\sentence} & ::= & {\record}\\
  & & \\
{\record} & ::= &  {\tt Record} {\ident} \zeroone{{\tt [} {\params} {\tt ]}} {\tt :} {\sort}
    \verb.:=. \zeroone{\ident} \verb!{! 
    \zeroone{\nelist{\field}{;}}
    \verb!}! \verb:.:\\
  & & \\
{\field} & ::= & {\ident} \verb.:. {\type} \\
 & $|$ & {\ident} \verb.:=. {\term} \\
 & $|$ & {\ident} \verb.:. {\type} \verb.:=. {\term} \\
\hline
\end{tabular}
\caption{Syntax for the definition of {\tt Record}}
\label{record-syntax}
\end{figure}

\noindent In the expression

\smallskip
{\tt Record} {\ident} {\tt [} {\params} {\tt ]} \texttt{:} 
   {\sort} := {\ident$_0$} \verb+{+
 {\ident$_1$} \texttt{:} {\term$_1$}; 
              \dots
  {\ident$_n$} \texttt{:} {\term$_n$} \verb+}+.
\smallskip
 
the identifier {\ident} is the name of the defined record and {\sort}
is its type. The identifier {\ident$_0$} is the name of its
constructor. If {\ident$_0$} is omitted, the default name {\tt
Build\_{\ident}} is used. The identifiers {\ident$_1$}, ..,
{\ident$_n$} are the names of fields and {\term$_1$}, .., {\term$_n$}
their respective types. Remark that the type of {\ident$_i$} may
depend on the previous {\ident$_j$} (for $j<i$). Thus the order of the
fields is important. Finally, {\params} are the parameters of the
record.

More generally, a record may have explicitly defined (a.k.a.
manifest) fields. For instance, {\tt Record} {\ident} {\tt [}
{\params} {\tt ]} \texttt{:} {\sort} := \verb+{+ {\ident$_1$}
\texttt{:} {\type$_1$} \verb+;+ {\ident$_2$} \texttt{:=} {\term$_2$} \verb+;+.
{\ident$_3$} \texttt{:} {\type$_3$} \verb+}+ in which case the correctness of {\type$_3$} may rely on the instance {\term$_2$}  of  {\ident$_2$ and  {\term$_2$}  in turn may depend on {\ident$_1$}.


\Example
The set of rational numbers may be defined as:
\begin{coq_eval}
Restore State Initial.
\end{coq_eval}
\begin{coq_example}
Record Rat : Set := mkRat {
 sign     : bool;
 top      : nat;
 bottom   : nat;
 Rat_bottom_cond : ~O=bottom;
 Rat_irred_cond:(x,y,z:nat)(mult x y)=top/\(mult x z)=bottom->x=(S O)}.
\end{coq_example}

Remark here that the field
\verb+Rat_cond+ depends on the field \verb+bottom+. 

Let us now see the work done by the {\tt Record} macro.
First the macro generates a inductive definition
with just one constructor:

\medskip
\noindent
{\tt Inductive {\ident} [ {\params} ] : {\sort} := \\
\mbox{}\hspace{0.4cm} {\ident$_0$} : ({\ident$_1$}:{\term$_1$}) .. 
({\ident$_n$}:{\term$_n$})({\ident} {\rm\sl params}).}
\medskip

To build an object of type {\ident}, one should provide the
constructor {\ident$_0$} with $n$ terms filling the fields of
the record.

Let us define the rational $1/2$.

\begin{coq_example*}
Require Arith.
Theorem one_two_irred: (x,y,z:nat)(mult x y)=(1)/\(mult x z)=(2)->x=(1).
\end{coq_example*}
\begin{coq_eval}
Lemma plus_m_n_eq_n_O : (m,n:nat)(plus m n)=O -> n=O.
Destruct m; Trivial.
Intros; Discriminate.
Qed.

Lemma mult_m_n_eq_m_1: (m,n:nat)(mult m n)=((S O))->m=(S O).
Destruct m;Trivial.
Intros; Apply f_equal with f:=S.
Generalize H.
Case n; Trivial.
Simpl.
Case n0;Simpl.
Intro; Rewrite <- mult_n_O; Intro; Discriminate.
Intros n1 n2 H0; Injection H0.
Intro H1.
LetTac H2:=(plus_m_n_eq_n_O n1 (S (plus n1 (mult n2 (S n1)))) H1).
Discriminate.
Qed.

Intros x y z (H1,H2). Apply mult_m_n_eq_m_1 with n:=y; Trivial.
\end{coq_eval}
\ldots
\begin{coq_example*}
Qed.
\end{coq_example*}
\begin{coq_example}
Definition half := (mkRat true (1) (2) (O_S (1)) one_two_irred).
\end{coq_example*}
\begin{coq_example}
Check half.
\end{coq_example}

The macro generates also, when it is possible, the projection
functions for destructuring an object of type {\ident}.
These projection functions have the same name that the corresponding
fields. In our example:

\begin{coq_example}
Eval Compute in (top half).
Eval Compute in (bottom half).
Eval Compute in (Rat_bottom_cond half).
\end{coq_example}
\begin{coq_eval}
Restore State Initial.
\end{coq_eval}

\begin{Warnings}
\item {\tt Warning: {\ident$_i$} cannot be defined.}\\
  It can happens that the definition of a projection is impossible.
  This message is followed by an explanation of this impossibility.
  There may be three reasons:
   \begin{enumerate}
   \item The name {\ident$_i$} already exists in the environment (see
     section \ref{Axiom}).
   \item The body of {\ident$_i$} uses a incorrect elimination for
     {\ident} (see sections \ref{Fixpoint} and \ref{Caseexpr}).
   \item {\tt The projections [ {\rm\sl idents} ] were not defined.}\\ 
     The body of {\term$_i$} uses the projections {\rm\sl idents}
     which are not defined for one of these three reasons listed here.
   \end{enumerate}  
\end{Warnings}     

\begin{ErrMsgs}
\item \errindex{A record cannot be recursive}\\
  The record name {\ident} appears in the type of its fields.

\item
  During the definition of the one-constructor inductive definition,
  all the errors of inductive definitions, as described in section
  \ref{gal_Inductive_Definitions}, may also occur.
\end{ErrMsgs}

\SeeAlso Coercions and records in section \ref{Coercions-and-records}
of the chapter devoted to coercions.

\section{Variants and extensions of {\tt Cases}}
\label{ExtensionsOfCases}
\index{Cases@{\tt Cases\ldots of\ldots end}}

\subsection{ML-style pattern-matching}
\index{ML-like patterns}
\label{Mult-Cases}

The basic version of \verb+Cases+ allows pattern-matching on simple
patterns. As an extension, multiple and nested patterns are
allowed, as in ML-like languages.

The extension just acts as a macro that is expanded during parsing
into a sequence of {\tt Cases} on simple patterns. Especially, a
construction defined using the extended {\tt Cases} is printed under
its expanded form.

The syntax of the extended {\tt Cases} is presented in figure 
\ref{ecases-grammar}.
Note the annotation is mandatory when the sequence of equation is
empty.

\begin{figure}[t]
\begin{tabular}{|rcl|}
\hline
{\nestedpattern}  & := & {\ident} \\
 & $|$ & \_ \\
 & $|$ & \texttt{(} {\ident} \nelist{\nestedpattern}{} \texttt{)} \\
 & $|$ & \texttt{(} {\nestedpattern} \texttt{as}  {\ident} \texttt{)} \\
 & $|$ & \texttt{(} {\nestedpattern} \texttt{,} {\nestedpattern} \texttt{)} \\
 & $|$ & \texttt{(} {\nestedpattern} \texttt{)} \\
 &&\\

{\multpattern} & := &  \nelist{nested\_pattern}{} \\
 && \\

{\exteqn} & := &  {\multpattern} \texttt{=>} {\term} \\
 && \\

{\term} & := &  
 \zeroone{\annotation} \texttt{Cases} \nelist{\term}{} \texttt{of} 
\sequence{\exteqn}{$|$} \texttt{end} \\
\hline
\end{tabular}
\caption{extended Cases syntax.}
\label{ecases-grammar}
\end{figure}

\SeeAlso chapter \ref{Mult-Cases-full}.

\subsection{Pattern-matching on boolean values: the {\tt if} expression}
\index{if@{\tt if ... then ... else}}

For inductive types isomorphic to the boolean types (i.e. two
constructors without arguments), it is possible to use a {\tt if
... then ... else} notation. This enriches the syntax of terms as follows:

\medskip
\begin{tabular}{rcl}
term & := & \zeroone{\annotation} {\tt if} {\term} {\tt then} {\term} {\tt else} {\term}\\
\end{tabular}
\medskip

For instance, the definition

\begin{coq_example}
Definition not := [b:bool] Cases b of true => false | false => true end.
\end{coq_example}

can be alternatively written

\begin{coq_eval}
Reset not.
\end{coq_eval}
\begin{coq_example}
Definition not := [b:bool] if b then false else true.
\end{coq_example}

\subsection{Irrefutable patterns: the destructuring {\tt let}}
\index{let in@{\tt let ... in}}
\label{Letin}

Terms in an inductive type having only one constructor, say {\tt foo}, have
necessarily the form \texttt{(foo ...)}. In this case, the {\tt Cases}
construction can be replaced by a {\tt let ... in ...} construction.
This enriches the syntax of terms as follows:

\medskip
\begin{tabular}{rcl}
 & $|$ & \zeroone{\annotation} {\tt let (} \nelist{\ident}{,} {\tt ) =}  {\term} {\tt in} {\term} \\
\end{tabular}
\medskip

For instance, the definition

\begin{coq_example}
Definition fst := [A,B:Set][H:A*B] Cases H of (pair x y) => x end.
\end{coq_example}

can be alternatively written 

\begin{coq_eval}
Reset fst.
\end{coq_eval}
\begin{coq_example}
Definition fst := [A,B:Set][p:A*B] let (x,_) = p in x.
\end{coq_example}

The pretty-printing of a definition by cases on a irrefutable pattern
can either be done using {\tt Cases} or the {\tt let}
construction (see section \ref{printing-options}).

\subsection{Options for pretty-printing of {\tt Cases}}
\label{printing-options}

There are three options controlling the pretty-printing of {\tt Cases}
expressions.

\subsubsection{Printing of wildcard pattern}
\comindex{Set Printing Wildcard}
\comindex{Unset Printing Wildcard}
\comindex{Print Printing Wildcard}

Some variables in a pattern may not occur in the right-hand side of
the pattern-matching clause.  There are options to control the
display of these variables.

\subsubsection{\tt Set Printing Wildcard.}
 The variables having no occurrences
in the right-hand side of the pattern-matching clause are just
printed using the wildcard symbol ``{\tt \_}''.

\subsubsection{\tt Unset Printing Wildcard.}
The variables, even useless, are printed using their usual name. But some
non dependent variables have no name. These ones are still printed
using a ``{\tt \_}''.

\subsubsection{\tt Print Printing Wildcard.}
This tells if the wildcard
printing mode is on or off. The default is to print wildcard for
useless variables.

\subsubsection{Printing of the elimination predicate}
\comindex{Set Printing Synth}
\comindex{Unset Printing Synth}
\comindex{Print Printing Synth}

In most of the cases, the type of the result of a matched term is
mechanically synthesizable. Especially, if the result type does not
depend of the matched term.

\subsubsection{\tt Set Printing Synth.}
The result type is not printed
when it is easily synthesizable.

\subsubsection{\tt Unset Printing Synth.}
This forces the result type to be always printed (and then between
angle brackets).

\subsubsection{\tt Print Printing Synth.}
This tells if the non-printing
of synthesizable types is on or off. The default is to not print 
synthesizable types.

\subsubsection{Printing matching on irrefutable pattern}
\comindex{Add Printing Let {\ident}}
\comindex{Remove Printing Let {\ident}}
\comindex{Test Printing Let {\ident}}
\comindex{Print Printing Let}

If an inductive type has just one constructor,
pattern-matching can be written using {\tt let} ... {\tt =}
... {\tt in}~...

\subsubsection{\tt Add Printing Let {\ident}.}
This adds {\ident} to the list
of inductive types for which pattern-matching is written using a {\tt
let} expression.

\subsubsection{\tt Remove Printing Let {\ident}.}
This removes {\ident} from this list.

\subsubsection{\tt Test Printing Let {\ident}.}
This tells if {\ident} belongs
to the list.

\subsubsection{\tt Print Printing Let.}
This prints the list of inductive types
for which pattern-matching is written using a {\tt
let} expression.

The table of inductive types for which pattern-matching is written
using a {\tt let} expression is managed synchronously. This means that
it is sensible to the command {\tt Reset}.


\subsubsection{Printing matching on booleans}
\comindex{Add Printing If {\ident}}
\comindex{Remove Printing If {\ident}}
\comindex{Test Printing If {\ident}}
\comindex{Print Printing If}

If an inductive type is isomorphic to the boolean type,
pattern-matching can be written using {\tt if} ... {\tt then}
... {\tt else} ...

\subsubsection{\tt Add Printing If {\ident}.}
This adds {\ident} to the list
of inductive types for which pattern-matching is written using an {\tt
if} expression.

\subsubsection{\tt Remove Printing If {\ident}.}
This removes {\ident} from this list.

\subsubsection{\tt Test Printing If {\ident}.}
This tells if {\ident} belongs
to the list.

\subsubsection{\tt Print Printing If.}
This prints the list of inductive types
for which pattern-matching is written using an {\tt
if} expression.

The table of inductive types for which pattern-matching is written
using an {\tt if} expression is managed synchronously. This means that
it is sensible to the command {\tt Reset}.

\subsubsection{Example}

This example emphasizes what the printing options offer.

\begin{coq_example}
Test Printing Let prod.
Print fst.
Remove Printing Let prod.
Unset Printing Synth.
Unset Printing Wildcard.
Print snd.
\end{coq_example}

% \subsection{Still not dead old notations}

% The following variant of {\tt Cases} is inherited from older version
% of {\Coq}. 

% \medskip
% \begin{tabular}{lcl}
% {\term} & ::= & {\annotation} {\tt Match} {\term} {\tt with} {\terms} {\tt end}\\
% \end{tabular}
% \medskip

% This syntax is a macro generating a combination of {\tt Cases} with {\tt
% Fix} implementing a combinator for primitive recursion equivalent to
% the {\tt Match} construction of \Coq\ V5.8. It is provided only for
% sake of compatibility with \Coq\ V5.8. It is recommended to avoid it.
% (see section~\ref{Matchexpr}).

% There is also a notation \texttt{Case} that is the
% ancestor of \texttt{Cases}. Again, it is still in the code for
% compatibility with old versions but the user should not use it.

\section{Forced type}

In some cases, one want to assign a particular type to a term. The
syntax to force the type of a term is the following:

\medskip
\begin{tabular}{lcl}
{\term} & ::= & {\tt (} {\term} {\tt ::} {\term} {\tt )}\\
\end{tabular}
\medskip

It forces the first term to be of type the second term. The
type must be compatible with
the term. More precisely it must be either a type convertible to
the automatically inferred type (see chapter \ref{Cic}) or a type
coercible to it, (see \ref{Coercions}). When the type of a
whole expression is forced, it is usually not necessary to give the types of
the variables involved in the term.

Example:

\begin{coq_example}
Definition ID := (X:Set) X -> X.
Definition id := (([X][x]x) :: ID).
Check id.
\end{coq_example}

\section{Section mechanism}\index{Sections}\label{Section}
The sectioning mechanism allows to organize a proof in structured
sections. Then local declarations become available (see section
\ref{Simpl-definitions}).

\subsection{\tt Section {\ident}}\comindex{Section}
This command is used to open a section named {\ident}.


\begin{Variants}
\comindex{Chapter}
\item{\tt Chapter {\ident}}\\
        Same as {\tt Section {\ident}}
\end{Variants}

\subsection{\tt End {\ident}}\comindex{End}
This command closes the section named {\ident}. When a section is
closed, all local declarations are discharged. This means that all
global objects defined in the section are {\it closed} (in the sense
of $\lambda$-calculus) with as many abstractions as there were local
declarations in the section explicitly occurring in the term.  A local
object in the section is not exported and its value will be
substituted in the other definitions.

Here is an example :
\begin{coq_example}
Section s1.
Variables x,y : nat.
Local y' := y.
Definition x' := (S x).
Print x'.
End s1.
Print x'.
\end{coq_example}
Note the difference between the value of {\tt x'} inside section {\tt
  s1} and outside.

\begin{ErrMsgs}
\item \errindex{Section {\ident} does not exist (or is already closed)}
\item \errindex{Section {\ident} is not the innermost section}
\end{ErrMsgs}

\begin{Remarks}
\item Most commands, like {\tt Hint \ident} or {\tt Syntactic
  Definition} which appear inside a section are cancelled when the
section is closed.
% See \ref{LongNames}
%\item Usually all identifiers must be distinct. 
%However, a name already used in a closed section (see \ref{Section})
%can be reused. In this case, the old name is no longer accessible.

% Obsol�te
%\item A module implicitly open a section. Be careful not to name a
%module with an identifier already used in the module (see \ref{compiled}).
\end{Remarks}

\section{Libraries and long names}
\label{LongNames}

\paragraph{Libraries}

The theories developed in {\Coq} are stored in {\em libraries}.  A
library is characterized by a name called {\it root} of the
library. The standard library of {\Coq} has root name {\tt Coq} and is
known by default when a {\Coq} session starts.

Libraries have a tree structure. E.g., the {\tt Coq} library
contains the sub-libraries {\tt Init}, {\tt Logic}, {\tt Arith}, {\tt
Lists}, ... The ``dot notation'' is used to separate the different
component of a library name. For instance, the {\tt Arith} library of
{\Coq} standard library is written ``{\tt Coq.Arith}''.

\medskip
\Rem no blank is allowed between the dot and the identifier on its
right, otherwise the dot is interpreted as the full stop (period) of
the command!
\medskip

\paragraph{Physical paths vs logical paths}

Libraries and sub-libraries are denoted by {\em logical directory
paths} (written {\dirpath} and of which the syntax is the same as
{\qualid}, see \ref{qualid}). Logical directory
paths can be mapped to physical directories of the
operating system using the command (see \ref{AddLoadPath})

\begin{quote}
{\tt Add LoadPath {\it physical\_path} as {\dirpath}}.
\end{quote}

A library can inherit the tree structure of a physical directory by
using the {\tt -R} option to {\tt coqtop} (see \ref{coqtop}) or the
command (see \ref{AddRecLoadPath})

\begin{quote}
{\tt Add Rec LoadPath {\it physical\_dir} as {\dirpath}}.
\end{quote}

\paragraph{Modules}
At some point, (sub-)libraries contain {\it modules} which coincide
with files at the physical level. 
As for sublibraries, the dot notation is used to denote a specific
module of a library. Typically, {\tt Coq.Init.Logic} is the logical name
associated to the file {\tt Logic.v} of {\Coq} standard library.

If the physical directory where a file {\tt file.v} lies is mapped to
the empty logical directory path (which is the default when using the
simple form of {\tt Add LoadPath} or {\tt -I} option to coqtop), then
the name of the module it defines is simply {\tt file}.

\paragraph{Constructions names}

Finally, modules contain constructions (axioms, parameters,
definitions, lemmas, theorems, remarks or facts). The (full) name of a
construction starts with the name of the module in which it is defined
followed by the (short) name of the construction.
Typically, {\tt Coq.Init.Logic.eq} denotes Leibniz' equality
defined in the module {\tt Logic} in the sublibrary {\tt Init} of the
standard library of \Coq.

\paragraph{Absolute and short names}

The full name of a library, module, section, definition, theorem,
... is called {\it absolute name}. 
The final identifier (in the example
above, it is {\tt eq}) is called {\it short name} (or sometimes {\it
base name}.  {\Coq} maintains a {\it names table} mapping short names
to absolute names. This greatly simplifies the notations.  {\Coq}
cannot accept two constructions (definition, theorem, ...) with the
same absolute name.

\paragraph{The special case of remarks and facts}

In contrast with definitions, lemmas, theorems, axioms and parameters,
the absolute name of remarks includes the segment of sections in which
it is defined. Concretely, if a remark {\tt R} is defined in
subsection {\tt S2} of section {\tt S1} in module {\tt M}, then its
absolute name is {\tt M.S1.S2.R}. The same for facts, except that the
name of the innermost section is dropped from the full name. Then, if
a fact {\tt F} is defined in subsection {\tt S2} of section {\tt S1}
in module {\tt M}, then its absolute name is {\tt M.S1.F}.

\paragraph{Visibility and qualified names}
An absolute path is called {\it visible} when its base name suffices
to denote it. This means the base name is mapped to the absolute name
in {\Coq} name table.

All the base names of sublibraries, modules, sections, definitions and
theorems are automatically put in the {\Coq} name table. But
sometimes, the short name of a construction defined in a module may
hide the short name of another construction defined in another module.
Instead of just distinguishing the clashing names by using the
absolute names, it is enough to prefix the base name just by the name
of the module in which the definition, theorem, ... is defined.
% Remark: when modules will be available, longer qualification may be needed
Such a name built from single identifiers separated by dots is called
a {\em partially qualified name} (or shortly a {\em qualified name},
written {\qualid}). Especially, both absolute names and short names
are qualified names. To ensure that a construction always remains
accessible, absolute names can never be hidden.

Examples:

\begin{coq_eval}
Reset Initial.
\end{coq_eval}

\begin{coq_example}
Check O.
Definition nat := bool.
Check O.
Check Datatypes.nat.
\end{coq_example}

\paragraph{Requiring a file}

A module compiled in a ``.vo'' file comes with a logical names (e.g. 
physical file \verb!theories/Init/Datatypes.vo! in {\Coq} installation directory contains logical module {\tt Coq.Init.Datatypes}). 
When requiring the file, the mapping between physical directories and logical library should be consistent with the mapping used to compile the file (for modules of the standard library, this is automatic -- check it by typing {\tt Print LoadPath}).

The command {\tt Add Rec LoadPath} is also available from {\tt coqtop}
and {\tt coqc} by using option~\verb=-R= (see section \ref{Tools}).

\section{Implicit arguments}\index{implicit arguments}

The {\Coq} system allows to skip during a function application certain
arguments that can be automatically inferred from the other
arguments. Such arguments are called {\em implicit}. Typical implicit
arguments are the type arguments in polymorphic functions.

The user can force a subterm to be guessed by replacing it by
{\tt ?}. If possible, the correct subterm will be automatically generated.

\ErrMsg
\begin{enumerate}
\item \errindex{There is an unknown subterm I cannot solve} \\
  {\Coq} was not able to deduce an instantiation of a ``?''.
\end{enumerate}

In addition, there are two ways to systematically avoid to write
``{\tt ?}'' where a term can be automatically inferred.

The first mode is automatic. Switching to this mode forces some
easy-to-infer subterms to always be implicit.
The command to use the second mode is  {\tt Syntactic
Definition}.

\subsection{Auto-detection of implicit arguments}
\label{Auto-implicit}

There is an automatic mode to declare as implicit some arguments of
constants and variables which have a functional type. In this mode,
to every declared object (even inductive types and theirs constructors) is
associated the list of the positions of its implicit arguments. These
implicit arguments correspond to the arguments which can be deduced
from the following ones. Thus when one applies these functions to
arguments, one can omit the implicit ones. They are then automatically
replaced by symbols ``?'', to be inferred by the mechanism of
synthesis of implicit arguments.

\subsubsection{\tt Set Implicit Arguments.}
\comindex{Set Implicit Arguments}
\label{Implicit Arguments}

This command switches the automatic implicit arguments
mode on. To switch it off, use {\tt Unset Implicit Arguments.}.
The mode is off by default.

The computation of implicit arguments takes account of the
unfolding of constants.  For instance, the variable {\tt p} below has
a type {\tt (Transitivity R)} which is reducible to {\tt (x,y:U)(R x
y) -> (z:U)(R y z) -> (R x z)}. As the variables {\tt x}, {\tt y} and
{\tt z} appear in the body of the type, they are said implicit; they
correspond respectively to the positions {\tt 1}, {\tt 2} and {\tt 4}.

\begin{coq_example*}
Set Implicit Arguments.
Variable X : Type.
Definition Relation := X -> X -> Prop.
Definition Transitivity := [R:Relation]
           (x,y:X)(R x y) -> (z:X)(R y z) -> (R x z).
Variables R:Relation; p:(Transitivity R).
\end{coq_example*}
\begin{coq_example}
Print p.
\end{coq_example}
\begin{coq_example*}
Variables a,b,c:X;  r1:(R a b); r2:(R b c).
\end{coq_example*}
\begin{coq_example}
Check (p r1 r2).
\end{coq_example}

\subsubsection{Explicit Applications}

The mechanism of synthesis of implicit arguments is not complete, so
we have sometimes to give explicitly certain implicit arguments of an
application. The syntax is {\tt $i$!}{\term} where $i$ is the position
of an implicit argument and {\term} is its corresponding explicit
term. The number $i$ is called {\em explicitation number}.  We can
also give all the arguments of an application, we have then to write
{\tt (!{\ident}~{\term}$_1$..{\term}$_n$)}.

\ErrMsg
\begin{enumerate}
\item \errindex{Bad explicitation number}
\end{enumerate}

\Example

\begin{coq_example}
Check (p r1 4!c).
Check (!p a b r1 c r2).
\end{coq_example}

\subsubsection{Implicit Arguments and Pretty-Printing}

The basic pretty-printing rules hide the implicit arguments of an
application. However an implicit argument {\term} of an application
which is not followed by any explicit argument is printed as follows
$i!${\term} where $i$ is its position.

\subsection{User-defined implicit arguments: {\tt Syntactic definition}}
\comindex{Syntactic Definition}
\label{Syntactic-Definition}

The syntactic definitions define syntactic constants, i.e. give a name
to a term possibly untyped but syntactically correct. Their syntax
is:

\begin{center}
\verb+Syntactic Definition+ $name$ \verb+:=+ $term$ \verb+.+ \\
\end{center}

Syntactic definitions behave like macros: every occurrence of a
syntactic constant in an expression is immediately replaced by its
body.

Let us extend our functional language with the definition of the
identity function:

\begin{coq_eval}
Unset Implicit Arguments.
Reset Initial.
\end{coq_eval}
\begin{coq_example}
Definition explicit_id := [A:Set][a:A]a.
\end{coq_example}

\index{questionmark@{\texttt{?}}}
We declare also a syntactic definition {\tt id}:

\begin{coq_example}
Syntactic Definition id := (explicit_id ?).
\end{coq_example}

The term {\tt (explicit\_id ?)} is untyped since the implicit
arguments cannot be synthesized. There is no type check during this
definition.  Let us see what happens when we use a syntactic constant
in an expression like in the following example.

\begin{coq_example}
Check (id O).
\end{coq_example}

\noindent First the syntactic constant {\tt id} is replaced by its
body {\tt (explicit\_id ?)} in the expression. Then the resulting
expression is evaluated by the typechecker, which fills in
``\verb+?+'' place-holders.

The standard usage of syntactic definitions is to give names to terms
applied to implicit arguments ``\verb+?+''. In this case, a special
command is provided:

\begin{center}
\verb+Syntactic Definition+ $name$ \verb+:=+ $term$ \verb+|+ $n$ \verb+.+ \\
\end{center}

\noindent The body of the syntactic constant is $term$ applied to $n$
place-holders ``\verb+?+''.

We can define a new syntactic definition {\tt id1} for {\tt
  explicit\_id} using this command. We changed the name of the
syntactic constant in order to avoid a name conflict with {\tt id}.

\begin{coq_example}
Syntactic Definition id1 := explicit_id | 1.
\end{coq_example}

The new syntactic constant {\tt id1} has the same behavior as {\tt
  id}:

\begin{coq_example}
Check (id1 O).
\end{coq_example}


\begin{Warnings}
\item Syntactic constants defined inside a section are no longer
  available after closing the section.
\item You cannot see the body of a syntactic constant with a {\tt
    Print} command.
\end{Warnings}

\section{Implicit Coercions}
\label{Coercions}\index{Coercions}

Coercions can be used to implicitly inject terms from one ``class'' in
which they reside into another one. A {\em class} is either a sort
(denoted by the keyword SORTCLASS), a product type (denoted by the
keyword FUNCLASS) or an inductive type (denoted by its name).

Then the user is able to apply an
object that is not a function, but can be coerced to a function, and
more generally to consider that a term of type A is of type B provided
that there is a declared coercion between A and B.

\subsection{\tt Class {\ident}.}\comindex{Class}
Declares the name {\ident} as a new class.

\begin{Variant}
\item {\tt Class Local {\ident}.} \\
Declares the name {\ident} as a new local class to the current section.
\end{Variant}

\subsection{\tt Coercion {\ident} : {\ident$_1$} >-> {\ident$_2$}.}
\comindex{Coercion}

Declares the name {\ident} as a coercion between {\ident$_1$} and
{\ident$_2$}. The classes {\ident$_1$} and {\ident$_2$} are first
declared if necessary.

\begin{Variants}
\item {\tt Coercion Local {\ident} : {\ident$_1$} >-> {\ident$_2$}.}\comindex{Coercion Local}\\
Declares the name {\ident} as a coercion local to the current section.

\item {\tt Identity Coercion {\ident}:{\ident$_1$} >->
    {\ident$_2$}.}\comindex{Identity Coercion}\\
Coerce an inductive type to a subtype of it.

\item {\tt Identity Coercion Local {\ident}:{\ident$_1$} >-> {\ident$_2$}.} \\
Idem but locally to the current section.

\item {\tt Coercion {\ident} := {\term}}\comindex{Coercion}\\
  This defines {\ident} just like \texttt{Definition {\ident} :=
    {\term}}, and then declares {\ident} as a coercion between it
  source and its target.

\item {\tt Coercion {\ident} := {\term} : {\type}}\\
  This defines {\ident} just like 
  \texttt{Definition {\ident} : {\type} := {\term}}, and then
  declares {\ident} as a coercion between it source and its target. 

\item {\tt Coercion Local {\ident} := {\term}}\comindex{Local Coercion}\\
  This defines {\ident} just like \texttt{Local {\ident} :=
    {\term}}, and then declares {\ident} as a coercion between it
  source and its target.

\end{Variants}

\SeeAlso the technical chapter \ref{Coercions-full} on coercions.

\subsection{Displaying available coercions}

\subsubsection{\tt Print Classes.}\comindex{Print Classes}
Print the list of declared classes in the current context.

\subsubsection{\tt Print Coercions.}\comindex{Print Coercions}
Print the list of declared coercions in the current context.

\subsubsection{\tt Print Graph.}\comindex{Print Graph}
Print the list of valid path coercions in the current context.

\subsection{Activating the printing of coercions}

\subsubsection{\tt Set Printing Coercions.}
\comindex{Set Printing Coercions}
\comindex{Unset Printing Coercions}

This command forces the coercions to be printed.
To skip the printing of coercions, use
 {\tt Unset Printing Coercions.}.
By default, coercions are not printed.

%%% Local Variables: 
%%% mode: latex
%%% TeX-master: "Reference-Manual"
%%% TeX-master: "Reference-Manual"
%%% End: 
