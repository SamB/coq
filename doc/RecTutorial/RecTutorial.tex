\documentclass[11pt]{article}
\title{A Tutorial on [Co-]Inductive Types in Coq}
\author{Eduardo Gim\'enez\thanks{Eduardo.Gimenez@inria.fr},
Pierre Cast\'eran\thanks{Pierre.Casteran@labri.fr}}
\date{May 1998 --- \today}

\usepackage{multirow}
\usepackage{aeguill}
%\externaldocument{RefMan-gal.v}
%\externaldocument{RefMan-ext.v}
%\externaldocument{RefMan-tac.v}
%\externaldocument{RefMan-oth}
%\externaldocument{RefMan-tus.v}
%\externaldocument{RefMan-syn.v}
%\externaldocument{Extraction.v}
%===================================
% Style of the document
%===================================
%\newtheorem{example}{Example}[section]
%\newtheorem{exercise}{Exercise}[section]


\newcommand{\comentario}[1]{\texttt{#1}}

%===================================
% Keywords
%===================================

\newcommand{\Prop}{\texttt{Prop}}
\newcommand{\Set}{\texttt{Set}}
\newcommand{\Type}{\texttt{Type}}
\newcommand{\true}{\texttt{true}}
\newcommand{\false}{\texttt{false}}
\newcommand{\Lth}{\texttt{Lth}}

\newcommand{\Nat}{\texttt{nat}}
\newcommand{\nat}{\texttt{nat}}
\newcommand{\Z}  {\texttt{O}}
\newcommand{\SUCC}{\texttt{S}}
\newcommand{\pred}{\texttt{pred}}

\newcommand{\False}{\texttt{False}}
\newcommand{\True}{\texttt{True}}
\newcommand{\I}{\texttt{I}}

\newcommand{\natind}{\texttt{nat\_ind}}
\newcommand{\natrec}{\texttt{nat\_rec}}
\newcommand{\natrect}{\texttt{nat\_rect}}

\newcommand{\eqT}{\texttt{eqT}}
\newcommand{\identityT}{\texttt{identityT}}

\newcommand{\map}{\texttt{map}}
\newcommand{\iterates}{\texttt{iterates}}


%===================================
% Numbering
%===================================


\newtheorem{definition}{Definition}[section]
\newtheorem{example}{Example}[section]


%===================================
% Judgements 
%===================================


\newcommand{\JM}[2]{\ensuremath{#1 : #2}}

%===================================
% Expressions 
%===================================

\newcommand{\Case}[3][]{\ensuremath{#1\textsf{Case}~#2~\textsf of}~#3~\textsf{end}}

%=======================================

\newcommand{\snreglados} [3] {\begin{tabular}{c} \ensuremath{#1} \\[2pt]
   \ensuremath{#2}\\ \hline  \ensuremath{#3}  \end{tabular}}


\newcommand{\snregla} [2] {\begin{tabular}{c}
   \ensuremath{#1}\\ \hline  \ensuremath{#2}  \end{tabular}}


%=======================================


\usepackage{url}

\newcommand{\variantspringer}[1]{#1}
\newcommand{\marginok}[1]{\marginpar{\raggedright OK:#1}}
\newcommand{\tab}{{\null\hskip1cm}}
\newcommand{\Ltac}{\mbox{\emph{$\cal L$}tac}}
\newcommand{\coq}{\mbox{\emph{Coq}}}
\newcommand{\lcf}{\mbox{\emph{LCF}}}
\newcommand{\hol}{\mbox{\emph{HOL}}}
\newcommand{\pvs}{\mbox{\emph{PVS}}}
\newcommand{\isabelle}{\mbox{\emph{Isabelle}}}
\newcommand{\prolog}{\mbox{\emph{Prolog}}}
\newcommand{\goalbar}{\tt{}============================\it}
\newcommand{\gallina}{\mbox{\emph{Gallina}}}
\newcommand{\joker}{\texttt{\_}}
\newcommand{\eprime}{\(\e^{\prime}\)}
\newcommand{\Ztype}{\citecoq{Z}}
\newcommand{\propsort}{\citecoq{Prop}}
\newcommand{\setsort}{\citecoq{Set}}
\newcommand{\typesort}{\citecoq{Type}}
\newcommand{\ocaml}{\mbox{\emph{OCAML}}}
\newcommand{\haskell}{\mbox{\emph{Haskell}}}
\newcommand{\why}{\mbox{\emph{Why}}}
\newcommand{\Pascal}{\mbox{\emph{Pascal}}}

\newcommand{\ml}{\mbox{\emph{ML}}}

\newcommand{\scheme}{\mbox{\emph{Scheme}}}
\newcommand{\lisp}{\mbox{\emph{Lisp}}}

\newcommand{\implarrow}{\mbox{$\Rightarrow$}}
\newcommand{\metavar}[1]{?#1}
\newcommand{\notincoq}[1]{#1}
\newcommand{\coqscope}[1]{\%#1}
\newcommand{\arrow}{\mbox{$\rightarrow$}}
\newcommand{\fleche}{\arrow}
\newcommand{\funarrow}{\mbox{$\Rightarrow$}}
\newcommand{\ltacarrow}{\funarrow}
\newcommand{\coqand}{\mbox{\(\wedge\)}}
\newcommand{\coqor}{\mbox{\(\vee\)}}
\newcommand{\coqnot}{\mbox{\(\neg\)}}
\newcommand{\hide}[1]{}
\newcommand{\hidedots}[1]{...}
\newcommand{\sig}[3]{\texttt{\{}#1\texttt{:}#2 \texttt{|} #3\texttt{\}}}
\renewcommand{\neg}{\sim}
\renewcommand{\marginpar}[1]{}

\addtocounter{secnumdepth}{1}
\providecommand{\og}{�}
\providecommand{\fg}{�}


\newcommand{\hard}{\mbox{\small *}}
\newcommand{\xhard}{\mbox{\small **}}
\newcommand{\xxhard}{\mbox{\small ***}}

%%% Operateurs, etc.
\newcommand{\impl}{\mbox{$\rightarrow$}}
\newcommand{\appli}[2]{\mbox{\tt{#1 #2}}}
\newcommand{\applis}[1]{\mbox{\texttt{#1}}}
\newcommand{\abst}[3]{\mbox{\tt{fun #1:#2 \funarrow #3}}}
\newcommand{\coqle}{\mbox{$\leq$}}
\newcommand{\coqge}{\mbox{$\geq$}}
\newcommand{\coqdiff}{\mbox{$\neq$}}
\newcommand{\coqiff}{\mbox{$\leftrightarrow$}}
\newcommand{\prodsym}{\mbox{\(\forall\,\)}}
\newcommand{\exsym}{\mbox{\(\exists\,\)}}

\newcommand{\substsign}{/}
\newcommand{\subst}[3]{\mbox{#1\{#2\substsign{}#3\}}}
\newcommand{\anoabst}[2]{\mbox{\tt[#1]#2}}
\newcommand{\letin}[3]{\mbox{\tt let #1:=#2 in #3}}
\newcommand{\prodep}[3]{\mbox{\tt \(\forall\,\)#1:#2,$\,$#3}}
\newcommand{\prodplus}[2]{\mbox{\tt\(\forall\,\)$\,$#1,$\,$#2}}
\newcommand{\dom}[1]{\textrm{dom}(#1)} % domaine d'un contexte (log function)
\newcommand{\norm}[1]{\textrm{n}(#1)} % forme normale (log function)
\newcommand{\coqZ}[1]{\mbox{\tt{`#1`}}}
\newcommand{\coqnat}[1]{\mbox{\tt{#1}}}
\newcommand{\coqcart}[2]{\mbox{\tt{#1*#2}}}
\newcommand{\alphacong}{\mbox{$\,\cong_{\alpha}\,$}} % alpha-congruence
\newcommand{\betareduc}{\mbox{$\,\rightsquigarrow_{\!\beta}$}\,} % beta reduction
%\newcommand{\betastar}{\mbox{$\,\Rightarrow_{\!\beta}^{*}\,$}} % beta reduction
\newcommand{\deltareduc}{\mbox{$\,\rightsquigarrow_{\!\delta}$}\,} % delta reduction
\newcommand{\dbreduc}{\mbox{$\,\rightsquigarrow_{\!\delta\beta}$}\,} % delta,beta reduction
\newcommand{\ireduc}{\mbox{$\,\rightsquigarrow_{\!\iota}$}\,} % delta,beta reduction


% jugement de typage
\newcommand{\these}{\boldsymbol{\large \vdash}}
\newcommand{\disj}{\mbox{$\backslash/$}}
\newcommand{\conj}{\mbox{$/\backslash$}}
%\newcommand{\juge}[3]{\mbox{$#1 \boldsymbol{\vdash} #2 : #3 $}}
\newcommand{\juge}[4]{\mbox{$#1,#2 \these #3 \boldsymbol{:} #4 $}}
\newcommand{\smalljuge}[3]{\mbox{$#1 \these #2 \boldsymbol{:} #3 $}}
\newcommand{\goal}[3]{\mbox{$#1,#2 \these^{\!\!\!?} #3  $}}
\newcommand{\sgoal}[2]{\mbox{$#1\these^{\!\!\!\!?} #2 $}}
\newcommand{\reduc}[5]{\mbox{$#1,#2 \these #3 \rhd_{#4}#5 $}}
\newcommand{\convert}[5]{\mbox{$#1,#2 \these #3 =_{#4}#5 $}}
\newcommand{\convorder}[5]{\mbox{$#1,#2 \these #3\leq _{#4}#5 $}}
\newcommand{\wouff}[2]{\mbox{$\emph{WF}(#1)[#2]$}}


%\newcommand{\mthese}{\underset{M}{\vdash}}
\newcommand{\mthese}{\boldsymbol{\vdash}_{\!\!M}}
\newcommand{\type}{\boldsymbol{:}}

% jugement absolu

%\newcommand{\ajuge}[2]{\mbox{$ \boldsymbol{\vdash} #1 : #2 $}}
\newcommand{\ajuge}[2]{\mbox{$\these #1 \boldsymbol{:} #2 $}}

%%% logique minimale
\newcommand{\propzero}{\mbox{$P_0$}} % types de Fzero

%%% logique propositionnelle classique
\newcommand {\ff}{\boldsymbol{f}} % faux
\newcommand {\vv}{\boldsymbol{t}} % vrai

\newcommand{\verite}{\mbox{$\cal{B}$}} % {\ff,\vv}
\newcommand{\sequ}[2]{\mbox{$#1 \vdash #2 $}} % sequent
\newcommand{\strip}[1]{#1^o} % enlever les variables d'un contexte



%%% tactiques
\newcommand{\decomp}{\delta} % decomposition
\newcommand{\recomp}{\rho} % recomposition

%%% divers
\newcommand{\cqfd}{\mbox{\textbf{cqfd}}}
\newcommand{\fail}{\mbox{\textbf{F}}}
\newcommand{\succes}{\mbox{$\blacksquare$}}
%%% Environnements


%% Fzero
\newcommand{\con}{\mbox{$\cal C$}}
\newcommand{\var}{\mbox{$\cal V$}}

\newcommand{\atomzero}{\mbox{${\cal A}_0$}} % types de base de Fzero
\newcommand{\typezero}{\mbox{${\cal T}_0$}} % types de Fzero
\newcommand{\termzero}{\mbox{$\Lambda_0$}} % termes de Fzero 
\newcommand{\conzero}{\mbox{$\cal C_0$}} % contextes de Fzero 

\newcommand{\buts}{\mbox{$\cal B$}} % buts

%%% for drawing terms
% abstraction [x:t]e
\newcommand{\PicAbst}[3]{\begin{bundle}{\bf abst}\chunk{#1}\chunk{#2}\chunk{#3}%
                        \end{bundle}}

% the same in DeBruijn form
\newcommand{\PicDbj}[2]{\begin{bundle}{\bf abst}\chunk{#1}\chunk{#2}
                       \end{bundle}}


% applications
\newcommand{\PicAppl}[2]{\begin{bundle}{\bf appl}\chunk{#1}\chunk{#2}%
                         \end{bundle}}

% variables
\newcommand{\PicVar}[1]{\begin{bundle}{\bf var}\chunk{#1}
                          \end{bundle}}

% constantes
\newcommand{\PicCon}[1]{\begin{bundle}{\bf const}\chunk{#1}\end{bundle}}

% arrows
\newcommand{\PicImpl}[2]{\begin{bundle}{\impl}\chunk{#1}\chunk{#2}%
                         \end{bundle}}



%%%% scripts coq
\newcommand{\prompt}{\mbox{\sl Coq $<\;$}}
\newcommand{\natquicksort}{\texttt{nat\_quicksort}}
\newcommand{\citecoq}[1]{\mbox{\texttt{#1}}}
\newcommand{\safeit}{\it}
\newtheorem{remarque}{Remark}[section]
%\newtheorem{definition}{Definition}[chapter]

\newcommand{\refmancite}[1]{{}}
%\newcommand{\refmancite}[1]{\cite{coqrefman}}
%\newcommand{\refmancite}[1]{\cite[#1] {]{coqrefman}}

\usepackage[latin1]{inputenc}
\usepackage[T1]{fontenc}
\usepackage{makeidx}
%\usepackage{multind}
\usepackage{alltt}   
\usepackage{verbatim}
\usepackage{amssymb}
\usepackage{amsmath}
\usepackage{theorem}
\usepackage[dvips]{epsfig}
\usepackage{epic}
\usepackage{eepic}
\usepackage{ecltree}
\usepackage{moreverb} 
\usepackage{color}
\usepackage{pifont}
\usepackage{xr}
\usepackage{url}

\usepackage{alltt}
\renewcommand{\familydefault}{ptm}
\renewcommand{\seriesdefault}{m}
\renewcommand{\shapedefault}{n}
\newtheorem{exercise}{Exercise}[section]	
\makeindex
\begin{document}
\maketitle

\begin{abstract}
This document\footnote{The first versions of this document were entirely written by Eduardo Gimenez.
Pierre Cast\'eran wrote the 2004 and 2006 revisions.} is an introduction to the definition and
use of inductive and co-inductive  types in the {\coq} proof environment. It explains how types like natural numbers and infinite streams are defined
in {\coq}, and the kind of proof techniques that can be used to reason
about them (case analysis, induction, inversion of predicates,
co-induction, etc).  Each technique is illustrated through an
executable and self-contained {\coq} script.  
\end{abstract}
%\RRkeyword{Proof environments, recursive types.}
%\makeRT

\addtocontents{toc}{\protect \thispagestyle{empty}}
\pagenumbering{arabic}

\cleardoublepage
\tableofcontents
\clearpage

\section{About this document}

This document is an introduction to the definition and use of
inductive and co-inductive  types in the {\coq} proof environment.  It was born from the
notes written for the course about the version V5.10 of {\coq}, given
by Eduardo Gimenez  at
the Ecole Normale Sup\'erieure de Lyon in March 1996. This article is
a revised and improved version of these notes for the version V8.0 of
the system.


We assume that the reader has some familiarity with the
proofs-as-programs paradigm of Logic \cite{Coquand:metamathematical} and the generalities
of the {\coq} system \cite{coqrefman}.  You would take a greater advantage of
this document if you first read the general tutorial about {\coq} and
{\coq}'s FAQ, both available on \cite{coqsite}.
A text book \cite{coqart},  accompanied with a lot of
examples and exercises \cite{Booksite}, presents a detailed description
of the {\coq} system and its underlying
formalism: the Calculus of Inductive Construction.
Finally, the complete description of {\coq}  is given in the reference manual
\cite{coqrefman}. Most of the tactics and commands we describe have
several options, which we do not present exhaustively. 
If some script herein uses a non described feature, please refer to
the Reference Manual.


If you are familiar with other proof environments
based on type theory and the LCF style ---like PVS, LEGO, Isabelle,
etc--- then you will find not difficulty to guess the unexplained
details.

The better way to read this document is to start up the {\coq} system,
type by yourself the examples and exercises, and observe the
behavior of the system. All the examples proposed in this tutorial
can be downloaded from the same site as the present document. 


The tutorial is organised as follows. The next section describes how
inductive types are defined in {\coq}, and introduces some useful ones,
like natural numbers, the empty type, the propositional equality type,
and the logical connectives. Section \ref{CaseAnalysis} explains
definitions by pattern-matching and their connection with the
principle of case analysis. This principle is the most basic
elimination rule associated with inductive or co-inductive types
 and follows a
general scheme that we illustrate for some of the types introduced in
Section \ref{Introduction}.  Section \ref{CaseTechniques} illustrates
the pragmatics of this principle, showing different proof techniques
based on it. Section \ref{StructuralInduction} introduces definitions
by structural recursion and proofs by induction. 
Section~\ref{CaseStudy} presents some elaborate techniques
about dependent case analysis. Finally, Section
\ref{CoInduction} is a brief introduction to co-inductive types
--i.e., types containing infinite objects-- and the principle of
co-induction.


Thanks to  Bruno Barras, Yves Bertot, Hugo Herbelin, Jean-Fran\c{c}ois Monin
and Michel L\'evy for their help.

\subsection*{Lexical conventions}
The \texttt{typewriter} font is used to represent text
input by the user, while the \textit{italic} font is used to represent
the text output by the system as answers.  


Moreover, the mathematical symbols \coqle{}, \coqdiff, \(\exists\),
\(\forall\), \arrow{}, $\rightarrow{}$ \coqor{}, \coqand{}, and \funarrow{} 
stand for the character strings \citecoq{<=}, \citecoq{<>},
\citecoq{exists}, \citecoq{forall}, \citecoq{->}, \citecoq{<-},
\texttt{\char'134/}, \texttt{/\char'134}, and \citecoq{=>},
respectively.  For instance, the \coq{} statement
%V8 A prendre
% inclusion numero 1
% traduction numero 1
\begin{alltt}
\hide{Open Scope nat_scope. Check (}forall A:Type,(exists x : A, forall (y:A), x <> y) -> 2 = 3\hide{).}
\end{alltt}
is written as follows in this tutorial:
%V8 A prendre
% inclusion numero 2
% traduction numero 2
\begin{alltt}
\hide{Check (}{\prodsym}A:Type,(\exsym{}x:A, {\prodsym}y:A, x {\coqdiff} y) \arrow{} 2 = 3\hide{).}
\end{alltt}

When a fragment of \coq{} input text appears in the middle of
regular text, we often place this fragment between double quotes
``\dots.''  These double quotes do not belong to the \coq{} syntax.

Finally, any
string enclosed between \texttt{(*} and \texttt{*)} is a comment and
is ignored by the \coq{} system.

\section{Introducing Inductive Types} 
\label{Introduction}

Inductive types are types closed with respect to their introduction
rules. These rules explain the most basic or \textsl{canonical} ways
of constructing an element of the type.  In this sense, they
characterize the recursive type. Different rules must be considered as
introducing different objects. In order to fix ideas, let us introduce
in {\coq} the most well-known example of a recursive type: the type of
natural numbers.

%V8 A prendre
\begin{alltt}
Inductive nat : Set := 
 | O : nat 
 | S : nat\arrow{}nat.
\end{alltt}

The definition of a recursive type has two main parts. First, we
establish what kind of recursive type we will characterize (a set, in
this case). Second, we present the introduction rules that define the
type ({\Z} and {\SUCC}), also called its {\sl constructors}. The constructors
{\Z} and {\SUCC} determine all the elements of this type. In other
words, if $n\mbox{:}\nat$, then $n$ must have been introduced either
by the rule {\Z} or by an application of the rule {\SUCC} to a
previously constructed natural number. In this sense, we can say
that {\nat} is \emph{closed}. On the contrary, the type
$\Set$ is an {\it open} type, since we do not know {\it a priori} all
the possible ways of introducing an object of type \texttt{Set}.

After entering this command, the constants {\nat}, {\Z} and {\SUCC} are
available in the current context. We can see their types using the
\texttt{Check} command \refmancite{Section \ref{Check}}:

%V8 A prendre 
\begin{alltt}
Check nat.
\it{}nat : Set
\tt{}Check O.
\it{}O : nat
\tt{}Check S.
\it{}S : nat {\arrow} nat
\end{alltt}

Moreover, {\coq} adds to the context three constants named
 $\natind$, $\natrec$ and $\natrect$, which
 correspond to different principles of structural induction on
natural numbers that {\coq} infers automatically from the definition.  We
will come back to them in Section \ref{StructuralInduction}.


In fact, the type of natural numbers as well as several useful
theorems about them are already defined in the basic library of {\coq},
so there is no need to introduce them.  Therefore, let us throw away
our (re)definition of {\nat}, using the command \texttt{Reset}.

%V8 A prendre
\begin{alltt}
Reset nat.
Print nat.
\it{}Inductive nat : Set :=  O : nat | S : nat \arrow{} nat
For S: Argument scope is [nat_scope]
\end{alltt}

Notice that \coq{}'s \emph{interpretation scope} for natural numbers
(called \texttt{nat\_scope}) 
allows us to read and write natural numbers in decimal form (see \cite{coqrefman}). For instance, the constructor \texttt{O} can be read or written
as the digit $0$, and the term ``~\texttt{S (S (S O))}~'' as $3$.

%V8 A prendre
\begin{alltt}
Check O.
\it 0 : nat.
\tt
Check (S (S (S O))).
\it 3 : nat
\end{alltt}

Let us now take a look to some other
recursive types contained in the standard library of {\coq}.

\subsection{Lists}
Lists are defined in library \citecoq{List}\footnote{Notice that in versions of 
{\coq}
prior to 8.1, the parameter $A$ had sort \citecoq{Set} instead of \citecoq{Type}; 
the constant \citecoq{list} was thus of type \citecoq{Set\arrow{} Set}.}


\begin{alltt}
Require Import List.
Print list.
\it
Inductive list (A : Type) : Type:=
    nil : list A | cons : A {\arrow} list A {\arrow} list A
For nil: Argument A is implicit
For cons: Argument A is implicit
For list: Argument scope is [type_scope]
For nil: Argument scope is [type_scope]
For cons: Argument scopes are [type_scope _ _]
\end{alltt}

In this definition, \citecoq{A} is a \emph{general parameter}, global
to both constructors.
This kind of definition allows us to build a whole family of
inductive types, indexed over the sort \citecoq{Type}.
This can be observed if we consider the type of identifiers
\citecoq{list}, \citecoq{cons} and \citecoq{nil}.
Notice the notation \citecoq{(A := \dots)} which must be used 
when {\coq}'s type inference algorithm cannot infer the implicit
parameter \citecoq{A}.
\begin{alltt}
Check list.
\it list
     : Type {\arrow} Type

\tt Check (nil (A:=nat)).
\it nil
     : list nat

\tt Check (nil (A:= nat {\arrow} nat)).
\it nil
     : list (nat {\arrow} nat)

\tt Check (fun A: Type {\funarrow} (cons (A:=A))).
\it fun A : Type {\funarrow} cons (A:=A)
     : {\prodsym} A : Type, A {\arrow} list A {\arrow} list A

\tt Check (cons 3 (cons 2 nil)).
\it 3 :: 2 :: nil
     : list nat

\tt Check (nat :: bool ::nil).
\it nat :: bool :: nil
     : list Set

\tt Check ((3<=4) :: True ::nil).
\it (3<=4) :: True :: nil
     : list Prop

\tt Check (Prop::Set::nil).
\it Prop::Set::nil
     : list Type
\end{alltt}

\subsection{Vectors.}
\label{vectors}

Like \texttt{list}, \citecoq{vector} is a polymorphic type:
if $A$ is a type, and $n$ a natural number, ``~\citecoq{vector $A$ $n$}~''
is the type of vectors of elements of $A$ and size $n$.


\begin{alltt}
Require Import  Bvector.

Print vector.
\it
Inductive vector (A : Type) : nat {\arrow} Type :=
    Vnil : vector A 0
  | Vcons : A {\arrow} {\prodsym} n : nat, vector A n {\arrow} vector A (S n)
For vector: Argument scopes are [type_scope nat_scope]
For Vnil: Argument scope is [type_scope]
For Vcons: Argument scopes are [type_scope _ nat_scope _]
\end{alltt}


Remark the difference between the two parameters $A$ and $n$:
The first one is a \textsl{general parameter}, global to all the
introduction rules,while the second one is an \textsl{index}, which is
instantiated differently in the introduction rules.
Such types parameterized  by regular
values are called \emph{dependent types}.

\begin{alltt}
Check (Vnil nat).
\it Vnil nat
     : vector nat 0

\tt Check (fun (A:Type)(a:A){\funarrow} Vcons _ a _ (Vnil _)).
\it fun (A : Type) (a : A) {\funarrow} Vcons A a 0 (Vnil A)
     : {\prodsym} A : Type, A {\arrow} vector A 1


\tt Check (Vcons _ 5 _ (Vcons _ 3 _ (Vnil _))).
\it Vcons nat 5 1 (Vcons nat 3 0 (Vnil nat))
     : vector nat 2
\end{alltt}

\subsection{The contradictory proposition.}
Another example of an inductive type is the contradictory proposition.
This type inhabits the universe of propositions, and has no element
at all.
%V8 A prendre
\begin{alltt}
Print False.
\it{} Inductive False : Prop :=
\end{alltt}

\noindent Notice that no constructor is given in this definition.

\subsection{The tautological proposition.}
Similarly, the
tautological proposition {\True} is defined as an inductive type
with only one element {\I}:

%V8 A prendre
\begin{alltt}
Print True.
\it{}Inductive True : Prop :=  I : True
\end{alltt}

\subsection{Relations as inductive types.}
Some relations can also be introduced in a smart way as an inductive family
of propositions. Let us take as example the order $n \leq m$ on natural
numbers, called \citecoq{le} in {\coq}.
 This relation is introduced through
the following definition, quoted from the standard library\footnote{In the interpretation scope
for Peano arithmetic:
\citecoq{nat\_scope},  ``~\citecoq{n <= m}~'' is equivalent to 
``~\citecoq{le n m}~'' .}:




%V8 A prendre
\begin{alltt}
Print le. \it
Inductive le (n:nat) : nat\arrow{}Prop := 
|  le_n: n {\coqle} n 
|  le_S: {\prodsym} m, n {\coqle} m \arrow{} n {\coqle} S m.
\end{alltt}

Notice that in this definition $n$ is a general parameter,
while the second argument of \citecoq{le} is an index (see section
~\ref{vectors}).
 This definition
introduces the binary relation $n {\leq} m$ as the family of unary predicates
``\textsl{to be greater or equal than a given $n$}'', parameterized by $n$.

The introduction rules of this type can be seen as a sort of Prolog
rules for proving that a given integer $n$ is less or equal than another one.
In fact, an object of type $n{\leq} m$ is nothing but a proof 
built up using the constructors \textsl{le\_n} and
\textsl{le\_S} of this type.  As an example, let us construct
a proof that zero is less or equal than three using {\coq}'s interactive
proof mode.
Such an object can be obtained applying three times the second
introduction rule of \citecoq{le}, to a proof that zero is less or equal
than itself,
which is provided by the first constructor of \citecoq{le}:

%V8 A prendre
\begin{alltt}
Theorem zero_leq_three: 0 {\coqle} 3.
Proof.
\it{} 1 subgoal

============================
 0 {\coqle} 3

\tt{}Proof.
 constructor 2. 

\it{} 1 subgoal
============================
  0 {\coqle} 2

\tt{} constructor 2.  
\it{} 1 subgoal
============================
  0 {\coqle} 1

\tt{} constructor 2
\it{} 1 subgoal
============================
  0 {\coqle} 0

\tt{} constructor 1.

\it{}Proof completed
\tt{}Qed.
\end{alltt}

\noindent When
the current goal is an inductive type, the tactic 
``~\citecoq{constructor $i$}~'' \refmancite{Section \ref{constructor}} applies the $i$-th constructor in the
definition of the type. We can take a look at the proof constructed
using the command \texttt{Print}:

%V8 A prendre
\begin{alltt}
Print Print zero_leq_three.
\it{}zero_leq_three = 
zero_leq_three = le_S 0 2 (le_S 0 1 (le_S 0 0 (le_n 0)))
     : 0 {\coqle} 3
\end{alltt}

When the parameter $i$ is not supplied, the tactic \texttt{constructor}
tries to apply ``~\texttt{constructor $1$}~'', ``~\texttt{constructor $2$}~'',\dots,
``~\texttt{constructor $n$}~'' where $n$ is the number of constructors
of the inductive type (2 in our example) of the conclusion of the goal.
Our little proof can thus be obtained iterating the tactic
\texttt{constructor} until it fails:

%V8 A prendre
\begin{alltt}
Lemma zero_leq_three': 0 {\coqle} 3.
 repeat constructor.
Qed.
\end{alltt}

Notice that the strict order on \texttt{nat}, called \citecoq{lt}
is not inductively defined: the proposition $n<p$ (notation for \citecoq{lt $n$ $p$})
is reducible to \citecoq{(S $n$) $\leq$ p}.

\begin{alltt}
Print lt.
\it
lt = fun n m : nat {\funarrow} S n {\coqle} m
     : nat {\arrow} nat {\arrow} Prop
\tt
Lemma zero_lt_three : 0 < 3.
Proof.
 repeat constructor. 
Qed.

Print zero_lt_three.
\it zero_lt_three = le_S 1 2 (le_S 1 1 (le_n 1))
     : 0 < 3
\end{alltt}



\subsection{About general parameters (\coq{} version $\geq$ 8.1)}
\label{parameterstuff}

Since version $8.1$, it is possible to write more compact inductive definitions
than in earlier versions.

Consider the following alternative definition of the relation $\leq$ on 
type \citecoq{nat}:

\begin{alltt}
Inductive le'(n:nat):nat -> Prop :=
 | le'_n : le' n n
 | le'_S : forall p, le' (S n) p -> le' n p.

Hint Constructors le'.
\end{alltt}

We notice that the type of the second constructor of \citecoq{le'}
has an argument whose type is \citecoq{le' (S n) p}. 
This constrasts with earlier versions 
of {\coq}, in which a general parameter $a$ of an inductive
type $I$ had to appear only in applications of the form $I\,\dots\,a$.

Since version $8.1$, if $a$ is a general parameter of an inductive 
type $I$, the type of an argument of a constructor of $I$ may be
of the form  $I\,\dots\,t_a$ , where $t_a$ is any term.
Notice that the final type of the constructors must be of the form
$I\,\dots\,a$, since these constructors describe how to form 
inhabitants of type $I\,\dots\,a$ (this is the role of parameter $a$).

Another example of this new feature is {\coq}'s definition of accessibility
(see Section~\ref{WellFoundedRecursion}), which has a general parameter
$x$; the constructor for the predicate
``$x$ is accessible'' takes an  argument of type ``$y$ is accessible''.



In earlier versions of {\coq}, a relation like \citecoq{le'} would have to be
defined without $n$ being a general parameter.

\begin{alltt}
Reset le'.

Inductive le': nat-> nat -> Prop :=
 | le'_n : forall n, le' n n
 | le'_S : forall n p, le' (S n) p -> le' n p.
\end{alltt}




\subsection{The propositional equality type.} \label{equality}
In {\coq}, the propositional equality between two inhabitants $a$ and
$b$ of
the same type $A$ ,
noted $a=b$, is introduced as a family of recursive predicates
``~\textsl{to be equal to $a$}~'',  parameterised by both $a$ and its type
$A$. This family of types has only one introduction rule, which
corresponds to reflexivity.
Notice that the syntax ``\citecoq{$a$ = $b$}~'' is an abbreviation  
for ``\citecoq{eq $a$  $b$}~'', and that the parameter $A$ is \emph{implicit},
as it can be infered from $a$.
%V8 A prendre
\begin{alltt}
Print eq.
\it{} Inductive eq (A : Type) (x : A) : A \arrow{} Prop :=  
    refl_equal : x = x
For eq: Argument A is implicit
For refl_equal: Argument A is implicit
For eq: Argument scopes are [type_scope _ _]
For refl_equal: Argument scopes are [type_scope _]
\end{alltt}

Notice also that  the first parameter $A$ of \texttt{eq} has type
\texttt{Type}. The type system of {\coq} allows us to consider equality between 
various kinds of terms: elements of a set, proofs, propositions,
types, and so on.
Look at \cite{coqrefman, coqart} to get more details on {\coq}'s type
system, as well as implicit arguments and argument scopes.


\begin{alltt}
Lemma eq_3_3 : 2 + 1 = 3.
Proof.
 reflexivity.
Qed.

Lemma eq_proof_proof : refl_equal (2*6) = refl_equal (3*4).
Proof.
 reflexivity.
Qed.

Print eq_proof_proof.
\it eq_proof_proof = 
refl_equal (refl_equal (3 * 4))
     : refl_equal (2 * 6) = refl_equal (3 * 4)
\tt

Lemma eq_lt_le : ( 2 < 4) = (3 {\coqle} 4).
Proof.
 reflexivity.
Qed.

Lemma eq_nat_nat : nat = nat.
Proof.
 reflexivity.
Qed.

Lemma eq_Set_Set : Set = Set.
Proof.
 reflexivity.
Qed.
\end{alltt}

\subsection{Logical connectives.} \label{LogicalConnectives}
The conjunction and disjunction of two propositions are also examples
of recursive types:

\begin{alltt}
Inductive or (A B : Prop) : Prop :=
    or_introl : A \arrow{} A {\coqor} B | or_intror : B \arrow{} A {\coqor} B

Inductive and (A B : Prop) : Prop :=  
    conj : A \arrow{} B \arrow{} A {\coqand} B

\end{alltt}

The propositions $A$ and $B$ are general parameters of these
connectives. Choosing different universes for 
$A$ and $B$ and for the inductive type itself gives rise to different
type constructors. For example, the type \textsl{sumbool} is a
disjunction but with computational contents.

\begin{alltt}
Inductive sumbool (A B : Prop) : Set :=
    left : A \arrow{} \{A\} + \{B\} | right : B \arrow{} \{A\} + \{B\}
\end{alltt}



This type --noted \texttt{\{$A$\}+\{$B$\}} in {\coq}-- can be used in {\coq}
programs as a sort of boolean type, to check whether it is $A$ or $B$
that is true.  The values ``~\citecoq{left $p$}~'' and
``~\citecoq{right $q$}~'' replace the boolean values \textsl{true} and
\textsl{false}, respectively. The advantage of this type over
\textsl{bool} is that it makes available the proofs $p$ of $A$ or $q$
of $B$, which could be necessary to construct a verification proof
about the program.
For instance, let us consider the certified program \citecoq{le\_lt\_dec}
of the Standard Library.

\begin{alltt}
Require Import Compare_dec.
Check le_lt_dec.
\it
le_lt_dec
     : {\prodsym} n m : nat, \{n {\coqle} m\} + \{m < n\}

\end{alltt}

We use \citecoq{le\_lt\_dec} to build a function for computing
the max of two natural numbers:

\begin{alltt}
Definition max (n p :nat) := match le_lt_dec n p with 
                             | left _ {\funarrow} p
                             | right _ {\funarrow} n
                             end.
\end{alltt}

In the following proof, the case analysis on the term
``~\citecoq{le\_lt\_dec n p}~'' gives us an access to proofs
of $n\leq p$ in the first case, $p<n$ in the other.

\begin{alltt}
Theorem le_max : {\prodsym} n p, n {\coqle} p {\arrow} max n p = p.
Proof.
 intros n p ; unfold max ; case (le_lt_dec n p); simpl.
\it
2 subgoals
  
  n : nat
  p : nat
  ============================
   n {\coqle} p {\arrow} n {\coqle} p {\arrow} p = p

subgoal 2 is:
 p < n {\arrow} n {\coqle} p {\arrow} n = p
\tt
 trivial.
 intros; absurd (p < p); eauto with arith.
Qed.
\end{alltt}


 Once the program verified, the proofs are
erased by the extraction procedure:

\begin{alltt}
Extraction max.
\it
(** val max : nat {\arrow} nat {\arrow} nat **)

let max n p =
  match le_lt_dec n p with
    | Left {\arrow} p
    | Right {\arrow} n
\end{alltt}

Another example of use of \citecoq{sumbool} is given  in Section
\ref{WellFoundedRecursion}: the theorem \citecoq{eq\_nat\_dec} of
library \citecoq{Coq.Arith.Peano\_dec} is used in an euclidean division
algorithm.

\subsection{The existential quantifier.}\label{ex-def}
The existential quantifier is yet another example of a logical
connective introduced as an inductive type.

\begin{alltt}
Inductive ex (A : Type) (P : A \arrow{} Prop) : Prop :=
    ex_intro : {\prodsym} x : A, P x \arrow{} ex P
\end{alltt}

Notice that {\coq} uses the abreviation ``~\citecoq{\exsym\,$x$:$A$, $B$}~''
for \linebreak ``~\citecoq{ex (fun $x$:$A$ \funarrow{} $B$)}~''.


\noindent The former quantifier inhabits the universe of propositions.
As for the conjunction and disjunction connectives, there is also another
version of existential quantification inhabiting the universes $\Type_i$,
which is noted \texttt{sig $P$}. The syntax
``~\citecoq{\{$x$:$A$ | $B$\}}~'' is an abreviation for ``~\citecoq{sig (fun $x$:$A$ {\funarrow} $B$)}~''.



%\paragraph{The logical connectives.} Conjuction and disjuction are 
%also introduced as recursive types:
%\begin{alltt}
%Print or.
%\end{alltt}
%begin{alltt}
%Print and.
%\end{alltt}


\subsection{Mutually Dependent Definitions}
\label{MutuallyDependent}

Mutually dependent definitions of recursive types are also allowed in
{\coq}.  A typical example of these kind of declaration is the
introduction of the trees of unbounded (but finite) width:
\label{Forest}
\begin{alltt} 
Inductive tree(A:Type)   : Type :=
    node : A {\arrow} forest A \arrow{} tree A 
with  forest (A: Set)   : Type := 
    nochild  : forest A |
    addchild : tree A \arrow{} forest A \arrow{} forest A.
\end{alltt}
\noindent  Yet another example of mutually dependent types are the
predicates \texttt{even} and \texttt{odd} on natural numbers:
\label{Even}
\begin{alltt} 
Inductive 
  even    : nat\arrow{}Prop :=
    evenO : even  O |
    evenS : {\prodsym} n, odd n \arrow{} even (S n)
with
  odd    : nat\arrow{}Prop :=
    oddS : {\prodsym} n, even n \arrow{} odd (S n).
\end{alltt}

\begin{alltt}
Lemma odd_49 : odd (7 * 7).
 simpl; repeat constructor.
Qed.
\end{alltt}



\section{Case Analysis and Pattern-matching}
\label{CaseAnalysis}
\subsection{Non-dependent Case Analysis}
An \textsl{elimination rule} for the type $A$ is some way to use an
object $a:A$ in order to define an object in some  type $B$. 
A natural elimination rule for an inductive type is \emph{case analysis}.


For instance, any value of type {\nat} is built using either \texttt{O} or \texttt{S}.
Thus, a systematic way of building a value of type $B$ from any 
value of type {\nat} is to associate to \texttt{O} a constant $t_O:B$ and
to every term of the form ``~\texttt{S $p$}~'' a term $t_S:B$. The following
construction has type $B$:
\begin{alltt}
match \(n\) return \(B\) with O \funarrow \(t\sb{O}\) | S p \funarrow \(t\sb{S}\) end
\end{alltt}


In most of the cases, {\coq} is able to infer the type $B$ of the object
defined, so the ``\texttt{return $B$}'' part can be omitted.

The computing rules associated with this construct are the expected ones 
(the notation $t_S\{q/\texttt{p}\}$ stands for the substitution of $p$ by
$q$ in $t_S$ :)

\begin{eqnarray*}
\texttt{match $O$ return $b$ with O {\funarrow} $t_O$ | S p {\funarrow} $t_S$ end} &\Longrightarrow& t_O\\
\texttt{match $S\;q$  return $b$ with O {\funarrow} $t_O$ | S p {\funarrow} $t_S$ end}  &\Longrightarrow& t_S\{q/\texttt{p}\}
\end{eqnarray*}


\subsubsection{Example: the predecessor function.}\label{firstpred}
An example of a definition by case analysis is the function which
computes the predecessor of any given natural number:
\begin{alltt}
Definition pred (n:nat) := match n with
                                   | O {\funarrow} O 
                                   | S m {\funarrow} m 
                           end.

Eval simpl in pred 56.
\it{}    = 55
     : nat
\tt
Eval simpl in pred 0.
\it{}    = 0
     : nat

\tt{}Eval simpl in fun p {\funarrow} pred (S p).
\it{}     = fun p : nat {\funarrow} p
     : nat {\arrow} nat
\end{alltt}

As in functional programming, tuples and wild-cards can be used in
patterns \refmancite{Section \ref{ExtensionsOfCases}}. Such
definitions are automatically compiled by {\coq} into an expression which
may contain several nested case expressions. For example, the 
exclusive \emph{or} on booleans can be defined as follows:
\begin{alltt}
Definition xorb (b1 b2:bool) :=
 match b1, b2 with 
 | false, true {\funarrow} true
 | true, false {\funarrow} true
 | _ , _       {\funarrow} false
 end.
\end{alltt}

This kind of definition is compiled in {\coq} as follows\footnote{{\coq} uses
the conditional ``~\citecoq{if $b$ then $a$ else $b$}~'' as an abreviation to
``~\citecoq{match $b$ with true \funarrow{} $a$ | false \funarrow{} $b$ end}~''.}:

\begin{alltt}
Print xorb.
xorb = 
fun b1 b2 : bool {\funarrow}
if b1 then if b2 then false else true 
      else if b2 then true else false
     : bool {\arrow} bool {\arrow} bool
\end{alltt}

\subsection{Dependent Case Analysis}
\label{DependentCase}

For a pattern matching construct of the form
``~\citecoq{match n with \dots end}~'' a more general typing rule
is obtained considering that the type of the whole expression
may also depend on \texttt{n}.
  For instance, let us consider some function 
$Q:\texttt{nat}\arrow{}\texttt{Type}$, and $n:\citecoq{nat}$.
In order to build a term of type $Q\;n$, we can associate
to the constructor \texttt{O} some term $t_O: Q\;\texttt{O}$ and to
the pattern ``~\texttt{S p}~''  some term $t_S : Q\;(S\;p)$.
Notice that the terms $t_O$ and $t_S$ do not have the same type.

The syntax of the \emph{dependent case analysis} and its
associated typing rule make precise how the resulting
type depends on the argument of the pattern matching, and
which constraint holds on the branches of the pattern matching:

\label{Prod-sup-rule}
\[
\begin{array}[t]{l}
Q: \texttt{nat}{\arrow}\texttt{Type}\quad{t_O}:{{Q\;\texttt{O}}}  \quad
\smalljuge{p:\texttt{nat}}{t_p}{{Q\;(\texttt{S}\;p)}} \quad n:\texttt{nat} \\
\hline
{\texttt{match \(n\) as \(n\sb{0}\) return \(Q\;n\sb{0}\) with | O \funarrow \(t\sb{O}\) | S p \funarrow \(t\sb{S}\) end}}:{{Q\;n}}
\end{array}
\]


The interest of this rule of \textsl{dependent} pattern-matching is
that it can also be read as the following logical principle (when $Q$ has type \citecoq{nat\arrow{}Prop}
by \texttt{Prop} in the type of $Q$): in order to prove
that a property $Q$ holds for all $n$, it is sufficient to prove that
$Q$ holds for {\Z} and that for all $p:\nat$, $Q$ holds for
$(\SUCC\;p)$.  The former, non-dependent version of case analysis can
be obtained from this latter rule just taking $Q$ as a constant
function on $n$.

Notice that destructuring $n$ into \citecoq{O} or ``~\citecoq{S p}~''
 doesn't
make appear in the goal the equalities ``~$n=\citecoq{O}$~''
 and ``~$n=\citecoq{S p}$~''.
They are ``internalized'' in the rules above (see section~\ref{inversion}.)

\subsubsection{Example: strong specification of the predecessor function.}

In Section~\ref{firstpred}, the predecessor function was defined directly
as a function from \texttt{nat} to \texttt{nat}. It remains to prove
that this function has some desired properties. Another way to proceed
is to, first introduce a specification of what is the predecessor of a 
natural number, under the form of a {\coq} type, then build an inhabitant 
of this type: in other words, a realization of this specification. This way, the correctness
of this realization is ensured by {\coq}'s type system.

A reasonable specification for $\pred$ is to say that for all $n$
there exists another $m$ such that either $m=n=0$, or $(\SUCC\;m)$
is equal to  $n$. The function $\pred$ should be just the way to
compute such an $m$. 

\begin{alltt}
Definition pred_spec (n:nat) := 
   \{m:nat | n=0{\coqand} m=0 {\coqor} n = S m\}.
  
Definition  predecessor : {\prodsym} n:nat, pred_spec n.
 intro n; case n.
\it{}  
  n : nat
  ============================
   pred_spec 0

\tt{} unfold pred_spec;exists 0;auto.
\it{}
 =========================================
 {\prodsym} n0 : nat, pred_spec (S n0)
\tt{}
 unfold pred_spec; intro n0; exists n0; auto.
Defined.
\end{alltt}

If we print the term built by {\coq}, we can observe its dependent pattern-matching structure:

\begin{alltt}
predecessor =  fun n : nat {\funarrow}
\textbf{match n as n0 return (pred_spec n0) with}
\textbf{| O {\funarrow}}
    exist (fun m : nat {\funarrow} 0 = 0 {\coqand} m = 0 {\coqor} 0 = S m) 0
      (or_introl (0 = 1) 
                 (conj (refl_equal 0) (refl_equal 0)))
\textbf{| S n0 {\funarrow}}
    exist (fun m : nat {\funarrow} S n0 = 0 {\coqand} m = 0 {\coqor} S n0 = S m) n0
      (or_intror (S n0 = 0 {\coqand} n0 = 0) (refl_equal (S n0)))
\textbf{end}  : {\prodsym} n : nat, \textbf{pred_spec n}
\end{alltt}


Notice that there are many variants to the pattern ``~\texttt{intros \dots; case \dots}~''. Look at the reference manual and/or the book: tactics
\texttt{destruct}, ``~\texttt{intro \emph{pattern}}~'', etc.

\noindent The command \texttt{Extraction} \refmancite{Section
\ref{ExtractionIdent}} can be used to see the computational
contents associated to the \emph{certified} function \texttt{predecessor}:
\begin{alltt}
Extraction predecessor.
\it
(** val predecessor : nat {\arrow} pred_spec **)

let predecessor = function
  | O {\arrow} O
  | S n0 {\arrow} n0
\end{alltt}


\begin{exercise} \label{expand}
Prove the following theorem:
\begin{alltt}
Theorem nat_expand : {\prodsym} n:nat, 
      n = match n with 
                  | 0 {\funarrow} 0 
                  | S p {\funarrow} S p 
          end.
\end{alltt}
\end{exercise}

\subsection{Some Examples of Case Analysis}
\label{CaseScheme}
The reader will find in the Reference manual all details about
typing case analysis (chapter 4: Calculus of Inductive Constructions,
and chapter 15: Extended Pattern-Matching).

The following commented examples will show the different situations to consider.


%\subsubsection{General Scheme}

%Case analysis is then the most basic elimination rule that {\coq}
%provides for inductive  types.  This rule follows a general schema,
%valid for any inductive type $I$. First, if $I$ has type
%``~$\forall\,(z_1:A_1)\ldots(z_r:A_r),S$~'', with $S$ either $\Set$, $\Prop$ or
%$\Type$, then a case expression on $p$ of type ``~$R\;a_1\ldots a_r$~''
% inhabits ``~$Q\;a_1\ldots a_r\;p$~''. The types of the branches of the case expression
%are obtained from the definition of the type in this way: if the type
%of the $i$-th constructor $c_i$  of $R$ is 
%``~$\forall\, (x_1:T_1)\ldots
%(x_n:T_n),(R\;q_1\ldots q_r)$~'', then the  $i-th$ branch must have the
%form ``~$c_i\; x_1\; \ldots \;x_n\;  \funarrow{}\; t_i$~'' where
%$$(x_1:T_1),\ldots, (x_n:T_n) \vdash t_i : Q\;q_1\ldots q_r)$$
% for non-dependent case
%analysis, and $$(x_1:T_1)\ldots (x_n:T_n)\vdash t_i :Q\;q_1\ldots
%q_r\;({c}_i\;x_1\;\ldots x_n)$$ for dependent one. In the
%following section, we illustrate this general scheme for different
%recursive types.
%%\textbf{A v�rifier}

\subsubsection{The Empty Type}

In a definition by case analysis, there is one branch for each
introduction rule of the type. Hence, in a definition by case analysis
on $p:\False$ there are no cases to be considered. In other words, the
rule of (non-dependent) case analysis for the type $\False$ is 
(for $s$ in  \texttt{Prop}, \texttt{Set} or \texttt{Type}):

\begin{center}
\snregla {\JM{Q}{s}\;\;\;\;\;
          \JM{p}{\False}}
         {\JM{\texttt{match $p$ return $Q$ with end}}{Q}}
\end{center}

As a corollary, if we could construct an object in $\False$, then it
could be possible to define an object in any type.  The tactic
\texttt{contradiction} \refmancite{Section \ref{Contradiction}}
corresponds to the application of the elimination rule above.  It
searches in the context for an absurd hypothesis (this is, a
hypothesis whose type is $\False$) and then proves the goal by a case
analysis of it.

\begin{alltt}
Theorem fromFalse : False \arrow{} 0=1.
Proof.
 intro H. 
 contradiction.
Qed.
\end{alltt}


In {\coq} the negation is defined as follows :

\begin{alltt}
Definition not (P:Prop) := P {\arrow} False
\end{alltt}

The proposition ``~\citecoq{not $A$}~'' is also written ``~$\neg A$~''.

If $A$ and $B$ are propositions, $a$ is a proof of $A$ and
$H$ is a proof of $\neg A$,
the term ``~\citecoq{match $H\;a$ return $B$ with end}~'' is a proof term of
$B$.
Thus, if your goal is $B$ and you have some hypothesis $H:\neg A$,
the tactic ``~\citecoq{case $H$}~'' generates a new subgoal with
statement $A$, as shown by  the following example\footnote{Notice that
$a\coqdiff b$ is just an abreviation for ``~\coqnot a= b~''}.

\begin{alltt}
Fact Nosense : 0 {\coqdiff} 0 {\arrow} 2 = 3.
Proof.
  intro H; case H.
\it
===========================
  0 = 0
\tt
  reflexivity.
Qed.
\end{alltt}

The tactic ``~\texttt{absurd $A$}~'' (where $A$ is any proposition), 
is based on the same principle, but
generates two subgoals: $A$ and $\neg A$, for solving $B$.

\subsubsection{The Equality Type}

Let $A:\Type$, $a$, $b$ of type $A$, and $\pi$ a proof of 
$a=b$. Non dependent case analysis of $\pi$ allows us to
associate to any proof  of ``~$Q\;a$~'' a proof of ``~$Q\;b$~'',
where $Q:A\arrow{} s$ (where $s\in\{\Prop, \Set, \Type\}$).
The following term is a proof  of ``~$Q\;a\, \arrow{}\, Q\;b$~''.

\begin{alltt}
fun H : Q a {\funarrow}
  match \(\pi\) in (_ = y) return Q y with
     refl_equal {\funarrow} H
  end
\end{alltt}
Notice the header of the \texttt{match} construct.
It expresses how the resulting type ``~\citecoq{Q y}~'' depends on 
the \emph{type} of \texttt{p}.
Notice also that in the pattern introduced by the keyword \texttt{in},
the parameter \texttt{a} in the type ``~\texttt{a = y}~'' must be
implicit, and replaced by a wildcard '\texttt{\_}'.


Therefore, case analysis on a proof of the equality $a=b$
amounts to replacing all the occurrences of the term $b$ with the term
$a$ in the goal to be proven. Let us illustrate this through an
example: the transitivity property of this equality. 
\begin{alltt}
Theorem trans : {\prodsym} n m p:nat, n=m \arrow{} m=p \arrow{} n=p.
Proof.
 intros n m p eqnm.  
\it{}  
  n : nat
  m : nat
  p : nat
  eqnm : n = m
  ============================
   m = p {\arrow} n = p
\tt{} case eqnm.
\it{}
  n : nat
  m : nat
  p : nat
  eqnm : n = m
  ============================
   n = p {\arrow} n = p
\tt{} trivial.
Qed.
\end{alltt}

%\noindent The case analysis on the hypothesis $H:n=m$ yields the
%tautological subgoal $n=p\rightarrow n=p$, that is directly proven by
%the tactic \texttt{Trivial}.

\begin{exercise}
Prove the symmetry property of  equality.
\end{exercise}

Instead of using \texttt{case}, we can use the tactic 
\texttt{rewrite} \refmancite{Section \ref{Rewrite}}.  If $H$ is a proof
of  $a=b$, then
``~\citecoq{rewrite $H$}~''
 performs a case analysis on a proof of $b=a$, obtained by applying a
symmetry theorem to $H$. This application of symmetry allows us to rewrite
the equality from left to right, which looks more natural. An optional
parameter (either \texttt{\arrow{}} or \texttt{$\leftarrow$}) can be used to precise
in which sense the equality must be rewritten.  By default,
``~\texttt{rewrite} $H$~'' corresponds to ``~\texttt{rewrite \arrow{}} $H$~''
\begin{alltt}
Lemma Rw :  {\prodsym} x y: nat, y = y * x {\arrow} y * x * x = y.
 intros x y e; do 2 rewrite <- e.
\it
1 subgoal
  
  x : nat
  y : nat
  e : y = y * x
  ============================
   y = y
\tt
 reflexivity.
Qed.
\end{alltt}

Notice that, if $H:a=b$, then the tactic ``~\texttt{rewrite $H$}~''
 replaces \textsl{all} the
occurrences of $a$ by $b$. However, in certain situations we could be
interested in rewriting some of the occurrences, but not all of them.
This can be done using the tactic \texttt{pattern} \refmancite{Section
\ref{Pattern}}.  Let us consider yet another example to
illustrate this.

Let us start with some simple theorems of arithmetic; two of them 
are already proven in the Standard Library, the last is left as an exercise.

\begin{alltt}
\it
mult_1_l
     : {\prodsym} n : nat, 1 * n = n

mult_plus_distr_r
     : {\prodsym} n m p : nat, (n + m) * p = n * p + m * p

mult_distr_S : {\prodsym} n p : nat, n * p + p = (S n)* p.
\end{alltt}

Let us now prove a simple result:

\begin{alltt}
Lemma four_n : {\prodsym} n:nat, n+n+n+n = 4*n.
Proof.
 intro n;rewrite <- (mult_1_l n).
\it
  n : nat
  ============================
   1 * n + 1 * n + 1 * n + 1 * n = 4 * (1 * n)
\end{alltt}

We can see that the \texttt{rewrite} tactic call replaced \emph{all}
the occurrences of \texttt{n} by the term ``~\citecoq{1 * n}~''.
If we want to do the rewriting ony on the leftmost occurrence of
\texttt{n}, we can mark this occurrence using the \texttt{pattern}
tactic:


\begin{alltt}
 Undo.
 intro n; pattern n at 1.
 \it
 n : nat
  ============================
 (fun n0 : nat {\funarrow} n0 + n + n + n = 4 * n) n
\end{alltt}
Applying the tactic ``~\citecoq{pattern  n at 1}~'' allowed us
to explicitly abstract the first occurrence of \texttt{n} from the
goal, putting this goal under the form ``~\citecoq{$Q$ n}~'',
thus pointing to \texttt{rewrite} the particular predicate on $n$
that we search to prove. 


\begin{alltt}
 rewrite <- mult_1_l.
\it
1 subgoal
  
  n : nat
  ============================
   1 * n + n + n + n = 4 * n
\tt
 repeat rewrite   mult_distr_S.
\it
  n : nat
  ============================
   4 * n = 4 * n
\tt
 trivial.
Qed.
\end{alltt}

\subsubsection{The Predicate $n {\leq} m$}


The last but one instance of the elimination schema that we will illustrate is
case analysis for the predicate $n {\leq} m$:

Let $n$ and $p$ be terms of type \citecoq{nat}, and $Q$ a predicate 
of type $\citecoq{nat}\arrow{}\Prop$.
If $H$ is a proof of ``~\texttt{n {\coqle} p}~'',
$H_0$ a proof of ``~\texttt{$Q$  n}~'' and
$H_S$ a proof of ``~\citecoq{{\prodsym}m:nat, n {\coqle}  m {\arrow} Q (S m)}~'',
then the term
\begin{alltt}
match H in (_ {\coqle} q) return (Q q) with
    | le_n {\funarrow} H0
    | le_S m Hm {\funarrow} HS m Hm
end
\end{alltt}
 is a proof term of ``~\citecoq{$Q$ $p$}~''.


The two patterns of this \texttt{match} construct describe
all possible forms of proofs of ``~\citecoq{n {\coqle} m}~'' (notice
again that the general parameter \texttt{n} is implicit in
 the ``~\texttt{in \dots}~''
clause and is absent from the match patterns.


Notice that the choice of introducing some of the arguments of the
predicate as being general parameters in its definition has
consequences on the rule of case analysis that is derived. In
particular, the type $Q$ of the object defined by the case expression
only depends on the indexes of the predicate, and not on the general
parameters.  In the definition of the predicate $\leq$, the first
argument of this relation is a general parameter of the
definition. Hence, the predicate $Q$ to be proven only depends on the
second argument of the relation. In other words, the integer $n$ is
also a general parameter of the rule of case analysis.

An example of an application of this rule is the following theorem,
showing that any integer greater or equal than $1$ is the successor of another
natural number:

\begin{alltt}
Lemma predecessor_of_positive : 
 {\prodsym} n, 1 {\coqle} n {\arrow} {\exsym} p:nat, n = S p.
Proof.
 intros n H;case H.
\it
  n : nat
  H : 1 {\coqle} n
  ============================
   {\exsym} p : nat, 1 = S p
\tt
  exists 0; trivial.
\it

  n : nat
  H : 1 {\coqle} n
  ============================
   {\prodsym} m : nat, 0 {\coqle} m {\arrow} {\exsym} p : nat, S m = S p
\tt
  intros m _  .
  exists m.
  trivial.
Qed.
\end{alltt}


\subsubsection{Vectors}

The \texttt{vector} polymorphic and dependent family of types will
give an idea of the most general scheme of pattern-matching.

For instance, let us define a function for computing the tail of
any vector. Notice that we shall build a \emph{total} function,
by considering that the tail of an empty vector is this vector itself.
In that sense, it will be slightly different from the \texttt{Vtail}
function of the Standard Library, which is defined only for vectors
of type ``~\citecoq{vector $A$ (S $n$)}~''.

The header of the function we want to build is the following:

\begin{verbatim}
Definition Vtail_total 
   (A : Type) (n : nat) (v : vector A n) : vector A (pred n):=
\end{verbatim}

Since the branches will not have the same type
(depending on the parameter \texttt{n}),
the body of this function is a dependent pattern matching on 
\citecoq{v}.
So we will have :
\begin{verbatim}
match v in (vector _ n0) return (vector A (pred n0)) with
\end{verbatim}

The first branch  deals with the constructor \texttt{Vnil} and must
return a value in ``~\citecoq{vector A (pred 0)}~'', convertible
to ``~\citecoq{vector A 0}~''. So, we propose:
\begin{alltt}
| Vnil {\funarrow} Vnil A
\end{alltt}

The second branch considers a vector in ``~\citecoq{vector A (S n0)}~''
of the form
``~\citecoq{Vcons A n0 v0}~'', with ``~\citecoq{v0:vector A n0}~'',
and must return a value of type ``~\citecoq{vector A (pred (S n0))}~'',
which is convertible to ``~\citecoq{vector A n0}~''.
This second branch is thus :
\begin{alltt}
| Vcons _ n0 v0 {\funarrow} v0
\end{alltt}

Here is the full definition:

\begin{alltt}
Definition Vtail_total 
   (A : Type) (n : nat) (v : vector A n) : vector A (pred n):=
match v in (vector _ n0) return (vector A (pred n0)) with
| Vnil {\funarrow} Vnil A
| Vcons _ n0 v0 {\funarrow} v0
end.
\end{alltt}


\subsection{Case Analysis and Logical Paradoxes}

In the previous section we have illustrated the general scheme for
generating the rule of case analysis associated to some recursive type
from the definition of the type. However, if the logical soundness is
to be preserved, certain restrictions to this schema are
necessary. This section provides a brief explanation of these
restrictions.


\subsubsection{The Positivity Condition}
\label{postypes}

In order to make sense of recursive types as types closed under their
introduction rules, a constraint has to be imposed on the possible
forms of such rules. This constraint, known as the
\textsl{positivity condition}, is necessary to prevent the user from
naively introducing some recursive types which would open the door to
logical paradoxes.  An example of such a dangerous type is the
``inductive type'' \citecoq{Lambda}, whose only constructor is 
\citecoq{lambda} of type \citecoq{(Lambda\arrow False)\arrow Lambda}.
 Following the pattern
given in Section \ref{CaseScheme}, the rule of (non dependent) case
analysis for \citecoq{Lambda} would be the following:

\begin{center}
\snregla {\JM{Q}{\Prop}\;\;\;\;\;
          \JM{p}{\texttt{Lambda}}\;\;\;\;\;
          {h : {\texttt{Lambda}}\arrow\False\; \vdash\; t\,:\,Q}}
         {\JM{\citecoq{match $p$ return $Q$ with lambda h {\funarrow} $t$  end}}{Q}}
\end{center}

In order to avoid paradoxes, it is impossible to construct
the type \citecoq{Lambda} in {\coq}:

\begin{alltt}
Inductive Lambda : Set :=
  lambda : (Lambda {\arrow} False) {\arrow} Lambda. 
\it
Error: Non strictly positive occurrence of "Lambda" in
 "(Lambda {\arrow} False) {\arrow} Lambda"
\end{alltt}

In order to explain this danger, we
will declare some constants for simulating the construction of 
\texttt{Lambda} as an inductive type.

Let us open some section, and declare two variables, the first one for
\texttt{Lambda}, the other for the constructor \texttt{lambda}.

\begin{alltt}
Section Paradox.
Variable Lambda : Set.
Variable lambda : (Lambda {\arrow} False) {\arrow}Lambda.
\end{alltt}

Since \texttt{Lambda} is not a truely inductive type, we can't use
the \texttt{match} construct. Nevertheless, we can simulate it by a
variable \texttt{matchL} such that the term 
``~\citecoq{matchL $l$  $Q$ (fun $h$ : Lambda {\arrow} False {\funarrow} $t$)}~''
should be understood  as 
``~\citecoq{match $l$ return $Q$ with | lambda h {\funarrow} $t$)}~''


\begin{alltt}
Variable matchL : Lambda {\arrow} 
                  {\prodsym} Q:Prop, ((Lambda {\arrow}False) {\arrow} Q) {\arrow}
                  Q.
\end{alltt}

>From these constants, it is possible to define application by case
analysis. Then, through auto-application, the well-known looping term
$(\lambda x.(x\;x)\;\lambda x.(x\;x))$ provides a proof of falsehood.

\begin{alltt}
Definition application (f x: Lambda) :False :=
  matchL f False (fun h {\funarrow} h x).

Definition Delta :  Lambda := 
  lambda (fun x : Lambda {\funarrow} application x x).

Definition loop : False := application Delta Delta.

Theorem two_is_three : 2 = 3.
Proof.
 elim loop.
Qed.

End Paradox.
\end{alltt}

\noindent This example can be seen as a formulation of Russell's
paradox in type theory associating $(\textsl{application}\;x\;x)$ to the
formula $x\not\in x$, and \textsl{Delta} to the set $\{ x \mid
x\not\in x\}$. If \texttt{matchL} would satisfy the reduction rule
associated to case analysis, that is, 
$$ \citecoq{matchL (lambda $f$) $Q$ $h$} \Longrightarrow h\;f$$
then the term \texttt{loop}
would compute into itself.  This is not actually surprising, since the
proof of the logical soundness of {\coq} strongly lays on the property
that any well-typed term must terminate. Hence, non-termination is
usually a synonymous of inconsistency.

%\paragraph{} In this case, the construction of a non-terminating
%program comes from the so-called \textsl{negative occurrence} of
%$\Lambda$ in the type of the constructor $\lambda$. In order to be
%admissible for {\coq}, all the occurrences of the recursive type in its
%own introduction rules must be positive, in the sense on the following
%definition:
%
%\begin{enumerate}
%\item $R$ is positive in $(R\;\vec{t})$;
%\item $R$ is positive in $(x: A)C$ if it does not 
%occur in $A$ and $R$ is positive in $C$; 
%\item if $P\equiv (\vec{x}:\vec{T})Q$, then $R$ is positive in $(P
%\rightarrow C)$ if $R$ does not occur in $\vec{T}$, $R$ is positive
%in $C$, and either
%\begin{enumerate} 
%\item $Q\equiv (R\;\vec{q})$ or 
%\item $Q\equiv (J\;\vec{t})$, \label{relax}
%      where $J$ is a recursive type, and for any term $t_i$ either : 
%   \begin{enumerate}
%      \item $R$ does not occur in $t_i$, or
%      \item $t_i\equiv (z:\vec{Z})(R\;\vec{q})$, $R$ does not occur
%       in $\vec{Z}$, $t_i$ instantiates a general
%       parameter of $J$, and this parameter is positive in the 
%       arguments of the constructors of $J$. 
%   \end{enumerate}
%\end{enumerate}
%\end{enumerate}
%\noindent Those types obtained by erasing option (\ref{relax}) in the
%definition above are called \textsl{strictly positive} types.


\subsubsection*{Remark} In this case, the construction of a non-terminating
program comes from the so-called \textsl{negative occurrence} of
\texttt{Lambda} in the argument of the constructor \texttt{lambda}. 

The reader will find in the Reference Manual a complete formal 
definition of the notions of \emph{positivity condition} and
\emph{strict positivity} that an inductive definition must satisfy.


%In order to be
%admissible for {\coq}, the type $R$ must be positive in the types of the
%arguments of its own introduction rules, in the sense on the following
%definition:

%\textbf{La d�finition du manuel de r�f�rence est plus complexe:
%la recopier ou donner seulement des exemples?
%}
%\begin{enumerate}
%\item $R$ is positive in $T$ if $R$ does not occur in $T$;
%\item $R$ is positive in $(R\;\vec{t})$ if $R$ does not occur in $\vec{t}$;
%\item $R$ is positive in $(x:A)C$ if it does not
%      occur in $A$ and $R$ is positive in $C$; 
%\item $R$ is positive in $(J\;\vec{t})$, \label{relax}
%      if $J$ is a recursive type, and for any term $t_i$ either : 
%   \begin{enumerate}
%      \item $R$ does not occur in $t_i$, or
%      \item $R$ is positive in $t_i$,  $t_i$ instantiates a general
%       parameter of $J$, and this parameter is positive in the 
%       arguments of the constructors of $J$. 
%   \end{enumerate}
%\end{enumerate}

%\noindent When we can show that $R$ is positive without using the item
%(\ref{relax}) of the definition above, then we say that $R$ is
%\textsl{strictly positive}.

%\textbf{Changer le discours sur les ordinaux}

Notice that the positivity condition does not forbid us to
put  functional recursive
arguments in the constructors. 

For instance, let us consider the type of infinitely branching trees,
with labels in \texttt{Z}.
\begin{alltt}
Require Import ZArith.

Inductive itree : Set :=
| ileaf : itree
| inode : Z {\arrow} (nat {\arrow} itree) {\arrow} itree.
\end{alltt}

In this representation, the $i$-th child of a tree 
represented by ``~\texttt{inode $z$ $s$}~'' is obtained by applying
the function $s$ to $i$.
The following definitions show how to construct a tree with a single 
node, a tree of height 1 and a tree of height 2:

\begin{alltt}
Definition isingle l := inode l (fun i {\funarrow} ileaf).

Definition t1 := inode 0 (fun n {\funarrow} isingle (Z_of_nat n)).

Definition t2 := 
 inode 0 
      (fun n : nat {\funarrow} 
                   inode (Z_of_nat n)
                   (fun p {\funarrow} isingle (Z_of_nat (n*p)))).
\end{alltt}


Let us define a preorder on infinitely branching trees.
 In order to compare two non-leaf trees,
it is necessary to compare each of their children 
 without taking care of the order in which they
appear:

\begin{alltt}
Inductive itree_le : itree{\arrow} itree {\arrow} Prop :=
  | le_leaf : {\prodsym} t, itree_le  ileaf t
  | le_node : {\prodsym} l l' s s', 
                Zle l l' {\arrow} 
                ({\prodsym} i, {\exsym} j:nat, itree_le (s i) (s' j)){\arrow} 
                itree_le  (inode  l s) (inode  l' s').

\end{alltt}

Notice that a call to the predicate \texttt{itree\_le} appears as
a general parameter of the inductive type  \texttt{ex} (see Sect.\ref{ex-def}).
This kind of definition is accepted by {\coq}, but may lead to some
difficulties, since the induction principle automatically 
generated by the system
is not the most appropriate (see chapter 14 of~\cite{coqart} for a detailed
explanation).


The following definition, obtained by 
skolemising the
proposition \linebreak $\forall\, i,\exists\, j,(\texttt{itree\_le}\;(s\;i)\;(s'\;j))$ in
the type of \texttt{itree\_le}, does not present this problem:


\begin{alltt} 
Inductive itree_le' : itree{\arrow} itree {\arrow} Prop :=
  | le_leaf'  : {\prodsym} t, itree_le'  ileaf t
  | le_node' : {\prodsym} l l' s s' g, 
                  Zle l l' {\arrow}  
                  ({\prodsym} i, itree_le' (s i) (s' (g i))) {\arrow} 
                  itree_le'  (inode  l s) (inode  l' s').

\end{alltt}
\iffalse
\begin{alltt}
Lemma t1_le'_t2 :  itree_le' t1 t2.
Proof.
 unfold t1, t2.
 constructor 2  with (fun i : nat {\funarrow} 2 * i).
 auto with zarith.
 unfold isingle;
 intro i ; constructor 2 with (fun i :nat {\funarrow} i).
 auto with zarith.
 constructor .
Qed.
\end{alltt}
\fi

%In general, strictly positive definitions are preferable to only
%positive ones. The reason is that it is sometimes difficult to derive
%structural induction combinators for the latter ones. Such combinators
%are automatically generated for strictly positive types, but not for
%the only positive ones. Nevertheless, sometimes non-strictly positive
%definitions provide a smarter or shorter way of declaring a recursive
%type.

Another example is the type of trees 
 of unbounded width, in which a recursive subterm 
\texttt{(ltree A)} instantiates the type  of polymorphic lists:

\begin{alltt} 
Require Import List.

Inductive ltree  (A:Set) : Set :=  
          lnode   : A {\arrow} list (ltree A) {\arrow} ltree A.
\end{alltt}

This declaration can be transformed  
adding an extra type to the definition, as was done in Section
\ref{MutuallyDependent}.


\subsubsection{Impredicative Inductive Types}

An inductive type $I$ inhabiting a universe $U$ is \textsl{predicative}
if the introduction rules of $I$ do not make a universal
quantification on a universe containing $U$. All the recursive types
previously introduced are examples of predicative types. An example of
an impredicative one is the following type:
%\textsl{exT}, the dependent product
%of a certain set (or proposition) $x$, and a proof of a property $P$
%about $x$.

%\begin{alltt} 
%Print exT.
%\end{alltt}
%\textbf{ttention, EXT c'est ex!}
%\begin{alltt}
%Check (exists P:Prop, P {\arrow} not P).  
%\end{alltt}

%This type is useful for expressing existential quantification over
%types, like ``there exists a proposition $x$ such that $(P\;x)$''
%---written $(\textsl{EXT}\; x:Prop \mid (P\;x))$ in {\coq}. However,

\begin{alltt}
Inductive prop : Prop :=
 prop_intro : Prop {\arrow} prop.
\end{alltt}

Notice
that the constructor of this type can be used to inject any
proposition --even itself!-- into the type. 

\begin{alltt}
Check (prop_intro prop).\it
prop_intro prop
     : prop
\end{alltt}

A careless use of such a
self-contained objects may lead to a variant of Burali-Forti's
paradox. The construction of Burali-Forti's paradox is more
complicated than Russel's one, so we will not describe it here, and
point the interested reader to \cite{Bar98,Coq86}.


Another example is the second order existential quantifier for propositions:

\begin{alltt}
Inductive ex_Prop  (P : Prop {\arrow} Prop) : Prop :=
  exP_intro : {\prodsym} X : Prop, P X {\arrow} ex_Prop P.
\end{alltt}

%\begin{alltt}
%(*
%Check (match prop_inject with (prop_intro p _) {\funarrow} p end).

%Error: Incorrect elimination of "prop_inject" in the inductive type
% ex
%The elimination predicate ""fun _ : prop {\funarrow} Prop" has type
% "prop {\arrow} Type"
%It should be one of :
% "Prop"

%Elimination of an inductive object of sort : "Prop"
%is not allowed on a predicate in sort : "Type"
%because non-informative objects may not construct informative ones.

%*)
%Print prop_inject.

%(*
%prop_inject = 
%prop_inject = prop_intro prop (fun H : prop {\funarrow} H)
%     : prop
%*)
%\end{alltt}

% \textbf{Et par �a?
%}

Notice that predicativity on sort \citecoq{Set} forbids us to build
the following definitions.


\begin{alltt}
Inductive aSet : Set :=
  aSet_intro: Set {\arrow} aSet.

\it{}User error: Large non-propositional inductive types must be in Type
\tt 
Inductive ex_Set  (P : Set {\arrow} Prop) : Set :=
  exS_intro : {\prodsym} X : Set, P X {\arrow} ex_Set P.

\it{}User error: Large non-propositional inductive types must be in Type
\end{alltt}

Nevertheless, one can define types like \citecoq{aSet} and \citecoq{ex\_Set}, as inhabitants of \citecoq{Type}.

\begin{alltt}
Inductive ex_Set  (P : Set {\arrow} Prop) : Type :=
  exS_intro : {\prodsym} X : Set, P X {\arrow} ex_Set P.
\end{alltt}

In the following example, the inductive type \texttt{typ} can be defined,
but the term associated with the interactive Definition of
\citecoq{typ\_inject} is incompatible with {\coq}'s hierarchy of universes:


\begin{alltt}
Inductive  typ : Type := 
  typ_intro : Type {\arrow} typ. 

Definition typ_inject: typ.
 split; exact typ.
\it Proof completed

\tt{}Defined.
\it Error: Universe Inconsistency.
\tt
Abort.
\end{alltt}

One possible way of avoiding this new source of paradoxes is to
restrict the kind of eliminations by case analysis that can be done on
impredicative types. In particular, projections on those universes
equal or bigger than the one inhabited by the impredicative type must
be forbidden \cite{Coq86}. A consequence of this restriction is that it
is not possible to define the first projection of the type
``~\citecoq{ex\_Prop $P$}~'':
\begin{alltt}
Check (fun (P:Prop{\arrow}Prop)(p: ex_Prop P) {\funarrow}
      match p with exP_intro X HX {\funarrow} X end).
\it
Error:
Incorrect elimination of "p" in the inductive type  
"ex_Prop", the return type has sort "Type" while it should be 
"Prop"

Elimination of an inductive object of sort "Prop"
is not allowed on a predicate in sort "Type"
because proofs can be eliminated only to build proofs.
\end{alltt}

%In order to explain why, let us consider for example the following
%impredicative type \texttt{ALambda}.
%\begin{alltt} 
%Inductive ALambda : Set := 
%          alambda : (A:Set)(A\arrow{}False)\arrow{}ALambda.
%
%Definition Lambda : Set := ALambda.
%Definition lambda : (ALambda\arrow{}False)\arrow{}ALambda := (alambda ALambda).
%Lemma CaseAL : (Q:Prop)ALambda\arrow{}((ALambda\arrow{}False)\arrow{}Q)\arrow{}Q.
%\end{alltt}
%
%This type contains all the elements of the dangerous type $\Lambda$
%described at the beginning of this section.  Try to construct the
%non-ending term $(\Delta\;\Delta)$ as an object of
%\texttt{ALambda}. Why is it not possible?

\subsubsection{Extraction Constraints}

There is a final constraint on case analysis that is not motivated by
the potential introduction of paradoxes, but for compatibility reasons
with {\coq}'s extraction mechanism \refmancite{Appendix
\ref{CamlHaskellExtraction}}. This mechanism is based on the
classification of basic types into the universe $\Set$ of sets and the
universe $\Prop$ of propositions.  The objects of a type in the
universe $\Set$ are considered as relevant for computation
purposes. The objects of a type in $\Prop$ are considered just as
formalised comments, not necessary for execution. The extraction
mechanism consists in erasing such formal comments in order to obtain
an executable program. Hence, in general, it is not possible to define
an object in a set (that should be kept by the extraction mechanism)
by case analysis of a proof (which will be thrown away).

Nevertheless, this general rule has an exception which is important in
practice: if the definition proceeds by case analysis on a proof of a
\textsl{singleton proposition} or an empty type (\emph{e.g.} \texttt{False}),
 then it is allowed. A singleton
proposition is a non-recursive proposition with a single constructor
$c$, all whose arguments are proofs. For example, the propositional
equality and the conjunction of two propositions are examples of
singleton propositions.

%From the point of view of the extraction
%mechanism, such types are isomorphic to a type containing a single
%object $c$, so a definition $\Case{x}{c \Rightarrow b}$ is 
%directly replaced by $b$ as an extra optimisation.

\subsubsection{Strong Case Analysis on Proofs}

One could consider allowing 
 to define a proposition $Q$ by case
analysis on the proofs of another recursive proposition $R$. As we
will see in Section \ref{Discrimination}, this would enable one to prove that
different introduction rules of $R$ construct different
objects. However, this property would be in contradiction with the principle
of excluded middle of classical logic, because this principle entails
that the proofs of a proposition cannot be distinguished. This
principle is not provable in {\coq}, but it is frequently introduced by
the users as an axiom, for reasoning in classical logic. For this
reason, the definition of propositions by case analysis on proofs is
 not allowed in {\coq}.

\begin{alltt}

Definition comes_from_the_left (P Q:Prop)(H:P{\coqor}Q): Prop :=
 match H with
         |  or_introl p {\funarrow} True 
         |  or_intror q {\funarrow} False
 end.
\it
Error:
Incorrect elimination of "H" in the inductive type  
"or", the return type has sort "Type" while it should be 
"Prop"

Elimination of an inductive object of sort "Prop"
is not allowed on a predicate in sort "Type"
because proofs can be eliminated only to build proofs.

\end{alltt}

On the other hand, if we replace the proposition $P {\coqor} Q$ with
the informative type $\{P\}+\{Q\}$, the elimination  is accepted:

\begin{alltt}
Definition comes_from_the_left_sumbool
            (P Q:Prop)(x:\{P\} + \{Q\}): Prop :=
  match x with
         |  left  p {\funarrow} True 
         |  right q {\funarrow} False
  end.
\end{alltt}


\subsubsection{Summary of Constraints}

To end with this section, the following table summarizes which
universe $U_1$ may inhabit an object of type $Q$ defined by case
analysis on $x:R$, depending on the universe $U_2$ inhabited by the
inductive types $R$.\footnote{In the box indexed by $U_1=\citecoq{Type}$
and $U_2=\citecoq{Set}$, the answer ``yes''  takes into account the 
predicativity of sort \citecoq{Set}. If you are working with the
option ``impredicative-set'', you must put in this box the
condition ``if $R$ is predicative''.}


\begin{center}
\renewcommand{\multirowsetup}{\centering} \newlength{\LL}
\settowidth{\LL}{$x : R : s_1$}
\begin{tabular}{|c|c|c|c|c|}
\hline
\multirow{5}{\LL}{$x : R : U_2$} &
\multicolumn{4}{|c|}{$Q : U_1$}\\
\hline
&                 &\textsl{Set}      & \textsl{Prop} & \textsl{Type}\\
\cline{2-5}
&\textsl{Set}     &  yes             &   yes         &  yes\\
\cline{2-5}
&\textsl{Prop}    & if $R$ singleton &   yes         &  no\\
\cline{2-5}
&\textsl{Type}    &  yes             &   yes         &  yes\\
\hline
\end{tabular}
\end{center}

\section{Some Proof Techniques Based on Case Analysis}
\label{CaseTechniques}

In this section we illustrate the use of case analysis as a proof
principle, explaining the proof techniques behind three very useful
{\coq} tactics, called \texttt{discriminate}, \texttt{injection} and
\texttt{inversion}. 

\subsection{Discrimination of introduction rules}
\label{Discrimination}

In the informal semantics of recursive types described in Section
\ref{Introduction} it was said that each of the introduction rules of a
recursive type is considered as being different from all the others. 
It is possible to capture this fact inside the logical system using
the propositional equality. We take as example the following theorem,
stating that \textsl{O} constructs a natural number different 
from any of those constructed with \texttt{S}. 

\begin{alltt}
Theorem S_is_not_O : {\prodsym} n, S n {\coqdiff} 0. 
\end{alltt}

In order to prove this theorem, we first define a proposition by case
analysis on natural numbers, so that the proposition is true for {\Z}
and false for any natural number constructed with {\SUCC}. This uses
the empty and singleton type introduced in Sections \ref{Introduction}.

\begin{alltt}
Definition Is_zero (x:nat):= match x with 
                                     | 0 {\funarrow} True  
                                     | _ {\funarrow} False
                             end.
\end{alltt}

\noindent Then, we prove the following lemma:

\begin{alltt}
Lemma O_is_zero : {\prodsym} m, m = 0 {\arrow} Is_zero m.
Proof.
  intros m H; subst m. 
\it{}
================
 Is_zero 0
\tt{}
simpl;trivial.
Qed.
\end{alltt}

\noindent Finally, the proof of \texttt{S\_is\_not\_O} follows by the
application of the previous lemma to $S\;n$.


\begin{alltt}

 red; intros n Hn.
 \it{}   
  n : nat
  Hn : S n = 0
  ============================
   False \tt

 apply O_is_zero with (m := S n).
 assumption.
Qed.
\end{alltt}


The tactic \texttt{discriminate} \refmancite{Section \ref{Discriminate}} is
a special-purpose tactic for proving disequalities between two
elements of a recursive type introduced by different constructors. It
generalizes the proof method described here for natural numbers to any
[co]-inductive type. This tactic is also capable of proving disequalities
where the difference is not in the constructors at the head of the
terms, but deeper inside them. For example, it can be used to prove
the following theorem:

\begin{alltt}
Theorem disc2 : {\prodsym} n, S (S n) {\coqdiff} 1. 
Proof.
 intros n Hn; discriminate.
Qed.
\end{alltt}

When there is an assumption $H$ in the context stating a false
equality $t_1=t_2$, \texttt{discriminate} solves the goal by first
proving $(t_1\not =t_2)$ and then reasoning by absurdity with respect
to $H$:

\begin{alltt}
Theorem disc3 : {\prodsym} n, S (S n) = 0 {\arrow} {\prodsym} Q:Prop, Q.
Proof.
 intros n Hn Q.
 discriminate.
Qed.
\end{alltt}

\noindent In this case, the proof proceeds by absurdity with respect
to the false equality assumed, whose negation is proved by
discrimination.

\subsection{Injectiveness of introduction rules}

Another useful property about recursive types is the
\textsl{injectiveness} of introduction rules, i.e., that whenever two
objects were built using the same introduction rule, then this rule
should have been applied to the same element. This can be stated
formally using the propositional equality:

\begin{alltt}
Theorem inj : {\prodsym} n m, S n = S m {\arrow} n = m.
Proof.
\end{alltt}

\noindent This theorem is just a corollary of a lemma about the
predecessor function:

\begin{alltt}
 Lemma inj_pred : {\prodsym} n m, n = m {\arrow} pred n = pred m.
 Proof.
  intros n m eq_n_m.
  rewrite eq_n_m.
  trivial.
 Qed.
\end{alltt}
\noindent Once this lemma is proven, the theorem follows directly
from it:
\begin{alltt}
 intros n m eq_Sn_Sm.
 apply inj_pred with (n:= S n) (m := S m); assumption.
Qed.
\end{alltt}

This proof method is implemented by the tactic \texttt{injection}
\refmancite{Section \ref{injection}}. This tactic is applied to
a term $t$ of type ``~$c\;{t_1}\;\dots\;t_n = c\;t'_1\;\dots\;t'_n$~'', where $c$ is some constructor of
an inductive type. The tactic \texttt{injection} is applied as deep as 
possible to derive the equality of all pairs of subterms of $t_i$ and $t'_i$
placed in the same position. All these equalities are put as antecedents 
of the current goal.



Like \texttt{discriminate}, the tactic \citecoq{injection} 
can be also applied if $x$ does not
occur in a direct sub-term, but somewhere deeper inside it. Its
application may leave some trivial goals that can be easily solved
using the tactic \texttt{trivial}.

\begin{alltt}

 Lemma list_inject : {\prodsym} (A:Type)(a b :A)(l l':list A),
             a :: b :: l = b :: a :: l' {\arrow} a = b {\coqand} l = l'.
Proof.
 intros A a b l l' e.


\it
  e : a :: b :: l = b :: a :: l'
  ============================
   a = b {\coqand} l = l'
\tt
 injection e.
\it
  ============================
   l = l' {\arrow} b = a {\arrow} a = b {\arrow} a = b {\coqand} l = l'

\tt{} auto.
Qed.
\end{alltt}

\subsection{Inversion Techniques}\label{inversion}

In section \ref{DependentCase}, we motivated the rule of dependent case
analysis as a way of internalizing the informal equalities $n=O$ and
$n=\SUCC\;p$ associated to each case. This internalisation
consisted in instantiating $n$ with the corresponding term in the type
of each branch. However, sometimes it could be better to internalise
these equalities as extra hypotheses --for example, in order to use
the tactics \texttt{rewrite}, \texttt{discriminate} or
\texttt{injection} presented in the previous sections. This is
frequently the case when the element analysed is denoted by a term
which is not a variable, or when it is an object of a particular
instance of a recursive family of types. Consider for example the
following theorem:

\begin{alltt}
Theorem not_le_Sn_0 : {\prodsym} n:nat, ~ (S n {\coqle} 0).
\end{alltt}

\noindent Intuitively, this theorem should follow by case analysis on
the hypothesis $H:(S\;n\;\leq\;\Z)$, because no introduction rule allows
to instantiate the arguments of \citecoq{le} with respectively a successor
and zero. However, there
is no way of capturing this with the typing rule for case analysis
presented in section \ref{Introduction}, because it does not take into
account what particular instance of the family the type of $H$ is.
Let us try it:
\begin{alltt}
Proof.
 red; intros n H; case H.
\it 2 subgoals
  
  n : nat
  H : S n {\coqle} 0
  ============================
   False

subgoal 2 is:
 {\prodsym} m : nat, S n {\coqle} m {\arrow} False
\tt
Undo.
\end{alltt}

\noindent What is necessary here is to make available the equalities
``~$\SUCC\;n = \Z$~'' and ``~$\SUCC\;m = \Z$~''
 as extra hypotheses of the
branches, so that the goal can be solved using the
\texttt{Discriminate} tactic. In order to obtain the desired
equalities as hypotheses, let us prove an auxiliary lemma, that our
theorem is a corollary of:

\begin{alltt}
 Lemma not_le_Sn_0_with_constraints :
  {\prodsym} n p , S n {\coqle} p {\arrow}  p = 0 {\arrow} False.
  Proof.
   intros n p H; case H .
\it
2 subgoals
  
  n : nat
  p : nat
  H : S n {\coqle} p
  ============================
   S n = 0 {\arrow} False

subgoal 2 is:
 {\prodsym} m : nat, S n {\coqle} m {\arrow} S m = 0 {\arrow} False
\tt
 intros;discriminate.
 intros;discriminate.
Qed.
\end{alltt}
\noindent Our main theorem can now be solved by an application of this lemma:
\begin{alltt}
Show.
\it
2 subgoals
  
  n : nat
  p : nat
  H : S n {\coqle} p
  ============================
   S n = 0 {\arrow} False

subgoal 2 is:
 {\prodsym} m : nat, S n {\coqle} m {\arrow} S m = 0 {\arrow} False
\tt
 eapply not_le_Sn_0_with_constraints; eauto.
Qed.
\end{alltt}


The general method to address such situations consists in changing the
goal to be proven into an implication, introducing as preconditions
the equalities needed to eliminate the cases that make no
sense. This proof technique is implemented by the tactic
\texttt{inversion} \refmancite{Section \ref{Inversion}}. In order
to prove a goal $G\;\vec{q}$ from an object of type $R\;\vec{t}$,
this tactic automatically generates a lemma $\forall, \vec{x}.
(R\;\vec{x}) \rightarrow \vec{x}=\vec{t}\rightarrow \vec{B}\rightarrow
(G\;\vec{q})$, where the list of propositions $\vec{B}$ correspond to
the subgoals that cannot be directly proven using
\texttt{discriminate}. This lemma can either be saved for later
use, or generated interactively. In this latter case, the subgoals
yielded by the tactic are the hypotheses $\vec{B}$ of the lemma. If the
lemma has been stored, then the tactic \linebreak
 ``~\citecoq{inversion \dots using \dots}~'' can be
used to apply it. 

Let us show both techniques on our previous example:

\subsubsection{Interactive mode}

\begin{alltt}
Theorem not_le_Sn_0' : {\prodsym} n:nat, ~ (S n {\coqle} 0).
Proof.
 red; intros n H ; inversion H.
Qed.
\end{alltt}


\subsubsection{Static mode}

\begin{alltt}

Derive Inversion le_Sn_0_inv with ({\prodsym} n :nat, S n {\coqle}  0).
Theorem le_Sn_0'' : {\prodsym} n p : nat, ~ S n {\coqle} 0 .
Proof.
 intros n p H; 
 inversion H using le_Sn_0_inv.
Qed.
\end{alltt}


In the example above, all the cases are solved using discriminate, so
there remains no subgoal to be proven (i.e. the list $\vec{B}$ is
empty). Let us present a second example, where this list is not empty:


\begin{alltt}
TTheorem le_reverse_rules : 
     {\prodsym} n m:nat, n {\coqle} m {\arrow} 
                     n = m {\coqor}  
                     {\exsym} p, n {\coqle}  p {\coqand} m = S p.
Proof.
 intros n m H; inversion H.
\it
2 subgoals



  
  n : nat
  m : nat
  H : n {\coqle} m
  H0 : n = m
  ============================
   m = m {\coqor} ({\exsym} p : nat, m {\coqle} p {\coqand} m = S p)

subgoal 2 is:
 n = S m0 {\coqor} ({\exsym} p : nat, n {\coqle} p {\coqand} S m0 = S p)
\tt
 left;trivial.
 right; exists m0; split; trivial.
\it
Proof completed
\end{alltt}

This example shows how this tactic can be used to ``reverse'' the
introduction rules of a recursive type, deriving the possible premises
that could lead to prove a given instance of the predicate. This is
why these tactics are called \texttt{inversion} tactics: they go back
from conclusions to premises.

The hypotheses corresponding to the propositional equalities are not
needed in this example, since the tactic does the necessary rewriting
to solve the subgoals.  When the equalities are no longer needed after
the inversion, it is better to use the tactic
\texttt{Inversion\_clear}. This variant of the tactic clears from the
context all the equalities introduced.

\begin{alltt}
Restart.
 intros n m H; inversion_clear H.
\it
\it
  
  n : nat
  m : nat
  ============================
   m = m {\coqor} ({\exsym} p : nat, m {\coqle} p {\coqand} m = S p)
\tt
 left;trivial.
\it
  n : nat
  m : nat
  m0 : nat
  H0 : n {\coqle} m0
  ============================
   n = S m0 {\coqor} ({\exsym} p : nat, n {\coqle} p {\coqand} S m0 = S p)
\tt
 right; exists m0; split; trivial.
Qed.
\end{alltt}


%This proof technique works in most of the cases, but not always.  In
%particular, it could not if the list $\vec{t}$ contains a term $t_j$
%whose type $T$ depends on a previous term $t_i$, with $i<j$. Remark
%that if this is the case, the propositional equality $x_j=t_j$ is not
%well-typed, since $x_j:T(x_i)$ but $t_j:T(t_i)$, and both types are
%not convertible (otherwise, the problem could be solved using the
%tactic \texttt{Case}).



\begin{exercise}
Consider the following language of arithmetic expression, and 
its operational semantics, described by a set of rewriting rules.
%\textbf{J'ai enlev� une r�gle de commutativit� de l'addition qui 
%me paraissait bizarre du point de vue de la s�mantique op�rationnelle}

\begin{alltt}
Inductive ArithExp : Set :=
   | Zero : ArithExp 
   | Succ : ArithExp {\arrow} ArithExp
   | Plus : ArithExp {\arrow} ArithExp {\arrow} ArithExp.

Inductive RewriteRel : ArithExp {\arrow} ArithExp {\arrow} Prop :=
  |  RewSucc  : {\prodsym} e1 e2 :ArithExp,
                  RewriteRel e1 e2 {\arrow}
                   RewriteRel (Succ e1) (Succ e2) 
  |  RewPlus0 : {\prodsym} e:ArithExp,
                  RewriteRel (Plus Zero e) e 
  |  RewPlusS : {\prodsym} e1 e2:ArithExp,
                  RewriteRel e1 e2 {\arrow}
                  RewriteRel (Plus (Succ e1) e2) 
                             (Succ (Plus e1 e2)).

\end{alltt}
\begin{enumerate}
\item Prove that \texttt{Zero} cannot be rewritten any further.
\item Prove that an expression of the form ``~$\texttt{Succ}\;e$~'' is always 
rewritten
into an expression of the same form.
\end{enumerate}
\end{exercise}

%Theorem zeroNotCompute : (e:ArithExp)~(RewriteRel Zero e).
%Intro e.
%Red.
%Intro H.
%Inversion_clear H.
%Defined.
%Theorem evalPlus : 
%  (e1,e2:ArithExp)
%    (RewriteRel (Succ e1) e2)\arrow{}(EX e3 : ArithExp | e2=(Succ e3)).
%Intros e1 e2 H.
%Inversion_clear H.
%Exists e3;Reflexivity.
%Qed.


\section{Inductive Types and Structural Induction} 
\label{StructuralInduction}

Elements of inductive types  are well-founded with
respect to the structural order induced by the constructors of the
type. In addition to case analysis, this extra hypothesis about
well-foundedness justifies a stronger elimination rule for them, called
\textsl{structural induction}.  This form of elimination consists in
defining a value ``~$f\;x$~'' from some element $x$ of the inductive type
$I$, assuming that values have been already associated in the same way
to the sub-parts of $x$ of type $I$.


Definitions by structural induction are expressed through the
\texttt{Fixpoint} command \refmancite{Section
\ref{Fixpoint}}. This command is quite close to the
\texttt{let-rec} construction of functional programming languages.
For example, the following definition introduces the addition of two
natural numbers (already defined in the Standard Library:)

\begin{alltt} 
Fixpoint plus (n p:nat) \{struct n\} : nat :=
  match n with
          | 0 {\funarrow} p
          | S m {\funarrow} S (plus m p)
 end.
\end{alltt}

The definition is by structural induction on the first argument of the
function. This is indicated  by the ``~\citecoq{\{struct n\}}~''
directive in the function's header\footnote{This directive is optional
in the case of a function of a single argument}.
  In
order to be accepted, the definition must satisfy a syntactical
condition, called the \textsl{guardedness condition}. Roughly
speaking, this condition constrains the arguments of a recursive call
to be pattern variables, issued from a case analysis of the formal
argument of the function pointed by the \texttt{struct} directive.
 In the case of the
function \texttt{plus}, the argument \texttt{m} in the recursive call is a
pattern variable issued from a case analysis of \texttt{n}. Therefore, the
definition is accepted.

Notice that we could have defined the addition with structural induction 
on its second argument:
\begin{alltt} 
Fixpoint plus' (n p:nat) \{struct p\} : nat :=
    match p with
          | 0 {\funarrow} n
          | S q {\funarrow} S (plus' n q)
    end.
\end{alltt}

%This notation is useful when defining a function whose decreasing
%argument has a dependent type. As an example, consider the following
%recursivly defined proof of the theorem
%$(n,m:\texttt{nat})n<m \rightarrow (S\;n)<(S\;m)$:
%\begin{alltt} 
%Fixpoint lt_n_S [n,m:nat;p:(lt n m)] : (lt (S n) (S m)) :=
% <[n0:nat](lt (S n) (S n0))>
%     Cases p of
%       lt_intro1        {\funarrow} (lt_intro1 (S n))
%    | (lt_intro2 m1 p2) {\funarrow} (lt_intro2 (S n) (S m1) (lt_n_S n m1 p2))
%     end.
%\end{alltt}

%The guardedness condition must be satisfied only by the last argument
%of the enclosed list. For example, the following declaration is an
%alternative way of defining addition:

%\begin{alltt}
%Reset add.
%Fixpoint add [n:nat] : nat\arrow{}nat :=
%  Cases n of 
%       O    {\funarrow} [x:nat]x
%    | (S m) {\funarrow} [x:nat](add m (S x))
%  end.
%\end{alltt}

In the following definition of addition, 
the second argument of \verb@plus''@ grows at each
recursive call. However, as the first one always decreases, the
definition is sound.
\begin{alltt}
Fixpoint plus'' (n p:nat) \{struct n\} : nat :=
 match n with
          | 0 {\funarrow} p
          | S m {\funarrow} plus'' m (S p)
 end.
\end{alltt}

 Moreover, the argument in the recursive call
could be a deeper component of $n$.  This is the case in the following
definition of a boolean function determining whether a number is even
or odd:

\begin{alltt} 
Fixpoint even_test (n:nat) : bool :=
  match n 
  with 0 {\funarrow}  true
     | 1 {\funarrow}  false
     | S (S p) {\funarrow} even_test p
  end.
\end{alltt}

Mutually dependent definitions by structural induction are also
allowed. For example, the previous function \textsl{even} could alternatively
be defined using an auxiliary function \textsl{odd}:

\begin{alltt}
Reset even_test.



Fixpoint even_test (n:nat) : bool :=
  match n 
  with 
      | 0 {\funarrow}  true
      | S p {\funarrow} odd_test p
  end
with odd_test (n:nat) : bool :=
  match n
  with 
     | 0 {\funarrow} false
     | S p {\funarrow} even_test p
 end.
\end{alltt}

%\begin{exercise}
%Define a function by structural induction that computes the number of
%nodes of a tree structure defined in page \pageref{Forest}.
%\end{exercise}

Definitions by structural induction are computed 
 only when they are applied, and the decreasing argument
is a term having a constructor at the head. We can check this using
the \texttt{Eval} command, which computes the normal form of a well
typed term.

\begin{alltt}
Eval simpl in even_test.
\it
    = even_test
     : nat {\arrow} bool
\tt 
Eval simpl in (fun x : nat {\funarrow} even x).
\it
     = fun x : nat {\funarrow} even x
     : nat {\arrow} Prop
\tt
Eval simpl in (fun x : nat => plus 5 x).
\it
     =  fun x : nat {\funarrow} S (S (S (S (S x))))

\tt
Eval simpl in (fun x : nat {\funarrow} even_test (plus 5 x)).
\it
    = fun x : nat {\funarrow} odd_test x
     : nat {\arrow} bool
\tt
Eval simpl in (fun x : nat {\funarrow} even_test (plus x 5)).
\it
    = fun x : nat {\funarrow} even_test (x + 5)
     : nat {\arrow} bool
\end{alltt}
 

%\begin{exercise}
%Prove that the second definition of even satisfies the following
%theorem:
%\begin{verbatim}
%Theorem unfold_even : 
% (x:nat)
%   (even x)= (Cases x of 
%                 O        {\funarrow} true
%              | (S O)     {\funarrow} false 
%              | (S (S m)) {\funarrow} (even m)
%              end).
%\end{verbatim}
%\end{exercise}

\subsection{Proofs by Structural Induction}

The principle of structural induction can be also used in order to
define proofs, that is, to prove theorems. Let us call an
\textsl{elimination combinator} any function that, given a predicate
$P$, defines a proof of ``~$P\;x$~'' by structural induction on $x$.  In
{\coq}, the principle of proof by induction on natural numbers is a
particular case of an elimination combinator. The definition of this
combinator depends on three general parameters: the predicate to be
proven, the base case, and the inductive step:

\begin{alltt}
Section Principle_of_Induction.
Variable    P               : nat {\arrow} Prop.
Hypothesis  base_case       : P 0.
Hypothesis  inductive_step  : {\prodsym} n:nat, P n {\arrow} P (S n).
Fixpoint nat_ind  (n:nat)   : (P n) := 
   match n return P n with
          | 0 {\funarrow} base_case
          | S m {\funarrow} inductive_step m (nat_ind m)
   end. 

End Principle_of_Induction.
\end{alltt}

As this proof principle is used very often, {\coq} automatically generates it
when an inductive type is introduced.  Similar principles
\texttt{nat\_rec} and \texttt{nat\_rect} for defining objects in the
universes $\Set$ and $\Type$ are also automatically generated
\footnote{In fact, whenever possible, {\coq} generates the
principle \texttt{$I$\_rect}, then derives from it the
weaker principles  \texttt{$I$\_ind} and \texttt{$I$\_rec}.
If some principle has to be defined by hand, the user may try
to build \texttt{$I$\_rect} (if possible). Thanks to {\coq}'s conversion
rule, this principle can be used directly to build proofs and/or
programs.}. The
command \texttt{Scheme} \refmancite{Section \ref{Scheme}} can be
used to generate an elimination combinator from certain parameters,
like the universe that the defined objects must inhabit, whether the
case analysis in the definitions must be dependent or not, etc. For
example, it can be used to generate an elimination combinator for
reasoning on even natural numbers from the mutually dependent
predicates introduced in page \pageref{Even}. We do not display the
combinators here by lack of space, but you can see them using the
\texttt{Print} command.

\begin{alltt}
Scheme Even_induction := Minimality for even Sort Prop
with   Odd_induction  := Minimality for odd  Sort Prop.
\end{alltt}

\begin{alltt}
Theorem even_plus_four : {\prodsym} n:nat, even n {\arrow} even (4+n).
Proof.
 intros n H.
 elim H using Even_induction with (P0 := fun n {\funarrow} odd (4+n));
 simpl;repeat constructor;assumption.
Qed.
\end{alltt}

Another example of an elimination combinator is the principle 
of double induction on natural numbers, introduced by the following
definition:

\begin{alltt}
Section Principle_of_Double_Induction.
Variable    P              : nat {\arrow} nat {\arrow}Prop.
Hypothesis  base_case1     : {\prodsym} m:nat, P 0 m.
Hypothesis  base_case2     : {\prodsym} n:nat, P (S n) 0.
Hypothesis  inductive_step : {\prodsym} n m:nat, P n m {\arrow}
                                       \,\, P (S n) (S m).

Fixpoint nat_double_ind (n m:nat)\{struct n\} : P n m := 
 match n, m return P n m with 
 |     0 ,    x   {\funarrow}  base_case1 x 
 |  (S x),    0   {\funarrow} base_case2 x
 |  (S x), (S y) {\funarrow} inductive_step x y (nat_double_ind x y)
 end.
End Principle_of_Double_Induction.
\end{alltt}

Changing the type of $P$ into $\nat\rightarrow\nat\rightarrow\Type$,
another combinator \texttt{nat\_double\_rect} for constructing 
(certified) programs can be defined in exactly the same way.
This definition is left as an exercise.\label{natdoublerect}

\iffalse
\begin{alltt}
Section Principle_of_Double_Recursion.
Variable    P               : nat {\arrow} nat {\arrow} Type.
Hypothesis  base_case1      : {\prodsym} x:nat, P 0 x.
Hypothesis  base_case2      : {\prodsym} x:nat, P (S x) 0.
Hypothesis  inductive_step   : {\prodsym} n m:nat, P n m {\arrow} P (S n) (S m).
Fixpoint nat_double_rect (n m:nat)\{struct n\} : P n m := 
  match n, m return P n m with 
            0 ,     x   {\funarrow}  base_case1 x 
         |  (S x),    0   {\funarrow} base_case2 x
         |  (S x), (S y) {\funarrow} inductive_step x y (nat_double_rect x y)
     end.
End Principle_of_Double_Recursion.
\end{alltt}
\fi
For instance the function computing the minimum of two natural
numbers can be defined in the following way:

\begin{alltt}
Definition min : nat {\arrow} nat {\arrow} nat  := 
  nat_double_rect (fun (x y:nat) {\funarrow} nat)
                 (fun (x:nat) {\funarrow} 0)
                 (fun (y:nat) {\funarrow} 0)
                 (fun (x y r:nat) {\funarrow} S r).
Eval compute in (min 5 8).
\it
= 5 : nat
\end{alltt}


%\begin{exercise}
%
%Define the combinator \texttt{nat\_double\_rec}, and apply it
%to give another definition of \citecoq{le\_lt\_dec} (using the theorems
%of the \texttt{Arith} library).
%\end{exercise}

\subsection{Using Elimination Combinators.} 
The tactic \texttt{apply} can be used to apply one of these proof
principles during the development of a proof. 

\begin{alltt}
Lemma not_circular : {\prodsym} n:nat, n {\coqdiff} S n.
Proof.
 intro n.
 apply nat_ind with (P:= fun n {\funarrow} n {\coqdiff} S n).
\it



2 subgoals
  
  n : nat
  ============================
   0 {\coqdiff} 1


subgoal 2 is:
 {\prodsym} n0 : nat, n0 {\coqdiff} S n0 {\arrow} S n0 {\coqdiff} S (S n0)

\tt
 discriminate.
 red; intros n0 Hn0 eqn0Sn0;injection eqn0Sn0;trivial.
Qed.
\end{alltt}

The tactic \texttt{elim} \refmancite{Section \ref{Elim}} is a
refinement of \texttt{apply}, specially designed for the application
of elimination combinators.  If $t$ is an object of an inductive type
$I$, then ``~\citecoq{elim $t$}~'' tries to find an abstraction $P$ of the
current goal $G$ such that $(P\;t)\equiv G$. Then it solves the goal
applying ``~$I\texttt{\_ind}\;P$~'', where $I$\texttt{\_ind} is the
combinator associated to $I$.  The different cases of the induction
then appear as subgoals that remain to be solved.
In the previous proof, the tactic call ``~\citecoq{apply nat\_ind with (P:= fun n {\funarrow} n {\coqdiff} S n)}~'' can simply be replaced with ``~\citecoq{elim n}~''.

The option ``~\citecoq{\texttt{elim} $t$ \texttt{using} $C$}~''
 allows to use a
derived combinator $C$ instead of the default one. Consider the
following theorem, stating that equality is decidable on natural
numbers:

\label{iseqpage}
\begin{alltt}
Lemma eq_nat_dec : {\prodsym} n p:nat, \{n=p\}+\{n {\coqdiff} p\}.
Proof.
 intros n p.
\end{alltt}

Let us prove this theorem using the combinator \texttt{nat\_double\_rect}
of section~\ref{natdoublerect}. The example also illustrates how
\texttt{elim} may sometimes fail in finding a suitable abstraction $P$
of the goal. Note that if ``~\texttt{elim n}~''
 is used directly on the
goal, the result is not the expected one.

\vspace{12pt}

%\pagebreak
\begin{alltt}
 elim n using nat_double_rect.
\it
4 subgoals
  
  n : nat
  p : nat
  ============================
   {\prodsym} x : nat, \{x = p\} + \{x {\coqdiff} p\}

subgoal 2 is:
 nat {\arrow} \{0 = p\} + \{0 {\coqdiff} p\}

subgoal 3 is:
 nat {\arrow} {\prodsym} m : nat, \{m = p\} + \{m {\coqdiff} p\} {\arrow} \{S m = p\} + \{S m {\coqdiff} p\}

subgoal 4 is:
 nat
\end{alltt}

The four sub-goals obtained do not correspond to the premises that
would be expected for the principle \texttt{nat\_double\_rec}. The
problem comes from the fact that 
this principle for eliminating $n$
has a universally quantified formula as conclusion, which confuses
\texttt{elim} about the right way of abstracting the goal. 

%In effect, let us consider the type of the goal before the call to
%\citecoq{elim}: ``~\citecoq{\{n = p\} + \{n {\coqdiff} p\}}~''.

%Among all the abstractions that can be built by ``~\citecoq{elim n}~''
%let us consider this one
%$P=$\citecoq{fun n :nat {\funarrow} fun q : nat {\funarrow} {\{q= p\} + \{q {\coqdiff} p\}}}.
%It is easy to verify that 
%$P$ has type \citecoq{nat {\arrow} nat {\arrow} Set}, and that, if some 
%$q:\citecoq{nat}$ is given, then $P\;q\;$ matches the current goal.
%Then applying \citecoq{nat\_double\_rec} with $P$ generates
%four goals, corresponding to 
 



Therefore,
in this case the abstraction must be explicited using the tactic
\texttt{pattern}. Once the right abstraction is provided, the rest of
the proof is immediate:

\begin{alltt}
Undo.
 pattern p,n.
\it
  n : nat
  p : nat
  ============================
   (fun n0 n1 : nat {\funarrow} \{n1 = n0\} + \{n1 {\coqdiff} n0\}) p n
\tt
 elim n using nat_double_rec.
\it
3 subgoals
  
  n : nat
  p : nat
  ============================
   {\prodsym} x : nat, \{x = 0\} + \{x {\coqdiff} 0\}

subgoal 2 is:
 {\prodsym} x : nat, \{0 = S x\} + \{0 {\coqdiff} S x\}
subgoal 3 is:
 {\prodsym} n0 m : nat, \{m = n0\} + \{m {\coqdiff} n0\} {\arrow} \{S m = S n0\} + \{S m {\coqdiff} S n0\}

\tt
 destruct x; auto.
 destruct x; auto.
 intros n0 m H; case H.
 intro eq; rewrite eq ; auto.
 intro neg; right; red ; injection 1; auto.
Defined.
\end{alltt}


Notice that the tactic ``~\texttt{decide equality}~''
\refmancite{Section\ref{DecideEquality}} generalises the proof
above to a large class of inductive types.  It can be used for proving
a proposition of the form 
$\forall\,(x,y:R),\{x=y\}+\{x{\coqdiff}y\}$, where $R$ is an inductive datatype
all whose constructors take informative arguments ---like for example
the type {\nat}:

\begin{alltt}
Definition eq_nat_dec' : {\prodsym} n p:nat, \{n=p\} + \{n{\coqdiff}p\}.
 decide equality.
Defined.
\end{alltt}

\begin{exercise}
\begin{enumerate}
\item Define a recursive  function \emph{nat2itree}
mapping any natural number $n$ into an infinitely branching
tree of height $n$.
\item Provide an elimination combinator for these trees.
\item Prove that the relation \citecoq{itree\_le} is a preorder 
(i.e. reflexive and transitive).
\end{enumerate}
\end{exercise}

\begin{exercise} \label{zeroton}
Define the type of lists, and a predicate ``being an ordered list''
using an inductive family. Then, define the function
$(from\;n)=0::1\;\ldots\; n::\texttt{nil}$ and prove that it always generates an
ordered list.
\end{exercise}

\begin{exercise}
Prove that \citecoq{le' n p} and \citecoq{n $\leq$ p} are logically equivalent
for all n and p. (\citecoq{le'} is defined in section \ref{parameterstuff}).
\end{exercise}


\subsection{Well-founded Recursion}
\label{WellFoundedRecursion}

Structural induction is a strong elimination rule for inductive types.
This method can be used to define any function whose termination is
based on the well-foundedness of certain order relation $R$ decreasing
at each recursive call. What makes this principle so strong is the
possibility of reasoning by structural induction on the proof that
certain $R$ is well-founded.  In order to illustrate this we have
first to introduce the predicate of accessibility.

\begin{alltt}
Print Acc.
\it
Inductive Acc (A : Type) (R : A {\arrow} A {\arrow} Prop) (x:A) : Prop :=
    Acc_intro : ({\prodsym} y : A, R y x {\arrow} Acc R y) {\arrow} Acc R x
For Acc: Argument A is implicit
For Acc_intro: Arguments A, R are implicit

\dots
\end{alltt}

\noindent This inductive predicate characterizes those elements $x$ of
$A$ such that any descending $R$-chain $\ldots x_2\;R\;x_1\;R\;x$
starting from $x$ is finite. A well-founded relation is a relation
such that all the elements of $A$ are accessible.  
\emph{Notice the use of parameter $x$ (see Section~\ref{parameterstuff}, page
\pageref{parameterstuff}).}

Consider now the problem of representing in {\coq} the following ML
function $\textsl{div}(x,y)$ on natural numbers, which computes
$\lceil\frac{x}{y}\rceil$ if $y>0$ and yields $x$ otherwise.

\begin{verbatim}
let rec div x y = 
  if x = 0 then 0
  else if y = 0 then x
       else (div (x-y) y)+1;;
\end{verbatim}


The equality test on natural numbers can be represented as the
function \textsl{eq\_nat\_dec} defined page \pageref{iseqpage}. Giving $x$ and
$y$, this function yields either the value $(\textsl{left}\;p)$ if
there exists a proof $p:x=y$, or the value $(\textsl{right}\;q)$ if
there exists $q:a\not = b$. The subtraction function is already
defined in the library \citecoq{Minus}. 

Hence, direct translation of the ML function \textsl{div} would be:

\begin{alltt}
Require Import Minus.

Fixpoint div (x y:nat)\{struct x\}: nat :=
 if eq_nat_dec x 0 
  then 0
  else if eq_nat_dec y 0
       then x
       else S (div (x-y) y).

\it Error:
Recursive definition of div is ill-formed.
In environment
div : nat {\arrow} nat {\arrow} nat
x : nat
y : nat
_ : x {\coqdiff} 0
_ : y {\coqdiff} 0

Recursive call to div has principal argument equal to
"x - y"
instead of a subterm of x
\end{alltt}


The program \texttt{div} is rejected by {\coq} because it does not verify
the syntactical condition to ensure termination. In particular, the
argument of the recursive call is not a pattern variable issued from a
case analysis on $x$. 
We would have the same problem if we had the directive
``~\citecoq{\{struct y\}}~'' instead of ``~\citecoq{\{struct x\}}~''.
However, we know that this program always
stops. One way to justify its termination is to define it by
structural induction on a proof that $x$ is accessible trough the
relation $<$. Notice that any natural number $x$ is accessible
for this relation. In order to do this, it is first necessary to prove
some auxiliary lemmas, justifying that the first argument of
\texttt{div} decreases at each recursive call.

\begin{alltt}
Lemma minus_smaller_S : {\prodsym} x y:nat, x - y < S x.
Proof.
 intros x y; pattern y, x;
 elim x using nat_double_ind.
 destruct x0; auto with arith.
 simpl; auto with arith.
 simpl; auto with arith.
Qed.


Lemma minus_smaller_positive : 
 {\prodsym} x y:nat, x {\coqdiff}0 {\arrow} y {\coqdiff} 0 {\arrow}  x - y < x.
Proof.
 destruct x; destruct y; 
 ( simpl;intros; apply minus_smaller || 
   intros; absurd (0=0); auto).
Qed.
\end{alltt}

\noindent The last two lemmas are necessary to prove that for any pair
of positive natural numbers $x$ and $y$, if $x$ is accessible with
respect to \citecoq{lt}, then so is $x-y$.

\begin{alltt}
Definition minus_decrease : {\prodsym} x y:nat, Acc lt x {\arrow} 
                                         x {\coqdiff} 0 {\arrow} 
                                         y {\coqdiff} 0 {\arrow}
                                         Acc lt (x-y).
Proof.
 intros x y H; case H.
 intros Hz posz posy. 
 apply Hz; apply minus_smaller_positive; assumption.
Defined.
\end{alltt}

Let us take a look at the proof of the lemma \textsl{minus\_decrease}, since
the way in which it has been proven is crucial for what follows.
\begin{alltt}
Print minus_decrease.
\it
minus_decrease = 
fun (x y : nat) (H : Acc lt x) {\funarrow}
match H in (Acc _ y0) return (y0 {\coqdiff} 0 {\arrow} y {\coqdiff} 0 {\arrow} Acc lt (y0 - y)) with
| Acc_intro z Hz {\funarrow}
    fun (posz : z {\coqdiff} 0) (posy : y {\coqdiff} 0) {\funarrow}
    Hz (z - y) (minus_smaller_positive z y posz posy)
end
     : {\prodsym} x y : nat, Acc lt x {\arrow} x {\coqdiff} 0 {\arrow} y {\coqdiff} 0 {\arrow} Acc lt (x - y)

\end{alltt}
\noindent Notice that the function call 
$(\texttt{minus\_decrease}\;n\;m\;H)$
indeed yields an accessibility proof that is \textsl{structurally
smaller} than its argument $H$, because it is (an application of) its
recursive component $Hz$.  This enables to justify the following
definition of \textsl{div\_aux}:

\begin{alltt}
Definition div_aux (x y:nat)(H: Acc lt x):nat.
 fix 3.
 intros.
 refine (if eq_nat_dec x 0 
         then 0 
         else if eq_nat_dec y 0 
              then y
              else div_aux (x-y) y _).
\it
 div_aux : {\prodsym} x : nat, nat {\arrow} Acc lt x {\arrow} nat
  x : nat
  y : nat
  H : Acc lt x
  _ : x {\coqdiff} 0
  _0 : y {\coqdiff} 0
  ============================
   Acc lt (x - y)

\tt
 apply (minus_decrease x y H);auto. 
Defined.
\end{alltt}

The main division function is easily defined, using the theorem
\citecoq{lt\_wf} of the library \citecoq{Wf\_nat}. This theorem asserts that
\citecoq{nat} is well founded w.r.t. \citecoq{lt}, thus any natural number
is accessible.
\begin{alltt}
Definition div x y := div_aux x y (lt_wf x). 
\end{alltt}

Let us explain the proof above. In the definition of \citecoq{div\_aux},
what decreases is not $x$ but the \textsl{proof} of the accessibility
of $x$. The tactic ``~\texttt{fix 3}~'' is used to indicate that the proof
proceeds by structural induction on the third argument of the theorem
--that is, on the accessibility proof. It also introduces a new
hypothesis in the context, named as the current theorem, and with the
same type as the goal. Then, the proof is refined with an incomplete
proof term, containing a hole \texttt{\_}.  This hole corresponds to the proof
of accessibility for $x-y$, and is filled up with the (smaller!)
accessibility proof provided by the function \texttt{minus\_decrease}. 


\noindent Let us take a look to the term \textsl{div\_aux} defined:

\pagebreak
\begin{alltt}
Print div_aux.
\it
div_aux = 
(fix div_aux (x y : nat) (H : Acc lt x) \{struct H\} : nat :=
   match eq_nat_dec x 0 with
   | left _ {\funarrow} 0
   | right _ {\funarrow}
       match eq_nat_dec y 0 with
       | left _ {\funarrow} y
       | right _0 {\funarrow} div_aux (x - y) y (minus_decrease x y H _ _0)
       end
   end)
     : {\prodsym} x : nat, nat {\arrow} Acc lt x {\arrow} nat

\end{alltt}

If the non-informative parts from this proof --that is, the
accessibility proof-- are erased, then we obtain exactly the program
that we were looking for. 
\begin{alltt}

Extraction div.

\it
let div x y =
  div_aux x y
\tt

Extraction div_aux.

\it
let rec div_aux x y =
  match eq_nat_dec x O with
    | Left {\arrow} O
    | Right {\arrow}
        (match eq_nat_dec y O with
           | Left {\arrow} y
           | Right {\arrow} div_aux (minus x y) y)
\end{alltt}

This methodology enables the representation
of any program whose termination can be proved in {\coq}. Once the
expected properties from this program have been verified, the
justification of its termination can be thrown away, keeping just the
desired computational behavior for it.

\section{A case study in dependent elimination}\label{CaseStudy}

Dependent types are very expressive, but ignoring some useful
techniques can cause some problems to the beginner.
Let us consider again the type of vectors (see section~\ref{vectors}).
We want to prove a quite trivial property: the only value of type
``~\citecoq{vector A 0}~'' is ``~\citecoq{Vnil $A$}~''.

Our first naive attempt leads to a \emph{cul-de-sac}.
\begin{alltt}
Lemma vector0_is_vnil : 
  {\prodsym} (A:Type)(v:vector A 0), v = Vnil A.
Proof.
 intros A v;inversion v.
\it
1 subgoal
  
  A : Set
  v : vector A 0
  ============================
   v = Vnil A
\tt
Abort.
\end{alltt}

Another attempt is to do a case analysis on a vector of any length
$n$, under an explicit hypothesis $n=0$. The tactic 
\texttt{discriminate} will help us to get rid of the case 
$n=\texttt{S $p$}$. 
Unfortunately, even the statement of our lemma is refused!

\begin{alltt}
 Lemma vector0_is_vnil_aux : 
 {\prodsym} (A:Type)(n:nat)(v:vector A n), n = 0 {\arrow} v = Vnil A.

\it
Error: In environment
A : Type
n : nat
v : vector A n
e : n = 0
The term "Vnil A" has type "vector A 0" while it is expected to have type
 "vector A n"
\end{alltt}

In effect, the equality ``~\citecoq{v = Vnil A}~'' is ill typed,
because the type ``~\citecoq{vector A n}~'' is not \emph{convertible}
with ``~\citecoq{vector A 0}~''.

This problem can be solved if we consider the heterogeneous 
equality \citecoq{JMeq} \cite{conor:motive}
which allows us to consider terms of different types, even if this
equality can only be proven for terms in the same type.
The axiom \citecoq{JMeq\_eq}, from the library \citecoq{JMeq} allows us to convert a
heterogeneous equality to a standard one.

\begin{alltt}
Lemma vector0_is_vnil_aux : 
   {\prodsym} (A:Type)(n:nat)(v:vector A n), 
      n= 0 {\arrow} JMeq v (Vnil A).
Proof.
 destruct v.
 auto.
 intro; discriminate.
Qed.
\end{alltt}

Our property of vectors of null length can be easily proven:

\begin{alltt}
Lemma vector0_is_vnil : {\prodsym} (A:Type)(v:vector A 0), v = Vnil A.
 intros a v;apply JMeq_eq.
 apply vector0_is_vnil_aux.
 trivial.
Qed.
\end{alltt}

It is interesting to look at another proof of 
\citecoq{vector0\_is\_vnil}, which illustrates a technique developed
and used by various people (consult in the \emph{Coq-club} mailing
list archive the contributions by Yves Bertot, Pierre Letouzey, Laurent Th�ry,
Jean Duprat, and Nicolas Magaud, Venanzio Capretta and Conor McBride).
This technique is also used for unfolding  infinite list definitions
(see chapter13 of~\cite{coqart}).
Notice that this definition does not rely on any axiom (\emph{e.g.} \texttt{JMeq\_eq}).

We first give a new definition of the identity on vectors. Before that,
we make  the use of constructors and selectors lighter thanks to
the implicit arguments feature:

\begin{alltt}
Implicit Arguments Vcons [A n].
Implicit Arguments Vnil [A].
Implicit Arguments Vhead [A n].
Implicit Arguments Vtail [A n].

Definition Vid : {\prodsym} (A : Type)(n:nat), vector A n {\arrow} vector A n.
Proof.
 destruct n; intro v.
 exact Vnil.
 exact (Vcons (Vhead v) (Vtail v)).
Defined.
\end{alltt}


Then we prove that \citecoq{Vid} is the identity on vectors:

\begin{alltt}
Lemma Vid_eq : {\prodsym} (n:nat) (A:Type)(v:vector A n), v=(Vid _ n v).
Proof.
 destruct v.

\it
   A : Type
  ============================
   Vnil = Vid A 0 Vnil

subgoal 2 is:
  Vcons a v = Vid A (S n) (Vcons a v)
\tt
 reflexivity.
 reflexivity.
Defined.
\end{alltt}

Why defining a new identity function on vectors? The following
dialogue shows that \citecoq{Vid} has some interesting computational
properties:

\begin{alltt}
Eval simpl in (fun (A:Type)(v:vector A 0) {\funarrow} (Vid _ _ v)).
\it = fun (A : Type) (_ : vector A 0) {\funarrow} Vnil
     : {\prodsym} A : Type, vector A 0 {\arrow} vector A 0

\end{alltt}

Notice that the plain identity on vectors doesn't convert \citecoq{v}
into \citecoq{Vnil}.
\begin{alltt}
Eval simpl in (fun (A:Type)(v:vector A 0) {\funarrow} v).
\it = fun (A : Type) (v : vector A 0) {\funarrow} v
     : {\prodsym} A : Type, vector A 0 {\arrow} vector A 0
\end{alltt}

Then we prove easily that any vector of length 0 is \citecoq{Vnil}:

\begin{alltt}
Theorem zero_nil : {\prodsym} A (v:vector A 0), v = Vnil.
Proof.
 intros.
 change (Vnil (A:=A)) with (Vid _ 0 v). 
\it
1 subgoal
  
  A : Type
  v : vector A 0
  ============================
   v = Vid A 0 v
\tt
 apply Vid_eq.
Defined.
\end{alltt}

A similar result can be proven about vectors of strictly positive
length\footnote{As for \citecoq{Vid} and \citecoq{Vid\_eq}, this definition 
is from Jean Duprat.}.

\begin{alltt}


Theorem decomp :
  {\prodsym} (A : Type) (n : nat) (v : vector A (S n)),
  v = Vcons (Vhead v) (Vtail v).
Proof.
 intros.
 change (Vcons (Vhead v) (Vtail v)) with (Vid _  (S n) v).
\it
 1 subgoal
  
  A : Type
  n : nat
  v : vector A (S n)
  ============================
   v = Vid A (S n) v

\tt{} apply Vid_eq.
Defined.
\end{alltt}


Both lemmas: \citecoq{zero\_nil} and \citecoq{decomp},
can be used to easily derive a double recursion principle
on vectors of same length:


\begin{alltt}
Definition vector_double_rect : 
    {\prodsym} (A:Type) (P: {\prodsym} (n:nat),(vector A n){\arrow}(vector A n) {\arrow} Type),
        P 0 Vnil Vnil {\arrow}
        ({\prodsym} n (v1 v2 : vector A n) a b, P n v1 v2 {\arrow}
             P (S n) (Vcons a v1) (Vcons  b v2)) {\arrow}
        {\prodsym} n (v1 v2 : vector A n), P n v1 v2.
 induction n.
 intros; rewrite (zero_nil _ v1); rewrite (zero_nil _ v2).
 auto.
 intros v1 v2; rewrite (decomp _ _ v1);rewrite (decomp _ _ v2).
 apply X0; auto.
Defined.
\end{alltt}

Notice that, due to the conversion rule of {\coq}'s type system,
this function can be used directly with \citecoq{Prop} or \citecoq{Type}
instead of type (thus it is useless to build 
\citecoq{vector\_double\_ind} and \citecoq{vector\_double\_rec}) from scratch.

We finish this example with showing how to define the bitwise
\emph{or} on boolean vectors of the same length, 
and proving a little property about this
operation.

\begin{alltt}
Definition bitwise_or n v1 v2 : vector bool n :=
   vector_double_rect 
    bool 
    (fun n v1 v2 {\funarrow} vector bool n)
    Vnil
    (fun n v1 v2 a b r {\funarrow} Vcons (orb a b) r) n v1 v2.
\end{alltt}

Let us  define recursively the $n$-th element of a vector. Notice
that it must be a partial function, in case $n$ is greater or equal
than the length of the vector. Since {\coq} only considers total
functions, the function returns a value in an \emph{option} type.

\begin{alltt}
Fixpoint vector_nth (A:Type)(n:nat)(p:nat)(v:vector A p)
                  \{struct v\}
                  : option A :=
  match n,v  with
    _   , Vnil {\funarrow} None
  | 0   , Vcons b  _ _ {\funarrow} Some b
  | S n', Vcons _  p' v' {\funarrow} vector_nth A n'  p' v'
  end.
Implicit Arguments vector_nth [A p].
\end{alltt}

We can now prove --- using the double induction combinator ---
a simple property relying \citecoq{vector\_nth} and \citecoq{bitwise\_or}:

\begin{alltt}
Lemma nth_bitwise :
   {\prodsym} (n:nat) (v1 v2: vector bool n) i  a b,
      vector_nth i v1 = Some a {\arrow}
      vector_nth i v2 = Some b {\arrow}
      vector_nth i (bitwise_or _ v1 v2) = Some (orb a b).
Proof.
 intros  n v1 v2; pattern n,v1,v2.
 apply vector_double_rect.
 simpl.
 destruct i; discriminate 1.
 destruct i; simpl;auto.
 injection 1; injection 2;intros; subst a; subst b; auto.
Qed.
\end{alltt}


\section{Co-inductive Types and Non-ending Constructions}
\label{CoInduction}

The objects of an inductive type are well-founded with respect to
the constructors of the type. In other words, these objects are built
by applying \emph{a finite number of times} the constructors of the type.
Co-inductive types are obtained by relaxing this condition,
and may contain non-well-founded objects \cite{EG96,EG95a}.  An
example of a co-inductive type is the type of  infinite
sequences formed with elements of type $A$, also called streams.  This
type can be introduced through the following definition:

\begin{alltt}
 CoInductive Stream (A: Type) :Type   := 
 | Cons : A\arrow{}Stream A\arrow{}Stream A.
\end{alltt}

If we are interested in finite or infinite sequences, we consider the type
of \emph{lazy lists}:

\begin{alltt}
CoInductive LList (A: Type) : Type :=
 |  LNil : LList A
 |  LCons : A {\arrow} LList A {\arrow} LList A.
\end{alltt}


It is also possible to define  co-inductive types for the 
trees with infinite branches (see Chapter 13 of~\cite{coqart}).

Structural induction is the way of expressing that inductive types
only contain well-founded objects. Hence, this elimination principle
is not valid for co-inductive types, and the only elimination rule for
streams is case analysis.  This principle can be used, for example, to
define the destructors \textsl{head} and \textsl{tail}.

\begin{alltt}
 Definition head (A:Type)(s : Stream A) := 
   match s with Cons a s' {\funarrow} a end.

 Definition tail (A : Type)(s : Stream A) :=
      match s with Cons a s' {\funarrow} s' end.
\end{alltt}

Infinite objects are defined by means of (non-ending) methods of
construction, like in lazy functional programming languages.  Such
methods can be defined using the \texttt{CoFixpoint} command
\refmancite{Section \ref{CoFixpoint}}. For example, the following
definition introduces the infinite list $[a,a,a,\ldots]$:

\begin{alltt}
 CoFixpoint repeat (A:Type)(a:A) : Stream A := 
   Cons a (repeat a).
\end{alltt}


However, not every co-recursive definition is an admissible method of
construction. Similarly to the case of structural induction, the
definition must verify a \textsl{guardedness} condition to be
accepted. This condition states that any recursive call in the
definition must be protected --i.e, be an argument of-- some
constructor, and only an argument of constructors \cite{EG94a}. The
following definitions are examples of valid methods of construction:

\begin{alltt}
CoFixpoint iterate (A: Type)(f: A {\arrow} A)(a : A) : Stream A:=
    Cons a (iterate f (f a)).

CoFixpoint map 
  (A B:Type)(f: A {\arrow} B)(s : Stream A) : Stream B:=
  match s with Cons a tl {\funarrow} Cons (f a) (map f tl) end.
\end{alltt}

\begin{exercise}
Define two different methods for constructing the stream which 
infinitely alternates the values \citecoq{true} and \citecoq{false}.
\end{exercise}
\begin{exercise}
Using the destructors \texttt{head} and \texttt{tail}, define a function
which takes the n-th element of an infinite stream.
\end{exercise}

A non-ending method of construction is computed lazily. This means
that its definition is unfolded only when the object that it
introduces is eliminated, that is, when it appears as the argument of
a case expression. We can check this using the command
\texttt{Eval}.

\begin{alltt}
Eval simpl in (fun (A:Type)(a:A) {\funarrow} repeat a).
\it  = fun (A : Type) (a : A) {\funarrow} repeat a
     : {\prodsym} A : Type, A {\arrow} Stream A
\tt
Eval simpl in (fun (A:Type)(a:A) {\funarrow} head (repeat a)).
\it  = fun (A : Type) (a : A) {\funarrow} a
     : {\prodsym} A : Type, A {\arrow} A
\end{alltt}

%\begin{exercise}
%Prove the following theorem:
%\begin{verbatim}
%Theorem expand_repeat : (a:A)(repeat a)=(Cons a (repeat a)). 
%\end{verbatim}
%Hint: Prove first the streams version of the lemma in exercise 
%\ref{expand}.
%\end{exercise}

\subsection{Extensional Properties}

Case analysis is also a valid proof principle for infinite
objects. However, this principle is not sufficient to prove
\textsl{extensional} properties, that is, properties concerning the
whole infinite object \cite{EG95a}. A typical example of an
extensional property is the predicate expressing that two streams have
the same elements. In many cases, the minimal reflexive relation $a=b$
that is used as equality for inductive types is too small to capture
equality between streams. Consider for example the streams
$\texttt{iterate}\;f\;(f\;x)$ and
$(\texttt{map}\;f\;(\texttt{iterate}\;f\;x))$. Even though these two streams have
the same elements, no finite expansion of their definitions lead to
equal terms. In other words, in order to deal with extensional
properties, it is necessary to construct infinite proofs. The type of
infinite proofs of equality can be introduced as a co-inductive
predicate, as follows:
\begin{alltt}
CoInductive EqSt (A: Type) : Stream A {\arrow} Stream A {\arrow} Prop :=
  eqst : {\prodsym} s1 s2: Stream A,
      head s1 = head s2 {\arrow}
      EqSt (tail s1) (tail s2) {\arrow}
      EqSt s1 s2.
\end{alltt}

It is possible to introduce proof principles for reasoning about
infinite objects as combinators defined through
\texttt{CoFixpoint}. However, oppositely to the case of inductive
types, proof principles associated to co-inductive types are not
elimination but \textsl{introduction} combinators. An example of such
a combinator is Park's principle for proving the equality of two
streams, usually called the \textsl{principle of co-induction}. It
states that two streams are equal if they satisfy a
\textit{bisimulation}.  A bisimulation is a binary relation $R$ such
that any pair of streams $s_1$ ad $s_2$ satisfying $R$ have equal
heads, and tails also satisfying $R$.  This principle is in fact a
method for constructing an infinite proof:

\begin{alltt}
Section Parks_Principle.
Variable A : Type.
Variable    R      : Stream A {\arrow} Stream A {\arrow} Prop.
Hypothesis  bisim1 : {\prodsym} s1 s2:Stream A, 
                       R s1 s2 {\arrow} head s1 = head s2.

Hypothesis  bisim2 : {\prodsym} s1 s2:Stream A, 
                       R s1 s2 {\arrow} R (tail s1) (tail s2).

CoFixpoint park_ppl     : 
 {\prodsym} s1 s2:Stream A, R s1 s2 {\arrow} EqSt s1 s2 :=
 fun s1 s2 (p : R s1 s2) {\funarrow}
      eqst s1 s2 (bisim1 s1 s2 p) 
                 (park_ppl (tail s1) 
                           (tail s2) 
                           (bisim2 s1 s2 p)).
End Parks_Principle.
\end{alltt}

Let us use the principle of co-induction to prove the extensional
equality mentioned above. 
\begin{alltt}
Theorem map_iterate : {\prodsym} (A:Type)(f:A{\arrow}A)(x:A),
                       EqSt (iterate f (f x)) 
                            (map f (iterate f x)).
Proof.
 intros A f x.
 apply park_ppl with
   (R:= fun s1 s2 {\funarrow}
       {\exsym} x: A, s1 = iterate f (f x) {\coqand} 
                s2 = map f (iterate f x)).

 intros s1 s2 (x0,(eqs1,eqs2));
    rewrite eqs1; rewrite eqs2; reflexivity.
 intros s1 s2 (x0,(eqs1,eqs2)).
 exists (f x0);split;
    [rewrite eqs1|rewrite eqs2]; reflexivity.
 exists x;split; reflexivity.
Qed.
\end{alltt}

The use of Park's principle is sometimes annoying, because it requires
to find an invariant relation and prove that it is indeed a
bisimulation.  In many cases, a shorter proof can be obtained trying
to construct an ad-hoc infinite proof, defined by a guarded
declaration.  The tactic ``~``\texttt{Cofix $f$}~'' can be used to do
that. Similarly to the tactic \texttt{fix} indicated in Section
\ref{WellFoundedRecursion}, this tactic introduces an extra hypothesis
$f$ into the context, whose type is the same as the current goal. Note
that the applications of $f$ in the proof \textsl{must be guarded}. In
order to prevent us from doing unguarded calls, we can define a tactic
that always apply a constructor before using $f$ \refmancite{Chapter
\ref{WritingTactics}} :

\begin{alltt}
Ltac infiniteproof f :=
  cofix f; 
  constructor; 
  [clear f| simpl; try (apply f; clear f)].
\end{alltt}


In the example above, this tactic produces a much simpler proof
that the former one:

\begin{alltt}
Theorem map_iterate' : {\prodsym} ((A:Type)f:A{\arrow}A)(x:A),
                       EqSt (iterate f (f x))
                            (map f (iterate f x)).
Proof.
 infiniteproof map_iterate'.
 reflexivity.
Qed.
\end{alltt}

\begin{exercise}
Define a co-inductive type $Nat$ containing non-standard 
natural numbers --this is, verifying 

$$\exists m  \in \mbox{\texttt{Nat}}, \forall\, n \in \mbox{\texttt{Nat}}, n<m$$.
\end{exercise}

\begin{exercise}
Prove that the extensional equality of streams is an equivalence relation
using Park's co-induction principle.
\end{exercise}


\begin{exercise}
Provide a suitable definition of ``being an ordered list'' for infinite lists
and define a principle for proving that an infinite list is ordered. Apply
this method to the list $[0,1,\ldots ]$. Compare the result with 
exercise \ref{zeroton}.
\end{exercise}

\subsection{About injection, discriminate, and inversion}
Since co-inductive types are closed w.r.t. their constructors,
the techniques shown in Section~\ref{CaseTechniques} work also 
with these types.

Let us consider the type of lazy lists, introduced on page~\pageref{CoInduction}.
The following lemmas are straightforward applications
 of \texttt{discriminate} and \citecoq{injection}:

\begin{alltt}
Lemma Lnil_not_Lcons : {\prodsym} (A:Type)(a:A)(l:LList A),
                               LNil {\coqdiff} (LCons a l).
Proof.
 intros;discriminate.
Qed.

Lemma injection_demo : {\prodsym} (A:Type)(a b : A)(l l': LList A),
                       LCons a (LCons b l) = LCons b (LCons a l') {\arrow}
                       a = b {\coqand} l = l'.
Proof.
 intros A a b l l' e; injection e; auto.
Qed.

\end{alltt}

In order to show \citecoq{inversion} at work, let us define
two predicates on lazy lists:

\begin{alltt}
Inductive Finite (A:Type) : LList A {\arrow} Prop :=
|  Lnil_fin : Finite (LNil (A:=A))
|  Lcons_fin : {\prodsym} a l, Finite l {\arrow} Finite (LCons a l).

CoInductive Infinite  (A:Type) : LList A {\arrow} Prop :=
| LCons_inf : {\prodsym} a l, Infinite l {\arrow} Infinite (LCons a l).
\end{alltt}

\noindent
First, two easy theorems:
\begin{alltt}
Lemma LNil_not_Infinite : {\prodsym} (A:Type), ~ Infinite (LNil (A:=A)).
Proof.
  intros A H;inversion H.
Qed.

Lemma Finite_not_Infinite : {\prodsym} (A:Type)(l:LList A),
                                Finite l {\arrow} ~ Infinite l.
Proof.
 intros A l H; elim H.
 apply LNil_not_Infinite.
 intros a l0 F0 I0' I1.
 case I0'; inversion_clear I1.
 trivial.
Qed.
\end{alltt}


On the other hand, the next proof uses the \citecoq{cofix} tactic. 
Notice the destructuration of \citecoq{l}, which allows us to
apply  the constructor \texttt{LCons\_inf}, thus  satisfying
 the guard condition: 
\begin{alltt}
Lemma Not_Finite_Infinite : {\prodsym} (A:Type)(l:LList A),
                            ~ Finite l {\arrow} Infinite l.
Proof.
 cofix H.
 destruct l.
 intro; 
  absurd (Finite (LNil (A:=A)));
  [auto|constructor].
\it




1 subgoal
  
  H : forall (A : Type) (l : LList A), ~ Finite l -> Infinite l
  A : Type
  a : A
  l : LList A
  H0 : ~ Finite (LCons a l)
  ============================
   Infinite l
\end{alltt}
At this point, one must not apply \citecoq{H}! . It would be possible
to solve the current goal by an inversion of ``~\citecoq{Finite (LCons a l)}~'', but, since the guard condition would be violated, the user
would get an error message after typing \citecoq{Qed}.
In order to satisfy the guard condition, we apply the constructor of
\citecoq{Infinite}, \emph{then} apply \citecoq{H}.

\begin{alltt}
 constructor.
 apply H.
 red; intro H1;case H0.
 constructor.
 trivial.
Qed.
\end{alltt}




The reader is invited to replay this proof and understand each of its steps.


\bibliographystyle{abbrv}
\bibliography{manbiblio,morebib}

\end{document}

