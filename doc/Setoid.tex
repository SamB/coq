\achapter{\protect{The \texttt{setoid$\_$replace} tactic}}
\aauthor{Cl\'ement Renard}
\label{setoid_replace}
\tacindex{setoid\_replace}

This chapter presents  the \texttt{setoid\_replace} tactic.

\asection{Description of \texttt{setoid$\_$replace}}

Working on user-defined structures in \Coq\ is not very easy if
Leibniz equality does not denote the intended equality. For example
using lists to denote finite sets drive to difficulties since two
non convertible terms can denote the same set.

We present here a \Coq\ module, {\tt setoid\_replace}, which allow to
structure and automate some parts of the work. In particular, if
everything has been registered a simple
tactic can do replacement just as if the two terms were equal.

\asection{Adding new setoid or morphisms}

Under the toplevel
load the \texttt{setoid\_replace} files with the command:

\begin{coq_example*}
  Require Setoid.
\end{coq_example*}

A setoid is just a type \verb+A+ and an equivalence relation on \verb+A+.

The specification of a setoid can be found in the file

\begin{quotation}
\begin{verbatim}
theories/Setoids/Setoid.v
\end{verbatim}
\end{quotation}

It looks like :
\begin{small}
\begin{flushleft}
\begin{verbatim}
Section Setoid.

Variable A : Type.
Variable Aeq : A -> A -> Prop.

Record Setoid_Theory : Prop :=
{ Seq_refl : (x:A) (Aeq x x);
  Seq_sym : (x,y:A) (Aeq x y) -> (Aeq y x);
  Seq_trans : (x,y,z:A) (Aeq x y) -> (Aeq y z) -> (Aeq x z)
}.
\end{verbatim}
\end{flushleft}
\end{small}

To define a setoid structure on \verb+A+, you must provide a relation
\verb|Aeq| on \verb+A+ and prove that \verb|Aeq| is an equivalence
relation. That is, you have to define an object of type
\verb|(Setoid_Theory A Aeq)|.

Finally to register a setoid the syntax is:

\comindex{Add Setoid}
\begin{quotation}
  \texttt{Add Setoid} \textit{ A Aeq ST}
\end{quotation}

\noindent where \textit{Aeq} is a term of type \texttt{A->A->Prop} and
\textit{ST} is a term of type 
\texttt{(Setoid\_Theory }\textit{A Aeq}\texttt{)}.

\begin{ErrMsgs}
\item \errindex{Not a valid setoid theory}.\\ 
  That happens when the typing condition does not hold.
\item \errindex{A Setoid Theory is already declared for \textit{A}}.\\
  That happens when you try to declare a second setoid theory for the
  same type.
\end{ErrMsgs}

Currently, only one setoid structure
may be declared for a given type.
This allows automatic detection of the theory used to achieve the
replacement.

The table of setoid theories is compatible with the \Coq\ 
sectioning mechanism. If you declare a setoid inside a section, the
declaration will be thrown away when closing the section.
And when you load a compiled file, all the \texttt{Add Setoid}
commands of this file that are not inside a section will be loaded.

\Warning Only the setoid on \texttt{Prop} is loaded by default with the
\texttt{setoid\_replace} module. The equivalence relation used is
\texttt{iff} {\it i.e.} the logical equivalence.

\asection{Adding new morphisms}

A morphism is nothing else than a function compatible with the
equivalence relation. 
You can only replace a term by an equivalent in position of argument
of a morphism. That's why each morphism has to be
declared to the system, which will ask you to prove the accurate
compatibility lemma.

The syntax is the following :
\comindex{Add Morphism}
\begin{quotation}
  \texttt{Add Morphism} \textit{ f }:\textit{ ident}
\end{quotation}

\noindent where f is the name of a term which type is a non dependent
product (the term you want to declare as a morphism) and
\textit{ident} is a new identifier which will denote the
compatibility lemma.

\begin{ErrMsgs}
\item \errindex{The term \term \ is already declared as a morphism}
\item \errindex{The term \term \ is not a product}
\item \errindex{The term \term \ should not be a dependent product}
\end{ErrMsgs}

The compatibility lemma genereted depends on the setoids already
declared.

\asection{The tactic itself}
\tacindex{setoid\_replace}
\tacindex{setoid\_rewrite}

After having registered all the setoids and morphisms you need, you can
use the tactic called \texttt{setoid\_replace}. The syntax is

\begin{quotation}
\texttt{setoid\_replace} $ term_1$ with $term_2$
\end{quotation}

The effect is similar to the one of \texttt{Replace}.

You also have a tactic called \texttt{setoid\_rewrite} which is the
equivalent of \texttt{Rewrite} for setoids. The syntax is 

\begin{quotation}
\texttt{setoid\_rewrite} \term
\end{quotation}

\begin{Variants}
 \item \texttt{setoid\_rewrite ->} \term
 \item \texttt{setoid\_rewrite <-} \term
\end{Variants}

The arrow tells the systems in which direction the rewriting has to be
done. Moreover, you can use \texttt{Rewrite} for setoid
rewriting. In that case the system will check if the term you give is
an equality or a setoid equivalence and do the appropriate work.

%%% Local Variables: 
%%% mode: latex
%%% TeX-master: "Reference-Manual"
%%% End: 
