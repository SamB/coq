%%%%%%%%%%%%%%%%%%%%%%%%%%%%%%%%%%%%
% File title.tex
% Pretty Headers
% And commands for multiple indexes
%%%%%%%%%%%%%%%%%%%%%%%%%%%%%%%%%%%%
\usepackage{fancyhdr}

\setlength{\headheight}{14pt}

\pagestyle{fancyplain}

\newcommand{\coqfooter}{\tiny Coq Reference Manual, V\coqversion{}, \today}

\cfoot{}
\lfoot[{\coqfooter}]{}
\rfoot[]{{\coqfooter}}

\newcommand{\setheaders}[1]{\rhead[\fancyplain{}{\textbf{#1}}]{\fancyplain{}{\thepage}}\lhead[\fancyplain{}{\thepage}]{\fancyplain{}{\textbf{#1}}}}
\newcommand{\defaultheaders}{\rhead[\fancyplain{}{\leftmark}]{\fancyplain{}{\thepage}}\lhead[\fancyplain{}{\thepage}]{\fancyplain{}{\rightmark}}}

\renewcommand{\chaptermark}[1]{\markboth{{\bf \thechapter~#1}}{}}
\renewcommand{\sectionmark}[1]{\markright{\thesection~#1}}
%BEGIN LATEX
\renewcommand{\contentsname}{%
\protect\setheaders{Table of contents}Table of contents}
\renewcommand{\bibname}{\protect\setheaders{Bibliography}%
\protect\RefManCutCommand{BEGINBIBLIO=\thepage}%
\protect\addcontentsline{toc}{chapter}{Bibliography}Bibliography}
%END LATEX

%%%%%%%%%%%%%%%%%%%%%%%%%%%%%%%%%%%%%%
% For the Addendum table of contents
%%%%%%%%%%%%%%%%%%%%%%%%%%%%%%%%%%%%%%
\newcommand{\aauthor}[1]{{\LARGE \bf #1} \bigskip \bigskip \bigskip}
\makeatletter
%BEGIN LATEX
\newcommand{\atableofcontents}{\section*{Contents}\@starttoc{atoc}}
\newcommand{\achapter}[1]{
  \chapter{#1}\addcontentsline{atoc}{chapter}{#1}}
\newcommand{\asection}[1]{
  \section{#1}\addcontentsline{atoc}{section}{#1}}
\newcommand{\asubsection}[1]{
  \subsection{#1}\addcontentsline{atoc}{subsection}{#1}}
\newcommand{\asubsubsection}[1]{
  \subsubsection{#1}\addcontentsline{atoc}{subsubsection}{#1}}
%END LATEX
%HEVEA \newcommand{\atableofcontents}{}
%HEVEA \newcommand{\achapter}[1]{\chapter{#1}}
%HEVEA \newcommand{\asection}{\section}
%HEVEA \newcommand{\asubsection}{\subsection}
%HEVEA \newcommand{\asubsubsection}{\subsubsection}

%%%%%%%%%%%%%%%%%%%%%%%%%%%%%%%%%%%%%%%%%%%%%%%%%%%%%%%%%
% Reference-Manual.sh is generated to cut the Postscript
%%%%%%%%%%%%%%%%%%%%%%%%%%%%%%%%%%%%%%%%%%%%%%%%%%%%%%%%%
%\@starttoc{sh}
%BEGIN LATEX
\newwrite\RefManCut@out%
\immediate\openout\RefManCut@out\jobname.sh
\newcommand{\RefManCutCommand}[1]{%
\immediate\write\RefManCut@out{#1}}
\newcommand{\RefManCutClose}{%
\immediate\closeout\RefManCut@out}
%END LATEX

%%%%%%%%%%%%%%%%%%%%%%%%%%%%
% Commands for indexes
%%%%%%%%%%%%%%%%%%%%%%%%%%%%
\usepackage{index}
\makeindex
\newindex{tactic}{tacidx}{tacind}{%
\protect\setheaders{Tactics Index}%
\protect\addcontentsline{toc}{chapter}{Tactics Index}Tactics Index}

\newindex{command}{comidx}{comind}{%
\protect\setheaders{Vernacular Commands Index}%
\protect\addcontentsline{toc}{chapter}{Vernacular Commands Index}%
Vernacular Commands Index}

\newindex{error}{erridx}{errind}{%
\protect\setheaders{Index of Error Messages}%
\protect\addcontentsline{toc}{chapter}{Index of Error Messages}Index of Error Messages}

\renewindex{default}{idx}{ind}{%
\protect\addcontentsline{toc}{chapter}{Global Index}%
\protect\setheaders{Global Index}Global Index}

\newcommand{\tacindex}[1]{%
\index{#1@\texttt{#1}}\index[tactic]{#1@\texttt{#1}}}
\newcommand{\comindex}[1]{%
\index{#1@\texttt{#1}}\index[command]{#1@\texttt{#1}}}
\newcommand{\errindex}[1]{\texttt{#1}\index[error]{#1}}
\newcommand{\errindexbis}[2]{\texttt{#1}\index[error]{#2}}
\newcommand{\ttindex}[1]{\index{#1@\texttt{#1}}}
\makeatother



%%% Local Variables: 
%%% mode: latex
%%% TeX-master: "Reference-Manual"
%%% End: 
