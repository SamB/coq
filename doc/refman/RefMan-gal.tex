\chapter{The \gallina{} specification language
\label{Gallina}\index{Gallina}}

This chapter describes \gallina, the specification language of {\Coq}.
It allows to develop mathematical theories and to prove specifications
of programs.  The theories are built from axioms, hypotheses,
parameters, lemmas, theorems and definitions of constants, functions,
predicates and sets. The syntax of logical objects involved in
theories is described in section \ref{term}. The language of
commands, called {\em The Vernacular} is described in section
\ref{Vernacular}.

In {\Coq}, logical objects are typed to ensure their logical
correctness. The rules implemented by the typing algorithm are described in
chapter \ref{Cic}.

\subsection*{About the grammars in the manual
\label{BNF-syntax}\index{BNF metasyntax}}

Grammars are presented in Backus-Naur form (BNF). Terminal symbols are
set in {\tt typewriter font}.  In addition, there are special
notations for regular expressions.

An expression enclosed in square brackets \zeroone{\ldots} means at
most one occurrence of this expression (this corresponds to an
optional component).

The notation ``\nelist{\entry}{sep}'' stands for a non empty
sequence of expressions parsed by {\entry} and
separated by the literal ``{\tt sep}''\footnote{This is similar to the
expression ``{\entry} $\{$ {\tt sep} {\entry} $\}$'' in
standard BNF, or ``{\entry} $($ {\tt sep} {\entry} $)$*'' in
the syntax of regular expressions.}.

Similarly, the notation ``\nelist{\entry}{}'' stands for a non
empty sequence of expressions parsed by the ``{\entry}'' entry,
without any separator between.

At the end, the notation ``\sequence{\entry}{\tt sep}'' stands for a
possibly empty sequence of expressions parsed by the ``{\entry}'' entry,
separated by the literal ``{\tt sep}''.

\section{Lexical conventions
\label{lexical}\index{Lexical conventions}}

\paragraph{Blanks}
Space, newline and horizontal tabulation are considered as blanks.
Blanks are ignored but they separate tokens.

\paragraph{Comments}

Comments in {\Coq} are enclosed between {\tt (*} and {\tt
  *)}\index{Comments}, and can be nested. They can contain any
character. However, string literals must be correctly closed. Comments
are treated as blanks.

\paragraph{Identifiers and access identifiers}

Identifiers, written {\ident}, are sequences of letters, digits,
\verb!_! and \verb!'!, that do not start with a digit or \verb!'!.
That is, they are recognized by the following lexical class:

\index{ident@\ident}
\begin{center}
\begin{tabular}{rcl} 
{\firstletter} & ::= & {\tt a..z} $\mid$ {\tt A..Z} $\mid$ {\tt \_}
% $\mid$ {\tt unicode-letter}  
\\
{\subsequentletter} & ::= & {\tt a..z} $\mid$ {\tt A..Z} $\mid$ {\tt 0..9}
$\mid$ {\tt \_} % $\mid$ {\tt \$}
$\mid$ {\tt '} \\
{\ident} & ::= & {\firstletter} \sequencewithoutblank{\subsequentletter}{}
\end{tabular}
\end{center}
All characters are meaningful. In particular, identifiers are case-sensitive.
Access identifiers, written {\accessident}, are identifiers prefixed
by \verb!.! (dot) without blank. They are used in the syntax of qualified
identifiers.

\paragraph{Natural numbers and integers}
Numerals are sequences of digits. Integers are numerals optionally preceded by a minus sign.

\index{num@{\num}}
\index{integer@{\integer}}
\begin{center}
\begin{tabular}{r@{\quad::=\quad}l}
{\digit} & {\tt 0..9} \\
{\num} & \nelistwithoutblank{\digit}{} \\
{\integer} & \zeroone{\tt -}{\num} \\
\end{tabular}
\end{center}

\paragraph{Strings}
\label{strings}
\index{string@{\qstring}}
Strings are delimited by \verb!"! (double quote), and enclose a
sequence of any characters different from \verb!"! or the sequence
\verb!""! to denote the double quote character. In grammars, the
entry for quoted strings is {\qstring}.

\paragraph{Keywords}
The following identifiers are reserved keywords, and cannot be
employed otherwise:
\begin{center}
\begin{tabular}{llllll}
\verb!_!          &
\verb!as!         &
\verb!at!         &
\verb!cofix!      &
\verb!else!       &
\verb!end!        \\
%
\verb!exists!     &
\verb!exists2!    &
\verb!fix!        &
\verb!for!        &
\verb!forall!     &
\verb!fun!        \\
%
\verb!if!         &
\verb!IF!         &
\verb!in!         &
\verb!let!        &
\verb!match!      &
\verb!mod!        \\
%
\verb!Prop!       &
\verb!return!     &
\verb!Set!        &
\verb!then!       &
\verb!Type!       &
\verb!using!      \\
%
\verb!where!      &
\verb!with!       &
\end{tabular}
\end{center}


\paragraph{Special tokens}
The following sequences of characters are special tokens:
\begin{center}
\begin{tabular}{lllllll}
\verb/!/   &
\verb!%!  &
\verb!&!   &
\verb!&&!  &
\verb!(!   &
\verb!()!  &
\verb!)!   \\
%
\verb!*!   &
\verb!+!   &
\verb!++!  &
\verb!,!   &
\verb!-!   &
\verb!->!  &
\verb!.!   \\
%
\verb!.(!  &
\verb!..!  &
\verb!/!   &
\verb!/\!  &
\verb!:!   &
\verb!::!  &
\verb!:<!  \\
%
\verb!:=!  &
\verb!:>!  &
\verb!;!   &
\verb!<!   &
\verb!<-!  &
\verb!<->! &
\verb!<:!  \\
%
\verb!<=!  &
\verb!<>!  &
\verb!=!   &
\verb!=>!  &
\verb!=_D! &
\verb!>!   &
\verb!>->! \\
%
\verb!>=!  &
\verb!?!   &
\verb!?=!  &
\verb!@!   &
\verb![!   &
\verb!\/!  &
\verb!]!   \\
%
\verb!^!   &
\verb!{!   &
\verb!|!   &
\verb!|-!  &
\verb!||!  &
\verb!}!   &
\verb!~!   \\
\end{tabular}
\end{center}

Lexical ambiguities are resolved according to the ``longest match''
rule: when a sequence of non alphanumerical characters can be decomposed
into several different ways, then the first token is the longest
possible one (among all tokens defined at this moment), and so on.

\section{Terms \label{term}\index{Terms}}

\subsection{Syntax of terms}

Figures \ref{term-syntax} and \ref{term-syntax-aux} describe the basic
set of terms which form the {\em Calculus of Inductive Constructions}
(also called \CIC). The formal presentation of {\CIC} is given in
chapter \ref{Cic}. Extensions of this syntax are given in chapter
\ref{Gallina-extension}. How to customize the syntax is described in
chapter \ref{Addoc-syntax}.

\begin{figure}[htbp]
\begin{centerframe}
\begin{tabular}{lcl@{\qquad}r}
{\term} & ::= &
         {\tt forall} {\binderlist} {\tt ,} {\term}  &(\ref{products})\\
 & $|$ & {\tt fun} {\binderlist} {\tt =>} {\term} &(\ref{abstractions})\\
 & $|$ & {\tt fix} {\fixpointbodies} &(\ref{fixpoints})\\
 & $|$ & {\tt cofix} {\cofixpointbodies} &(\ref{fixpoints})\\
 & $|$ & {\tt let} {\idparams} {\tt :=} {\term}
         {\tt in} {\term} &(\ref{let-in})\\
 & $|$ & {\tt let fix} {\fixpointbody} {\tt in} {\term} &(\ref{fixpoints})\\
 & $|$ & {\tt let cofix} {\cofixpointbody}
         {\tt in} {\term} &(\ref{fixpoints})\\
 & $|$ & {\tt let} {\tt (} \sequence{\name}{,} {\tt )} \zeroone{\ifitem}
         {\tt :=} {\term}
         {\tt in} {\term}  &(\ref{caseanalysis}, \ref{Mult-match})\\
 & $|$ & {\tt if} {\term} \zeroone{\ifitem} {\tt then} {\term}
         {\tt else} {\term} &(\ref{caseanalysis}, \ref{Mult-match})\\
 & $|$ & {\term} {\tt :} {\term} &(\ref{typecast})\\
 & $|$ & {\term} {\tt ->} {\term} &(\ref{products})\\
 & $|$ & {\term} \nelist{\termarg}{}&(\ref{applications})\\
 & $|$ & {\tt @} {\qualid} \sequence{\term}{}
            &(\ref{Implicits-explicitation})\\
 & $|$ & {\term} {\tt \%} {\ident} &(\ref{scopechange})\\
 & $|$ & {\tt match} \nelist{\caseitem}{\tt ,}
                 \zeroone{\returntype} {\tt with} &\\
    &&   ~~~\zeroone{\zeroone{\tt |} \nelist{\eqn}{|}} {\tt end}
    &(\ref{caseanalysis})\\
 & $|$ & {\qualid} &(\ref{qualid})\\
 & $|$ & {\sort} &(\ref{Gallina-sorts})\\
 & $|$ & {\num} &(\ref{numerals})\\
 & $|$ & {\_} &(\ref{hole})\\
 & & &\\
{\termarg} & ::= & {\term} &\\
 & $|$ & {\tt (} {\ident} {\tt :=} {\term} {\tt )}
         &(\ref{Implicits-explicitation})\\
%% & $|$ & {\tt (} {\num} {\tt :=} {\term} {\tt )}
%%         &(\ref{Implicits-explicitation})\\
&&&\\
{\binderlist} & ::= & \nelist{\name}{} {\typecstr} & \ref{Binders} \\
 & $|$ & {\binder} \nelist{\binderlet}{} &\\
&&&\\
{\binder} & ::= &   {\name} & \ref{Binders} \\
 & $|$ & {\tt (} \nelist{\name}{} {\tt :} {\term} {\tt )} &\\  
&&&\\
{\binderlet} & ::= & {\binder} & \ref{Binders} \\
 & $|$ & {\tt (} {\name} {\typecstr} {\tt :=} {\term} {\tt )} &\\
& & &\\
{\name} & ::= & {\ident} &\\
 & $|$ & {\tt \_} &\\
&&&\\
{\qualid} & ::= & {\ident} & \\
 & $|$ & {\qualid} {\accessident} &\\
 & & &\\
{\sort} & ::= & {\tt Prop} ~$|$~ {\tt Set} ~$|$~ {\tt Type} &
\end{tabular}
\end{centerframe}
\caption{Syntax of terms}
\label{term-syntax}
\index{term@{\term}}
\index{sort@{\sort}}
\end{figure}



\begin{figure}[htb]
\begin{centerframe}
\begin{tabular}{lcl}
{\idparams} & ::= &  {\ident} \sequence{\binderlet}{} {\typecstr} \\
&&\\
{\fixpointbodies} & ::= &
         {\fixpointbody} \\
 & $|$ & {\fixpointbody} {\tt with} \nelist{\fixpointbody}{{\tt with}}
         {\tt for} {\ident} \\
{\cofixpointbodies} & ::= &
         {\cofixpointbody} \\
 & $|$ & {\cofixpointbody} {\tt with} \nelist{\cofixpointbody}{{\tt with}}
         {\tt for} {\ident} \\
&&\\
{\fixpointbody} & ::= &
    {\ident} \nelist{\binderlet}{} \zeroone{\annotation} {\typecstr}
    {\tt :=} {\term} \\
{\cofixpointbody} & ::= & {\idparams} {\tt :=} {\term} \\
  & &\\
{\annotation} & ::= & {\tt \{ struct} {\ident} {\tt \}} \\ 
&&\\
{\caseitem} & ::= & {\term} \zeroone{{\tt as} \name}
     \zeroone{{\tt in} \term} \\
&&\\
{\ifitem} & ::= & \zeroone{{\tt as} {\name}} {\returntype} \\
&&\\
{\returntype} & ::= & {\tt return} {\term} \\
&&\\
{\eqn} & ::= & \nelist{\pattern}{\tt ,} {\tt =>} {\term}\\
&&\\
{\pattern} & ::= & {\qualid} \nelist{\pattern}{}  \\
 & $|$ & {\pattern} {\tt as} {\ident}             \\
 & $|$ & {\pattern} {\tt \%} {\ident}         \\
 & $|$ & {\qualid}                              \\
 & $|$ & {\tt \_}                                  \\
 & $|$ & {\num}                                 \\
 & $|$ & {\tt (} \nelist{\pattern}{,} {\tt )}     
\end{tabular}
\end{centerframe}
\caption{Syntax of terms (continued)}
\label{term-syntax-aux}
\end{figure}


%%%%%%%

\subsection{Types}

{\Coq} terms are typed. {\Coq} types are recognized by the same
syntactic class as {\term}. We denote by {\type} the semantic subclass
of types inside the syntactic class {\term}.
\index{type@{\type}}


\subsection{Qualified identifiers and simple identifiers
\label{qualid}
\label{ident}}

{\em Qualified identifiers} ({\qualid}) denote {\em global constants}
(definitions, lemmas, theorems, remarks or facts), {\em global
variables} (parameters or axioms), {\em inductive
types} or {\em constructors of inductive types}.
{\em Simple identifiers} (or shortly {\ident}) are a
syntactic subset of qualified identifiers.  Identifiers may also
denote local {\em variables}, what qualified identifiers do not.

\subsection{Numerals
\label{numerals}}

Numerals have no definite semantics in the calculus. They are mere
notations that can be bound to objects through the notation mechanism
(see chapter~\ref{Addoc-syntax} for details). Initially, numerals are
bound to Peano's representation of natural numbers
(see~\ref{libnats}).

Note: negative integers are not at the same level as {\num}, for this
would make precedence unnatural.

\subsection{Sorts 
\index{Sorts}
\index{Type@{\Type}}
\index{Set@{\Set}}
\index{Prop@{\Prop}}
\index{Sorts}
\label{Gallina-sorts}}

There are three sorts \Set, \Prop\ and \Type.
\begin{itemize}
\item \Prop\ is the universe of {\em logical propositions}.
The logical propositions themselves are typing the proofs.
We denote propositions by {\form}. This constitutes a semantic
subclass of the syntactic class {\term}.
\index{form@{\form}}
\item \Set\ is is the universe of {\em program
types} or {\em specifications}.
The specifications themselves are typing the programs.
We denote specifications by {\specif}. This constitutes a semantic
subclass of the syntactic class {\term}.
\index{specif@{\specif}}
\item {\Type} is the type of {\Set} and {\Prop}
\end{itemize}
\noindent More on sorts can be found in section \ref{Sorts}.

\subsection{Binders
\label{Binders}
\index{binders}}

Various constructions introduce variables which scope is some of its
sub-expressions. There is a uniform syntax for this. A binder may be
an (unqualified) identifier: the name to use to refer to this
variable. If the variable is not to be used, its name can be {\tt
\_}. When its type cannot be synthesized by the system, it can be
specified with notation {\tt (}\,{\ident}\,{\tt :}\,{\type}\,{\tt
)}. There is a notation for several variables sharing the same type:
{\tt (}\,{\ident$_1$}\ldots{\ident$_n$}\,{\tt :}\,{\type}\,{\tt )}.

Some constructions allow ``let-binders'', that is either a binder as
defined above, or a variable with a value. The notation is {\tt
(}\,{\ident}\,{\tt :=}\,{\term}\,{\tt )}. Only one variable can be
introduced at the same time. It is also possible to give the type of
the variable before the symbol {\tt :=}.

The last kind of binders is the ``binder list''. It is either a list
of let-binders (the first one not being a variable with value), or
{\ident$_1$}\ldots{\ident$_n$}\,{\tt :}\,{\type} if all variables
share the same type.

{\Coq} terms are typed. {\Coq} types are recognized by the same
syntactic class as {\term}. We denote by {\type} the semantic subclass
of types inside the syntactic class {\term}.
\index{type@{\type}}

\subsection{Abstractions
\label{abstractions}
\index{abstractions}}

The expression ``{\tt fun} {\ident} {\tt :} \type {\tt =>}~{\term}''
denotes the {\em abstraction} of the variable {\ident} of type
{\type}, over the term {\term}. Put in another way, it is function of
formal parameter {\ident} of type {\type} returning {\term}.

Keyword {\tt fun} is followed by a ``binder list'', so any of the
binders of Section~\ref{Binders} apply. Internally, abstractions are
only over one variable. Multiple variable binders are an iteration of
the single variable abstraction: notation {\tt
fun}~{\ident$_{1}$}~{\ldots}~{\ident$_{n}$}~{\tt :}~\type~{\tt
=>}~{\term} stands for {\tt fun}~{\ident$_{1}$}~{\tt :}~\type~{\tt
=>}~{\ldots}~{\tt fun}~{\ident$_{n}$}~{\tt :}~\type~{\tt =>}~{\term}.
Variables with a value expand to a local definition (see
Section~\ref{let-in}).

\subsection{Products
\label{products}
\index{products}}

The expression ``{\tt forall}~{\ident}~{\tt :}~\type~{\tt ,}~{\term}''
denotes the {\em product} of the variable {\ident} of type {\type},
over the term {\term}. As for abstractions, {\tt forall} is followed
by a binder list, and it is represented by an iteration of single
variable products.

Non dependent product types have a special notation ``$A$ {\tt ->}
$B$'' stands for ``{\tt forall \_:}$A${\tt ,}~$B$''. This is to stress
on the fact that non dependent product types are usual functional types.

\subsection{Applications
\label{applications}
\index{applications}}

The expression \term$_0$ \term$_1$ denotes the application of
  term \term$_0$ to \term$_1$.

The expression {\tt }\term$_0$ \term$_1$ ...  \term$_n${\tt}
denotes the application of the term \term$_0$ to the arguments
\term$_1$ ... then \term$_n$.  It is equivalent to {\tt } {\ldots}
{\tt (} {\term$_0$} {\term$_1$} {\tt )} {\ldots} {\term$_n$} {\tt }:
associativity is to the left.

When using implicit arguments mechanism, implicit positions can be
forced a value with notation {\tt (}\,{\ident}\,{\tt
:=}\,{\term}\,{\tt )} or {\tt (}\,{\num}\,{\tt
:=}\,{\term}\,{\tt )}. See Section~\ref{Implicits-explicitation} for
details.

\subsection{Type cast
\label{typecast}
\index{Cast}}

The expression ``{\term}~{\tt :}~{\type}'' is a type cast
expression. It enforces the type of {\term} to be {\type}.

\subsection{Inferable subterms
\label{hole}
\index{\_}}

Since there are redundancies, a term can be type-checked without
giving it in totality. Subterms that are left to guess by the
type-checker are replaced by ``\_''.


\subsection{Local definitions (let-in)
\label{let-in}
\index{Local definitions}
\index{let-in}}


{\tt let}~{\ident}~{\tt :=}~{\term$_1$}~{\tt in}~{\term$_2$} denotes
the local binding of \term$_1$ to the variable $\ident$ in
\term$_2$. 

There is a syntactic sugar for local definition of functions: {\tt
let} {\ident} {\binder$_1$} \ldots {\binder$_n$} {\tt :=} {\term$_1$}
{\tt in} {\term$_2$} stands for {\tt let} {\ident} {\tt := fun}
{\binder$_1$} \ldots {\binder$_n$} {\tt in} {\term$_2$}.


\subsection{Definition by case analysis
\label{caseanalysis}
\index{match@{\tt match\ldots with\ldots end}}}


This paragraph only shows simple variants of case analysis. See
Section~\ref{Mult-match} and Chapter~\ref{Mult-match-full} for
explanations of the general form.

Objects of inductive types can be destructurated by a case-analysis
construction, also called pattern-matching in functional languages. In
its simple form, a case analysis expression is used to analyze the
structure of an inductive objects (upon which constructor it is
built).

The expression {\tt match} {\term$_0$} {\returntype} {\tt with}
{\pattern$_1$} {\tt =>} {\term$_1$} {\tt $|$} {\ldots} {\tt $|$}
{\pattern$_n$} {\tt =>} {\term$_n$} {\tt end}, denotes a {\em
pattern-matching} over the term {\term$_0$} (expected to be of an
inductive type $I$). {\term$_1$}\ldots{\term$_n$} are called branches. In
a simple pattern \qualid~\nelist{\ident}{}, the qualified identifier
{\qualid} is intended to
be a constructor. There should be a branch for every constructor of
$I$.

The {\returntype} is used to compute the resulting type of the whole
{\tt match} expression and the type of the branches. Most of the time,
when this type is the same as the types of all the {\term$_i$}, the
annotation is not needed\footnote{except if no equation is given, to
match the term in an empty type, e.g. the type \texttt{False}}. This
annotation has to be given when the resulting type of the whole {\tt
match} depends on the actual {\term$_0$} matched.

There are specific notations for case analysis on types with one or
two constructors: {\tt if / then / else} and
{\tt let (}\ldots{\tt ) :=} \ldots {\tt in}\ldots. \SeeAlso
section~\ref{Mult-match} for details and examples.

\SeeAlso Section~\ref{Mult-match} for details and examples.

\subsection{Recursive functions
\label{fixpoints}
\index{fix@{fix \ident$_i$\{\dots\}}}}

Expression ``{\tt fix} \ident$_1$ \binder$_1$ {\tt :} {\type$_1$}
\texttt{:=} \term$_1$ {\tt with} {\ldots} {\tt with} \ident$_n$
\binder$_n$~{\tt :} {\type$_n$} \texttt{:=} \term$_n$ {\tt for}
{\ident$_i$}'' denotes the $i$th component of a block of functions
defined by mutual well-founded recursion. It is the local counterpart
of the {\tt Fixpoint} command. See Section~\ref{Fixpoint} for more
details. When $n=1$, the {\tt for}~{\ident$_i$} is omitted.

The expression ``{\tt cofix} \ident$_1$~\binder$_1$ {\tt :}
{\type$_1$} {\tt with} {\ldots} {\tt with} \ident$_n$ \binder$_n$ {\tt
:} {\type$_n$}~{\tt for} {\ident$_i$}'' denotes the $i$th component of
a block of terms defined by a mutual guarded co-recursion. It is the
local counterpart of the {\tt CoFixpoint} command. See
Section~\ref{CoFixpoint} for more details. When $n=1$, the {\tt
for}~{\ident$_i$} is omitted.

The association of a single fixpoint and a local
definition have a special syntax: ``{\tt let fix}~$f$~{\ldots}~{\tt
  :=}~{\ldots}~{\tt in}~{\ldots}'' stands for ``{\tt let}~$f$~{\tt :=
  fix}~$f$~\ldots~{\tt :=}~{\ldots}~{\tt in}~{\ldots}''. The same
  applies for co-fixpoints.


\section{The Vernacular
\label{Vernacular}}

\begin{figure}[tbp]
\begin{centerframe}
\begin{tabular}{lcl}
{\sentence} & ::= & {\declaration} \\
            & $|$ & {\definition} \\
            & $|$ & {\inductive} \\
            & $|$ & {\fixpoint} \\
            & $|$ & {\statement} \zeroone{\proof} \\
&&\\
%% Declarations
{\declaration} & ::= & {\declarationkeyword} {\assums} {\tt .} \\
&&\\
{\declarationkeyword} & ::= & {\tt Axiom} $|$ {\tt Conjecture} \\
  & $|$  & {\tt Parameter} $|$  {\tt Parameters} \\
  & $|$  & {\tt Variable}  $|$ {\tt Variables}  \\
  & $|$  & {\tt Hypothesis}  $|$ {\tt Hypotheses}\\
&&\\
{\assums} & ::= & \nelist{\ident}{} {\tt :} {\term} \\
          & $|$ & \nelist{\binder}{} \\
&&\\
%% Definitions
{\definition} & ::= & 
         {\tt Definition} {\idparams} {\tt :=} {\term} {\tt .} \\
 & $|$ & {\tt Let} {\idparams} {\tt :=} {\term} {\tt .} \\
&&\\
%% Inductives
{\inductive} & ::= & 
           {\tt Inductive} \nelist{\inductivebody}{with} {\tt .} \\
 & $|$ & {\tt CoInductive} \nelist{\inductivebody}{with} {\tt .} \\
           & & \\
{\inductivebody} & ::= & 
  {\ident} \sequence{\binderlet}{} {\tt :} {\term} {\tt :=} \\
   && ~~~\zeroone{\zeroone{\tt |} \nelist{\idparams}{|}} \\
           & & \\  %% TODO: where ...
%% Fixpoints
{\fixpoint} & ::= & {\tt Fixpoint} \nelist{\fixpointbody}{with} {\tt .} \\
       & $|$ &  {\tt CoFixpoint} \nelist{\cofixpointbody}{with} {\tt .} \\
&&\\
%% Lemmas & proofs
{\statement} & ::= &
  {\statkwd} {\ident} \sequence{\binderlet}{} {\tt :} {\term} {\tt .} \\
&&\\
  {\statkwd} & ::= & {\tt Theorem} $|$ {\tt Lemma} $|$ {\tt Definition} \\
&&\\
{\proof} & ::= & {\tt Proof} {\tt .} {\dots} {\tt Qed} {\tt .}\\
   & $|$ & {\tt Proof} {\tt .} {\dots} {\tt Defined} {\tt .}\\
   & $|$ & {\tt Proof} {\tt .} {\dots} {\tt Admitted} {\tt .}
\end{tabular}
\end{centerframe}
\caption{Syntax of sentences}
\label{sentences-syntax}
\end{figure}

Figure \ref{sentences-syntax} describes {\em The Vernacular} which is the
language of commands of \gallina.  A sentence of the vernacular
language, like in many natural languages, begins with a capital letter
and ends with a dot.

The different kinds of command are described hereafter. They all suppose
that the terms occurring in the sentences are well-typed.

%%
%% Axioms and Parameters
%%
\subsection{Declarations
\index{Declarations}
\label{Declarations}}

The declaration mechanism allows the user to specify his own basic
objects. Declared objects play the role of axioms or parameters in
mathematics. A declared object is an {\ident} associated to a \term. A
declaration is accepted by {\Coq} if and only if this {\term} is a
correct type in the current context of the declaration and \ident\ was
not previously defined in the same module. This {\term} is considered
to be the type, or specification, of the \ident.

\subsubsection{{\tt Axiom {\ident} :{\term} .}
\comindex{Axiom}
\comindex{Parameter}\comindex{Parameters}
\comindex{Conjecture}
\label{Axiom}}

This command links {\term} to the name {\ident} as its specification
in the global context. The fact asserted by {\term} is thus assumed as
a postulate.

\begin{ErrMsgs}
\item \errindex{{\ident} already exists}
\end{ErrMsgs}

\begin{Variants} 
\item {\tt Parameter {\ident} :{\term}.} \\
  Is equivalent to {\tt Axiom {\ident} : {\term}}

\item {\tt Parameter {\ident$_1$}\ldots{\ident$_n$} {\tt :}{\term}.}\\
  Adds $n$ parameters with specification {\term}

\item
 {\tt Parameter\,%
(\,{\ident$_{1,1}$}\ldots{\ident$_{1,k_1}$}\,{\tt :}\,{\term$_1$} {\tt )}\,%
\ldots\,{\tt (}\,{\ident$_{n,1}$}\ldots{\ident$_{n,k_n}$}\,{\tt :}\,%
{\term$_n$} {\tt )}.}\\ 
  Adds $n$ blocks of parameters with different specifications.

\item {\tt Conjecture {\ident} :{\term}.}\\
  Is equivalent to {\tt Axiom {\ident} : {\term}}.
\end{Variants}

\noindent {\bf Remark: } It is possible to replace {\tt Parameter} by
{\tt Parameters}.


\subsubsection{{\tt Variable {\ident} :{\term}}.
\comindex{Variable}
\comindex{Variables}
\comindex{Hypothesis}
\comindex{Hypotheses}}

This command links {\term} to the name {\ident} in the context of the
current section (see Section~\ref{Section} for a description of the section
mechanism). When the current section is closed, name {\ident} will be
unknown and every object using this variable will be explicitly
parameterized (the variable is {\em discharged}). Using the {\tt
Variable} command out of any section is equivalent to {\tt Axiom}.

\begin{ErrMsgs}
\item \errindex{{\ident} already exists}
\end{ErrMsgs}

\begin{Variants}
\item {\tt Variable {\ident$_1$}\ldots{\ident$_n$} {\tt :}{\term}.}\\
  Links {\term} to names {\ident$_1$}\ldots{\ident$_n$}.
\item
 {\tt Variable\,%
(\,{\ident$_{1,1}$}\ldots{\ident$_{1,k_1}$}\,{\tt :}\,{\term$_1$} {\tt )}\,%
\ldots\,{\tt (}\,{\ident$_{n,1}$}\ldots{\ident$_{n,k_n}$}\,{\tt :}\,%
{\term$_n$} {\tt )}.}\\ 
  Adds $n$ blocks of variables with different specifications.
\item {\tt Hypothesis {\ident} {\tt :}{\term}.} \\
  \texttt{Hypothesis} is a synonymous of \texttt{Variable}
\end{Variants}

\noindent {\bf Remark: } It is possible to replace {\tt Variable} by
{\tt Variables} and {\tt Hypothesis} by {\tt Hypotheses}.

It is advised to use the keywords \verb:Axiom: and \verb:Hypothesis:
for logical postulates (i.e. when the assertion {\term} is of sort
\verb:Prop:), and to use the keywords \verb:Parameter: and
\verb:Variable: in other cases (corresponding to the declaration of an
abstract mathematical entity).

%%
%% Definitions
%%
\subsection{Definitions
\index{Definitions}
\label{Simpl-definitions}}

Definitions differ from declarations in allowing to give a name to a
term whereas declarations were just giving a type to a name. That is
to say that the name of a defined object can be replaced at any time
by its definition.  This replacement is called
$\delta$-conversion\index{delta-reduction@$\delta$-reduction} (see
Section~\ref{delta}).  A defined object is accepted by the system if
and only if the defining term is well-typed in the current context of
the definition.  Then the type of the name is the type of term. The
defined name is called a {\em constant}\index{Constant} and one says
that {\it the constant is added to the
environment}\index{Environment}.

A formal presentation of constants and environments is given in
Section~\ref{Typed-terms}.

\subsubsection{\tt Definition {\ident} := {\term}.
\comindex{Definition}}

This command binds the value {\term} to the name {\ident} in the
environment, provided that {\term} is well-typed.

\begin{ErrMsgs}
\item \errindex{{\ident} already exists}
\end{ErrMsgs}

\begin{Variants}
\item {\tt Definition {\ident} {\tt :}{\term$_1$} := {\term$_2$}.}\\
  It checks that the type of {\term$_2$} is definitionally equal to
  {\term$_1$}, and registers {\ident} as being of type {\term$_1$},
  and bound to value {\term$_2$}.
\item {\tt Definition {\ident} {\binder$_1$}\ldots{\binder$_n$}
       {\tt :}\term$_1$ {\tt :=} {\term$_2$}.}\\
  This is equivalent to \\
   {\tt Definition\,{\ident}\,{\tt :\,forall}\,%
       {\binder$_1$}\ldots{\binder$_n$}{\tt ,}\,\term$_1$\,{\tt :=}}\,%
       {\tt fun}\,{\binder$_1$}\ldots{\binder$_n$}\,{\tt =>}\,{\term$_2$}\,%
       {\tt .}
\end{Variants}

\begin{ErrMsgs}
\item \errindex{In environment {\dots} the term: {\term$_2$} does not have type
    {\term$_1$}}.\\
    \texttt{Actually, it has type {\term$_3$}}.
\end{ErrMsgs}

\SeeAlso Sections \ref{Opaque}, \ref{Transparent}, \ref{unfold}

\subsubsection{\tt Let {\ident} := {\term}.
\comindex{Let}}

This command binds the value {\term} to the name {\ident} in the
environment of the current section. The name {\ident} disappears
when the current section is eventually closed, and, all
persistent objects (such as theorems) defined within the
section and depending on {\ident} are prefixed by the local definition
{\tt let {\ident} := {\term} in}.

\begin{ErrMsgs}
\item \errindex{{\ident} already exists}
\end{ErrMsgs}

\begin{Variants}
\item {\tt Let {\ident} : {\term$_1$} := {\term$_2$}.}
\end{Variants}

\SeeAlso Sections \ref{Section} (section mechanism), \ref{Opaque},
\ref{Transparent} (opaque/transparent constants), \ref{unfold}

%%
%% Inductive Types
%%
\subsection{Inductive definitions
\index{Inductive definitions} 
\label{gal_Inductive_Definitions}
\comindex{Inductive}
\label{Inductive}}

We gradually explain simple inductive types, simple
annotated inductive types, simple parametric inductive types, 
mutually inductive types. We explain also co-inductive types.

\subsubsection{Simple inductive types}

The definition of a simple inductive type has the following form:

\medskip
{\tt 
\begin{tabular}{l}
Inductive {\ident} : {\sort} :=  \\
\begin{tabular}{clcl}
   & {\ident$_1$}  &:& {\type$_1$} \\
 | & {\ldots} && \\
 | & {\ident$_n$} &:& {\type$_n$}
\end{tabular}
\end{tabular}
}
\medskip

The name {\ident} is the name of the inductively defined type and
{\sort} is the universes where it lives.
The names {\ident$_1$}, {\ldots}, {\ident$_n$}
are the names of its constructors and {\type$_1$}, {\ldots},
{\type$_n$} their respective types. The types of the constructors have
to satisfy a {\em positivity condition} (see Section~\ref{Positivity})
for {\ident}.  This condition ensures the soundness of the inductive
definition.  If this is the case, the constants {\ident},
{\ident$_1$}, {\ldots}, {\ident$_n$} are added to the environment with
their respective types.  Accordingly to the universe where
the inductive type lives ({\it e.g.} its type {\sort}), {\Coq} provides a
number of destructors for {\ident}.  Destructors are named
{\ident}{\tt\_ind}, {\ident}{\tt \_rec} or {\ident}{\tt \_rect} which
respectively correspond to elimination principles on {\tt Prop}, {\tt
Set} and {\tt Type}.  The type of the destructors expresses structural
induction/recursion principles over objects of {\ident}. We give below
two examples of the use of the {\tt Inductive} definitions.

The set of natural numbers is defined as:
\begin{coq_example}
Inductive nat : Set :=
  | O : nat
  | S : nat -> nat.
\end{coq_example}

The type {\tt nat} is defined as the least \verb:Set: containing {\tt
  O} and closed by the {\tt S} constructor. The constants {\tt nat},
{\tt O} and {\tt S} are added to the environment.

Now let us have a look at the elimination principles. They are three :
{\tt nat\_ind}, {\tt nat\_rec} and {\tt nat\_rect}.  The type of {\tt
  nat\_ind} is:
\begin{coq_example}
Check nat_ind.
\end{coq_example}

This is the well known structural induction principle over natural
numbers, i.e. the second-order form of Peano's induction principle.
It allows to prove some universal property of natural numbers ({\tt
forall n:nat, P n}) by induction on {\tt n}.

The types of {\tt nat\_rec} and {\tt nat\_rect} are similar, except
that they pertain to {\tt (P:nat->Set)} and {\tt (P:nat->Type)}
respectively . They correspond to primitive induction principles
(allowing dependent types) respectively over sorts \verb:Set: and
\verb:Type:. The constant {\ident}{\tt \_ind} is always provided,
whereas {\ident}{\tt \_rec} and {\ident}{\tt \_rect} can be impossible
to derive (for example, when {\ident} is a proposition).

\begin{coq_eval}
Reset Initial.
\end{coq_eval}
\begin{Variants}
\item 
\begin{coq_example*}
Inductive nat : Set := O | S (_:nat).
\end{coq_example*}
In the case where inductive types have no annotations (next section
gives an example of such annotations), the positivity condition
implies that a constructor can be defined by only giving the type of
its arguments.
\end{Variants}

\subsubsection{Simple annotated inductive types}

In an annotated inductive types, the universe where the inductive
type is defined is no longer a simple sort, but what is called an
arity, which is a type whose conclusion is a sort.

As an example of annotated inductive types, let us define the
$even$ predicate:

\begin{coq_example}
Inductive even : nat -> Prop :=
  | even_0 : even O
  | even_SS : forall n:nat, even n -> even (S (S n)).
\end{coq_example}

The type {\tt nat->Prop} means that {\tt even} is a unary predicate
(inductively defined) over natural numbers.  The type of its two
constructors are the defining clauses of the predicate {\tt even}. The
type of {\tt even\_ind} is:

\begin{coq_example}
Check even_ind.
\end{coq_example}

From a mathematical point of view it asserts that the natural numbers
satisfying the predicate {\tt even} are exactly the naturals satisfying
the clauses {\tt even\_0} or {\tt even\_SS}. This is why, when we want
to prove any predicate {\tt P} over elements of {\tt even}, it is
enough to prove it for {\tt O} and to prove that if any natural number
{\tt n} satisfies {\tt P} its double successor {\tt (S (S n))}
satisfies also {\tt P}. This is indeed analogous to the structural
induction principle we got for {\tt nat}.

\begin{ErrMsgs}
\item \errindex{Non strictly positive occurrence of {\ident} in {\type}}
\item \errindex{The conclusion of {\type} is not valid; it must be
built from {\ident}}
\end{ErrMsgs}

\subsubsection{Parameterized inductive types}

Inductive types may be parameterized. Parameters differ from inductive
type annotations in the fact that recursive invokations of inductive
types must always be done with the same values of parameters as its
specification.

The general scheme is:
\begin{center}
{\tt Inductive} {\ident} {\binder$_1$}\ldots{\binder$_k$} : {\term} :=
    {\ident$_1$}: {\term$_1$} | {\ldots} | {\ident$_n$}: \term$_n$
{\tt .}
\end{center}

A typical example is the definition of polymorphic lists:
\begin{coq_example*}
Inductive list (A:Set) : Set :=
  | nil : list A
  | cons : A -> list A -> list A.
\end{coq_example*}

Note that in the type of {\tt nil} and {\tt cons}, we write {\tt
  (list A)} and not just {\tt list}.\\ The constants {\tt nil} and
{\tt cons} will have respectively types:

\begin{coq_example}
Check nil.
Check cons.
\end{coq_example}

Types of destructors are also quantified with {\tt (A:Set)}.

\begin{coq_eval}
Reset Initial.
\end{coq_eval}
\begin{Variants}
\item
\begin{coq_example*}
Inductive list (A:Set) : Set := nil | cons (_:A) (_:list A).
\end{coq_example*}
This is an alternative definition of lists where we specify the
arguments of the constructors rather than their full type.
\end{Variants}

\begin{ErrMsgs}
\item \errindex{The {\num}th argument of {\ident} must be {\ident'} in {\type}}
\end{ErrMsgs}

\SeeAlso Sections~\ref{Cic-inductive-definitions} and~\ref{elim}.


\subsubsection{Mutually defined inductive types
\comindex{Mutual Inductive}
\label{Mutual-Inductive}}

The definition of a block of mutually inductive types has the form:

\medskip
{\tt 
\begin{tabular}{l}
Inductive {\ident$_1$} : {\type$_1$} :=  \\
\begin{tabular}{clcl}
   & {\ident$_1^1$}     &:& {\type$_1^1$} \\
 | & {\ldots} && \\
 | & {\ident$_{n_1}^1$} &:& {\type$_{n_1}^1$}
\end{tabular}  \\
with\\
~{\ldots} \\
with {\ident$_m$} : {\type$_m$} := \\
\begin{tabular}{clcl}
   & {\ident$_1^m$}     &:& {\type$_1^m$} \\
 | & {\ldots} \\
 | & {\ident$_{n_m}^m$} &:& {\type$_{n_m}^m$}.
\end{tabular}
\end{tabular}
}
\medskip

\noindent It has the same semantics as the above {\tt Inductive}
definition for each \ident$_1$, {\ldots}, \ident$_m$. All names
\ident$_1$, {\ldots}, \ident$_m$ and \ident$_1^1$, \dots,
\ident$_{n_m}^m$ are simultaneously added to the environment. Then
well-typing of constructors can be checked. Each one of the
\ident$_1$, {\ldots}, \ident$_m$ can be used on its own.

It is also possible to parameterize these inductive definitions.
However, parameters correspond to a local
context in which the whole set of inductive declarations is done.  For
this reason, the parameters must be strictly the same for each
inductive types The extended syntax is:

\medskip
{\tt 
Inductive {{\ident$_1$} {\params} : {\type$_1$} :=  \\
\mbox{}\hspace{0.4cm}  {\ident$_1^1$} : {\type$_1^1$} \\
\mbox{}\hspace{0.1cm}| ..  \\
\mbox{}\hspace{0.1cm}| {\ident$_{n_1}^1$} : {\type$_{n_1}^1$}  \\
with\\
\mbox{}\hspace{0.1cm} .. \\
with {\ident$_m$} {\params} : {\type$_m$} := \\
\mbox{}\hspace{0.4cm}{\ident$_1^m$} : {\type$_1^m$} \\
\mbox{}\hspace{0.1cm}| .. \\
\mbox{}\hspace{0.1cm}| {\ident$_{n_m}^m$} : {\type$_{n_m}^m$}.
}}
\medskip

\Example
The typical example of a mutual inductive data type is the one for
trees and forests. We assume given two types $A$ and $B$ as variables.
It can be declared the following way.

\begin{coq_eval}
Reset Initial.
\end{coq_eval}
\begin{coq_example*}
Variables A B : Set.
Inductive tree : Set :=
    node : A -> forest -> tree
with forest : Set :=
  | leaf : B -> forest
  | cons : tree -> forest -> forest.
\end{coq_example*}

This declaration generates automatically six induction
principles. They are respectively 
called {\tt tree\_rec}, {\tt tree\_ind}, {\tt
  tree\_rect}, {\tt forest\_rec}, {\tt forest\_ind}, {\tt
  forest\_rect}.  These ones are not the most general ones but are
just the induction principles corresponding to each inductive part
seen as a single inductive definition.

To illustrate this point on our example, we give the types of {\tt
  tree\_rec} and {\tt forest\_rec}.

\begin{coq_example}
Check tree_rec.
Check forest_rec.
\end{coq_example}

Assume we want to parameterize our mutual inductive definitions with
the two type variables $A$ and $B$, the declaration should be done the
following way:

\begin{coq_eval}
Reset tree.
\end{coq_eval}
\begin{coq_example*}
Inductive tree (A B:Set) : Set :=
    node : A -> forest A B -> tree A B
with forest (A B:Set) : Set :=
  | leaf : B -> forest A B
  | cons : tree A B -> forest A B -> forest A B.
\end{coq_example*}

Assume we define an inductive definition inside a section.  When the
section is closed, the variables declared in the section and occurring
free in the declaration are added as parameters to the inductive
definition. 

\SeeAlso Section~\ref{Section}

\subsubsection{Co-inductive types
\label{CoInductiveTypes}
\comindex{CoInductive}}

The objects of an inductive type are well-founded with respect to the
constructors of the type. In other words, such objects contain only a
{\it finite} number constructors. Co-inductive types arise from
relaxing this condition, and admitting types whose objects contain an
infinity of constructors. Infinite objects are introduced by a
non-ending (but effective) process of construction, defined in terms
of the constructors of the type.

An example of a co-inductive type is the type of infinite sequences of
natural numbers, usually called streams. It can be introduced in \Coq\
using the \texttt{CoInductive} command:
\begin{coq_example}
CoInductive Stream : Set :=
    Seq : nat -> Stream -> Stream.
\end{coq_example}

The syntax of this command is the same as the command \texttt{Inductive}
(cf. Section~\ref{gal_Inductive_Definitions}). Notice that no
principle of induction is derived from the definition of a
co-inductive type, since such principles only make sense for inductive
ones. For co-inductive ones, the only elimination principle is case
analysis. For example, the usual destructors on streams
\texttt{hd:Stream->nat} and \texttt{tl:Str->Str} can be defined as
follows:
\begin{coq_example}
Definition hd (x:Stream) := let (a,s) := x in a.
Definition tl (x:Stream) := let (a,s) := x in s.
\end{coq_example}

Definition of co-inductive predicates and blocks of mutually
co-inductive definitions are also allowed. An example of a
co-inductive predicate is the extensional equality on streams:

\begin{coq_example}
CoInductive EqSt : Stream -> Stream -> Prop :=
    eqst :
      forall s1 s2:Stream,
        hd s1 = hd s2 -> EqSt (tl s1) (tl s2) -> EqSt s1 s2.
\end{coq_example}

In order to prove the extensionally equality of two streams $s_1$ and
$s_2$ we have to construct and infinite proof of equality, that is,
an infinite object of type $(\texttt{EqSt}\;s_1\;s_2)$. We will see
how to introduce infinite objects in Section~\ref{CoFixpoint}.

%%
%% (Co-)Fixpoints
%%
\subsection{Definition of recursive functions}

\subsubsection{\tt Fixpoint {\ident} {\params} {\tt \{struct}
  \ident$_0$ {\tt \}} : type$_0$ := \term$_0$ 
\comindex{Fixpoint}
\label{Fixpoint}}

This command allows to define inductive objects using a fixed point
construction. The meaning of this declaration is to define {\it ident}
a recursive function with arguments specified by
{\binder$_1$}\ldots{\binder$_n$} such that {\it ident} applied to
arguments corresponding to these binders has type \type$_0$, and is
equivalent to the expression \term$_0$. The type of the {\ident} is
consequently {\tt forall {\params} {\tt,} \type$_0$}
and the value is equivalent to {\tt fun {\params} {\tt =>} \term$_0$}.

To be accepted, a {\tt Fixpoint} definition has to satisfy some
syntactical constraints on a special argument called the decreasing
argument. They are needed to ensure that the {\tt Fixpoint} definition
always terminates. The point of the {\tt \{struct \ident {\tt \}}}
annotation is to let the user tell the system which argument decreases
along the recursive calls. This annotation may be left implicit for
fixpoints with one argument. For instance, one can define the addition
function as :

\begin{coq_example}
Fixpoint add (n m:nat) {struct n} : nat :=
  match n with
  | O => m
  | S p => S (add p m)
  end.
\end{coq_example}

The {\tt match} operator matches a value (here \verb:n:) with the
various constructors of its (inductive) type. The remaining arguments
give the respective values to be returned, as functions of the
parameters of the corresponding constructor. Thus here when \verb:n:
equals \verb:O: we return \verb:m:, and when \verb:n: equals 
\verb:(S p): we return \verb:(S (add p m)):.

The {\tt match} operator is formally described
in detail in Section~\ref{Caseexpr}.  The system recognizes that in
the inductive call {\tt (add p m)} the first argument actually
decreases because it is a {\em pattern variable} coming from {\tt match
  n with}.

\Example The following definition is not correct and generates an
error message:

\begin{coq_eval}
Set Printing Depth 50.
(********** The following is not correct and should produce **********)
(*********      Error: Recursive call to wrongplus ...      **********)
\end{coq_eval}
\begin{coq_example}
Fixpoint wrongplus (n m:nat) {struct n} : nat :=
  match m with
  | O => n
  | S p => S (wrongplus n p)
  end.
\end{coq_example}

because the declared decreasing argument {\tt n} actually does not
decrease in the recursive call.  The function computing the addition
over the second argument should rather be written:

\begin{coq_example*}
Fixpoint plus (n m:nat) {struct m} : nat :=
  match m with
  | O => n
  | S p => S (plus n p)
  end.
\end{coq_example*}

The ordinary match operation on natural numbers can be mimicked in the
following way.
\begin{coq_example*}
Fixpoint nat_match 
  (C:Set) (f0:C) (fS:nat -> C -> C) (n:nat) {struct n} : C :=
  match n with
  | O => f0
  | S p => fS p (nat_match C f0 fS p)
  end.
\end{coq_example*}
The recursive call may not only be on direct subterms of the recursive
variable {\tt n} but also on a deeper subterm and we can directly
write the function {\tt mod2} which gives the remainder modulo 2 of a
natural number.
\begin{coq_example*}
Fixpoint mod2 (n:nat) : nat :=
  match n with
  | O => O
  | S p => match p with
           | O => S O
           | S q => mod2 q
           end
  end.
\end{coq_example*}
In order to keep the strong normalisation property, the fixed point
reduction will only be performed when the argument in position of the
decreasing argument (which type should be in an inductive definition)
starts with a constructor.

The {\tt Fixpoint} construction enjoys also the {\tt with} extension
to define functions over mutually defined inductive types or more
generally any mutually recursive definitions.

\begin{Variants}
\item {\tt Fixpoint {\ident$_1$} {\params$_1$} :{\type$_1$} := {\term$_1$}\\
        with {\ldots} \\
        with {\ident$_m$} {\params$_m$} :{\type$_m$} :=  {\type$_m$}}\\
        Allows to define simultaneously {\ident$_1$}, {\ldots},
        {\ident$_m$}.
\end{Variants}

\Example 
The size of trees and forests can be defined the following way: 
\begin{coq_eval}
Reset Initial.
Variables A B : Set.
Inductive tree : Set :=
    node : A -> forest -> tree
with forest : Set :=
  | leaf : B -> forest
  | cons : tree -> forest -> forest.
\end{coq_eval}
\begin{coq_example*}
Fixpoint tree_size (t:tree) : nat :=
  match t with
  | node a f => S (forest_size f)
  end
 with forest_size (f:forest) : nat :=
  match f with
  | leaf b => 1
  | cons t f' => (tree_size t + forest_size f')
  end.
\end{coq_example*}
A generic command {\tt Scheme} is useful to build automatically various
mutual induction principles. It is described in Section~\ref{Scheme}.


\subsubsection{\tt Function {\ident} {\params} {\tt \{struct|measure|wf}
  \ident$_0$ {\tt \}} : type$_0$ := \term$_0$.
\comindex{Function}
\label{Function}}

This \emph{experimental} command allows to define a function by
writing its \emph{fixpoint equality}. It is actually a wrapper for
several ways of defining a function and other useful related objects.
The meaning of this declaration is to define a not necessarily
(structural) recursive function {\it ident} with arguments specified
by {\params$_1$}\ldots{\params$_n$} such that {\it ident} applied to
arguments corresponding to these binders has type \type$_0$, and is
\emph{provably equal} to the expression \term$_0$.  The type of the
{\ident} is consequently {\tt forall {\params} {\tt,} \type$_0$}.


\paragraph{If the function is not recursive}
\label{sec:if-function-not-rec}
then it is defined as if declared with \texttt{Definition}, and a
scheme {\tt\ident\_ind} is generated, which reflects the pattern
matching structure of \term$_0$. {\tt \{ \}} flag is ignored.


\paragraph{If the function is structurally recursive}
\label{sec:if-funct-struct}
then the decreasing argument must be declared using the {\tt \{struct}
\ident$_0$ {\tt \}} flag as for a \texttt{Fixpoint} definition. The
function is actually be declared as a fixpoint, and an induction
scheme {\tt\ident\_ind} is generated, which reflects the recursive
structure of \term$_0$.


\paragraph{If the function is not structurally recursive}
\label{sec:if-function-not-struct-rec}
then the decreasing criteria given between {\tt \{ \}} must be one of
the following:
\begin{itemize}
\item {\tt \{measure} \ident$_0$ {\tt \}} with \ident$_0$ being a
  function from type of arguments to \texttt{nat},
\item {\tt \{wf} \ident$_0$ {\tt \}} in which case \ident$_0$ is an
  ordering relation on the type of arguments.
\end{itemize}
. Depending on the decreasing criteria, the user is
left with some proof obligations that will be used to define the
function. These proofs are: proofs that each recursive call is
actually decreasing with respect to the given criteria, and (if the
criteria is \texttt{wf}) a proof that the ordering relation is well
founded.

%Completer sur measure et wf

The fixpoint equality \texttt{\ident\_eq}, which is not trivial to
prove in this case, is automatically generated and proved, together
with an induction scheme {\tt\ident\_ind}, which reflects the recursive
structure of \term$_0$.



%Complete!!
The way this recursive function is defined is the subject of several
papers by Yves Bertot, Julien forest, David Pichardie.

%Exemples ok ici

\paragraph{Limitations}
\label{sec:Function-limitations}
\term$_0$ must be build as an imbrication of pattern matchings
(\texttt{match...with}) with $\lambda$-abstraction and applications only
\emph{at the end} of each branch.

%Exemples faux ici





\subsubsection{\tt CoFixpoint {\ident} : \type$_0$ := \term$_0$.
\comindex{CoFixpoint}
\label{CoFixpoint}}

The {\tt CoFixpoint} command introduces a method for constructing an
infinite object of a coinduc\-tive type. For example, the stream
containing all natural numbers can be introduced applying the
following method to the number \texttt{O} (see
Section~\ref{CoInductiveTypes} for the definition of {\tt Stream},
{\tt hd} and {\tt tl}):
\begin{coq_eval}
Reset Initial.
CoInductive Stream : Set :=
    Seq : nat -> Stream -> Stream.
Definition hd (x:Stream) := match x with
                            | Seq a s => a
                            end.
Definition tl (x:Stream) := match x with
                            | Seq a s => s
                            end.
\end{coq_eval}
\begin{coq_example}
CoFixpoint from (n:nat) : Stream := Seq n (from (S n)).
\end{coq_example}

Oppositely to recursive ones, there is no decreasing argument in a
co-recursive definition. To be admissible, a method of construction
must provide at least one extra constructor of the infinite object for
each iteration. A syntactical guard condition is imposed on
co-recursive definitions in order to ensure this: each recursive call
in the definition must be protected by at least one constructor, and
only by constructors. That is the case in the former definition, where
the single recursive call of \texttt{from} is guarded by an
application of \texttt{Seq}. On the contrary, the following recursive
function does not satisfy the guard condition:

\begin{coq_eval}
Set Printing Depth 50.
(********** The following is not correct and should produce **********)
(***************** Error: Unguarded recursive call *******************)
\end{coq_eval}
\begin{coq_example}
CoFixpoint filter (p:nat -> bool) (s:Stream) : Stream :=
  if p (hd s) then Seq (hd s) (filter p (tl s)) else filter p (tl s).
\end{coq_example}

The elimination of co-recursive definition is done lazily, i.e. the
definition is expanded only when it occurs at the head of an
application which is the argument of a case analysis expression.  In
any other context, it is considered as a canonical expression which is
completely evaluated. We can test this using the command
\texttt{Eval}, which computes the normal forms of a term:

\begin{coq_example}
Eval compute in (from 0).
Eval compute in (hd (from 0)).
Eval compute in (tl (from 0)).
\end{coq_example}

\begin{Variants}
\item{\tt CoFixpoint {\ident$_1$} {\params} :{\type$_1$} :=
  {\term$_1$}}\\ As for most constructions, arguments of co-fixpoints
  expressions can be introduced before the {\tt :=} sign.
\item{\tt CoFixpoint {\ident$_1$} :{\type$_1$} := {\term$_1$}\\
     with\\
        \mbox{}\hspace{0.1cm} $\ldots$  \\
        with {\ident$_m$}   : {\type$_m$} := {\term$_m$}}\\
As in the \texttt{Fixpoint} command (cf. Section~\ref{Fixpoint}), it
is possible to introduce a block of mutually dependent methods.
\end{Variants}

%%
%% Theorems & Lemmas
%%
\subsection{Statement and proofs}

A statement claims a goal of which the proof is then interactively done
using tactics. More on the proof editing mode, statements and proofs can be
found in chapter \ref{Proof-handling}.

\subsubsection{\tt Theorem {\ident} : {\type}.
\comindex{Theorem}
\comindex{Lemma}
\comindex{Remark}
\comindex{Fact}}

This command binds {\type} to the name {\ident} in the
environment, provided that a proof of {\type} is next given.

After a statement, {\Coq} needs a proof.

\begin{Variants} 
\item {\tt Lemma {\ident} : {\type}.}\\ 
It is a synonymous of \texttt{Theorem}
\item {\tt Remark {\ident} : {\type}.}\\ 
It is a synonymous of \texttt{Theorem}
% Same as {\tt Theorem} except
% that if this statement is in one or more levels of sections then the
% name {\ident} will be accessible only prefixed by the sections names
% when the sections (see \ref{Section} and \ref{LongNames}) will be
% closed.
% %All proofs of  persistent objects (such as theorems) referring to {\ident}
% %within the section will be replaced by the proof of  {\ident}.
 \item {\tt Fact {\ident} : {\type}.}\\ 
It is a synonymous of \texttt{Theorem}
% Same as {\tt Remark} except
% that the innermost section name is dropped from the full name.
\item {\tt Definition {\ident} : {\type}.} \\
Allow to define a term of type {\type} using the proof editing mode. It
behaves as {\tt Theorem} but is intended for the interactive
definition of expression which computational behaviour will be used by
further commands. \SeeAlso~\ref{Transparent} and \ref{unfold}. 
\end{Variants}

\subsubsection{{\tt Proof} {\tt .} \dots {\tt Qed} {\tt .}
\comindex{Proof}
\comindex{Qed}
\comindex{Defined}
\comindex{Save}
\comindex{Goal}
\comindex{Admitted}}

A proof starts by the keyword {\tt Proof}.  Then {\Coq} enters the
proof editing mode until the proof is completed. The proof editing
mode essentially contains tactics that are described in chapter
\ref{Tactics}. Besides tactics, there are commands to manage the proof
editing mode. They are described in chapter \ref{Proof-handling}. When
the proof is completed it should be validated and put in the
environment using the keyword {\tt Qed}.
\medskip

\ErrMsg
\begin{enumerate}
\item \errindex{{\ident} already exists}
\end{enumerate}

\begin{Remarks}
\item Several statements can be simultaneously opened.
\item Not only other statements but any vernacular command can be given
within the proof editing mode. In this case, the command is
understood as if it would have been given before the statements still to be
proved. 
\item {\tt Proof} is recommended but can currently be omitted. On the
opposite, {\tt Qed} (or {\tt Defined}, see below) is mandatory to validate a proof.
\item Proofs ended by {\tt Qed} are declared opaque (see \ref{Opaque})
and cannot be unfolded by conversion tactics (see \ref{Conversion-tactics}).
To be able to unfold a proof, you should end the proof by {\tt Defined}
 (see below). 
\end{Remarks}

\begin{Variants}
\item {\tt Proof} {\tt .} \dots {\tt Defined} {\tt .}\\
  Same as {\tt Proof} {\tt .} \dots {\tt Qed} {\tt .} but the proof is
  then declared transparent (see \ref{Transparent}), which means it
  can be unfolded in conversion tactics (see \ref{Conversion-tactics}).
\item {\tt Proof} {\tt .} \dots {\tt Save.}\\
  Same as {\tt Proof} {\tt .} \dots {\tt Qed} {\tt .}
\item {\tt Goal} \type \dots {\tt Save} \ident \\
  Same as {\tt Lemma} \ident {\tt :} \type \dots {\tt Save.}
  This is intended to be used in the interactive mode. Conversely to named
  lemmas, anonymous goals cannot be nested.
\item {\tt Proof.} \dots {\tt Admitted.}\\
  Turns the current conjecture into an axiom and exits editing of
  current proof.
\end{Variants}

% Local Variables: 
% mode: LaTeX
% TeX-master: "Reference-Manual"
% End: 

% $Id$ 
