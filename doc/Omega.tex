\achapter{\texttt{Omega}: a solver of quantifier-free problems in
Presburger Arithmetic}
\aauthor{Pierre Cr�gut}

\asection{Description of {\tt Omega}}
\tacindex{Omega}
\label{description}

{\tt Omega}~solves a goal in Presburger arithmetic, ie a universally
quantified formula made of equations and inequations. Equations may
be specified either on the type \verb=nat= of natural numbers or on
the type \verb=Z= of binary-encoded integer numbers. Formulas on
\verb=nat= are automatically injected into \verb=Z=.  The procedure
may use any hypothesis of the current proof session to solve the goal.

Multiplication is handled by {\tt Omega}~but only goals where at
least one of the two multiplicands of products is a constant are
solvable. This is the restriction meaned by ``Presburger arithmetic''.

If the tactic cannot solve the goal, it fails with an error message.
In any case, the computation eventually stops.

\asubsection{Arithmetical goals recognized by {\tt Omega}}

{\tt Omega}~applied only to quantifier-free formulas built from the
connectors

\begin{quote}
\verb=/\, \/, ~, ->=
\end{quote}

on atomic formulas. Atomic formulas are built from the predicates 

\begin{quote}
\verb!=, le, lt, gt, ge!
\end{quote}

 on \verb=nat= or from the predicates

\begin{quote}
\verb!=, <, <=, >, >=!
\end{quote}

 on \verb=Z=. In expressions of type \verb=nat=, {\tt Omega}~recognizes 

\begin{quote}
\verb!plus, minus, mult, pred, S, O!
\end{quote}

and in expressions of type \verb=Z=, {\tt Omega}~recognizes 

\begin{quote}
\verb!+,_,*, Zs!, and constants.
\end{quote}

All expressions of type \verb=nat= or \verb=Z= not built on these
operators are considered abstractly as if they
were arbitrary variables of type \verb=nat= or \verb=Z=.

\asubsection{Messages from {\tt Omega}}
\label{errors}

When {\tt Omega}~does not solve the goal, one of the following errors
is generated:

\begin{ErrMsgs}
\item \errindex{Omega can't solve this system}\\
  This may happen if your goal is not quantifier-free (if it is
  universally quantified, try {\tt Intros} first; if it contains
  existentials quantifiers too, {\tt Omega}\ is not strong enough to solve your
  goal). This may happen also if your goal contains arithmetical
  operators unknown from {\tt Omega}. Finally, your goal may be really
  wrong !

\item \errindex{Omega: Not a quantifier-free goal}\\
  If your goal is universally quantified, you should first apply {\tt
    Intro} as many time as needed.

\item \errindex{Omega: Unrecognized predicate or connective:{\rm\sl ident}}

\item \errindex{Omega: Unrecognized atomic proposition:{\rm\sl prop}}
  
\item \errindex{Omega: Can't solve a goal with proposition variables}

\item \errindex{Omega: Unrecognized proposition}

\item \errindex{Omega: Can't solve a goal with non-linear products}

\item \errindex{Omega: Can't solve a goal with equality on {\rm\sl type}}

\end{ErrMsgs}

%% Ce code est d�branch� pour l'instant
%% 
% \asubsection{Control over the output}
% There are some flags that can be set to get more information on the procedure

% \begin{itemize}
% \item \verb=Time= to get the time used by the procedure
% \item \verb=System= to visualize the normalized systems.
% \item \verb=Action= to visualize the actions performed by the OMEGA
%      procedure (see \ref{technical}).
% \end{itemize}

% \comindex{Set Omega Time}
% \comindex{UnSet Omega Time}
% \comindex{Switch Omega Time}
% \comindex{Set Omega System}
% \comindex{UnSet Omega System}
% \comindex{Switch Omega System}
% \comindex{Set Omega Action}
% \comindex{UnSet Omega Action}
% \comindex{Switch Omega Action}

 Use {\tt Set Omega {\rm\sl flag}} to set the flag
 {\rm\sl flag}. Use {\tt Unset Omega {\rm\sl flag}} to unset it and
{\tt Switch Omega {\rm\sl flag}} to toggle it.

\section{Using {\tt Omega}}

The tactic {\tt Omega}~does not belong to the core system. It should be
loaded by

\begin{coq_example*}
Require Omega.
\end{coq_example*}

\example{}

\begin{coq_example}
Goal (m,n:Z) ~ `1+2*m = 2*n`.
Intros; Omega.
\end{coq_example}
\begin{coq_eval}
Abort.
\end{coq_eval}

\example{}

\begin{coq_example}
Goal (z:Z)`z>0` -> `2*z + 1 > z`.
Intro; Omega.
\end{coq_example}

Other examples can be found in \verb+$COQLIB/theories/DEMOS/OMEGA+.

\asection{Technical data}
\label{technical}

\asubsection{Overview of the tactic}
\begin{itemize}

\item The goal is negated twice and the first negation is introduced as an
     hypothesis.
\item Hypothesis are decomposed in simple equations or inequations. Multiple
     goals may result from this phase.
\item Equations and inequations over \verb=nat= are translated over
     \verb=Z=, multiple goals may result from the translation of
     substraction.
\item Equations and inequations are normalized.
\item Goals are solved by the {\it OMEGA} decision procedure.
\item The script of the solution is replayed.

\end{itemize}

\asubsection{Overview of the {\it OMEGA} decision procedure}

The {\it OMEGA} decision procedure involved in the {\tt Omega}~tactic uses
a small subset of the decision procedure presented in

\begin{quote}
  "The Omega Test: a fast and practical integer programming
algorithm for dependence analysis", William Pugh, Communication of the
ACM , 1992, p 102-114.
\end{quote}

Here is an overview.
The reader is refered to the original paper for more information.

\begin{itemize}

\item Equations and inequations are normalized by division by the GCD of their
     coefficients.
\item Equations are eliminated, using the Banerjee test to get a coefficient 
     equal to one.
\item Note that each inequation defines a half space in the space of real value
     of the variables.
\item Inequations are solved by projecting on the hyperspace defined by 
     cancelling one of the variable. 
     They are partitioned according to the sign of
     the coefficient of the eliminated variable. Pairs of inequations from
     different classes define a new edge in the projection.
\item Redundant inequations are eliminated or merged in new equations that can
     be eliminated by the Banerjee test.
\item The last two steps are iterated until a contradiction is reached
     (success) or there is no more variable to eliminate (failure).

\end{itemize}

It may happen that there is a real solution and no integer one. The last
steps of the Omega procedure (dark shadow) are not implemented, so the 
decision procedure is only partial.

\asection{Bugs}

\begin{itemize}
\item The simplification procedure is very dumb and this results in
  many redundant cases to explore.

\item Much too slow.

\item Certainely plenty other bugs !! You can report them to

\begin{quote}
  \verb=Pierre.Cregut@cnet.francetelecom.fr= 
\end{quote}

\end{itemize}

%%% Local Variables: 
%%% mode: latex
%%% TeX-master: "Reference-Manual"
%%% End: 
