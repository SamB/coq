\chapter{The {\Coq} library}
\index{Theories}\label{Theories}

The \Coq\ library is structured into three parts:

\begin{description}
\item[The initial library:] it contains
  elementary logical notions and datatypes. It constitutes the
  basic state of the system directly available when running
  \Coq;

\item[The standard library:] general-purpose libraries containing
  various developments of \Coq\ axiomatizations about sets, lists,
  sorting, arithmetic, etc. This library comes with the system and its
  modules are directly accessible through the \verb!Require! command
  (see~\ref{Require});

\item[User contributions:] Other specification and proof developments
  coming from the \Coq\ users' community. These libraries are no
  longer distributed with the system.  They are available by anonymous
  FTP (see section \ref{Contributions}).
\end{description}

This chapter briefly reviews these libraries.

\section{The basic library}
\label{Prelude}

This section lists the basic notions and results which are directly
available in the standard \Coq\ system
\footnote{These constructions are defined in the
{\tt Prelude} module in directory {\tt theories/INIT} at the {\Coq}
root directory; this includes the modules
% {\tt Core} l'inexplicable let
{\tt Logic},
{\tt Datatypes},
{\tt Specif},
{\tt Peano},
and {\tt Wf}
plus the module {\tt Logic\_Type}}.

\subsection{Logic} \label{Logic}

The basic library of {\Coq} comes with the definitions of standard
(intuitionistic) logical connectives (they are defined as inductive
constructions). They are equipped with an appealing syntax enriching the
(subclass {\form}) of the syntactic class {\term}. The syntax
extension
\footnote{This syntax is defined in module {\tt LogicSyntax}}
 is shown on figure \ref{formulas-syntax}.

\begin{figure}
\label{formulas-syntax}
\begin{center}
\begin{tabular}{|lclr|}
\hline
{\form} & ::= & {\tt True} & ({\tt True})\\
  & $|$ & {\tt False} & ({\tt False})\\
  & $|$ & {\verb|~|} {\form} & ({\tt not})\\
  & $|$ & {\form} {\tt /$\backslash$} {\form} & ({\tt and})\\
  & $|$ & {\form} {\tt $\backslash$/} {\form} & ({\tt or})\\
  & $|$ & {\form} {\tt ->} {\form} & (\em{primitive implication})\\
  & $|$ & {\form} {\tt <->} {\form} & ({\tt iff})\\
  & $|$ & {\tt (} {\ident} {\tt :} {\type} {\tt )}
  {\form} & (\em{primitive for all})\\
  & $|$ & {\tt ( ALL} {\ident} \zeroone{{\tt :} {\specif}} {\tt |}
  {\form} {\tt )} & ({\tt all})\\
  & $|$ & {\tt ( EX} {\ident} \zeroone{{\tt :} {\specif}} {\tt
  |} {\form} {\tt )}  & ({\tt ex})\\
  & $|$ & {\tt ( EX} {\ident} \zeroone{{\tt :} {\specif}} {\tt
  |} {\form}  {\tt \&} {\form} {\tt )} & ({\tt ex2})\\
  & $|$ & {\term} {\tt =} {\term} & ({\tt eq})\\
\hline
\end{tabular}
\end{center}

\medskip

\noindent Remark: The implication is not defined but primitive
(it is a non-dependent product of a proposition over another proposition).
There is also a primitive universal quantification (it is a
dependent product over a proposition). The primitive universal
quantification allows both first-order and higher-order
quantification. There is no need to
use the notation {\tt ( ALL} {\ident} \zeroone{{\tt :} {\specif}} {\tt
|} {\form} {\tt )} propositions), except to have a notation dual of
the notation for first-order existential quantification.
\caption{Syntax of formulas}
\end{figure}

\subsubsection{Propositional Connectives} \label{Connectives}
\index{Connectives}

First, we find propositional calculus connectives:
\ttindex{True}
\ttindex{I}
\ttindex{False}
\ttindex{not}
\ttindex{and}
\ttindex{conj}
\ttindex{proj1}
\ttindex{proj2}

\begin{coq_example*}
Inductive True : Prop := I : True. 
Inductive False : Prop := . 
Definition not := [A:Prop] A->False.
Inductive and [A,B:Prop] : Prop := conj : A -> B -> A/\B.
Section Projections.
Variables A,B : Prop.
Theorem proj1 : A/\B -> A.
Theorem proj2 : A/\B -> B.
\end{coq_example*}
\begin{coq_eval}
Abort.
Abort.
\end{coq_eval}
\ttindex{or}
\ttindex{or\_introl}
\ttindex{or\_intror}
\ttindex{iff}
\ttindex{IF}
\begin{coq_example*}
End Projections.
Inductive or [A,B:Prop] : Prop
    := or_introl : A -> A\/B 
     | or_intror : B -> A\/B.
Definition iff := [P,Q:Prop] (P->Q) /\ (Q->P).
Definition IF := [P,Q,R:Prop] (P/\Q) \/ (~P/\R).
\end{coq_example*}

\subsubsection{Quantifiers} \label{Quantifiers}
\index{Quantifiers}

Then we find first-order quantifiers:
\ttindex{all}
\ttindex{All}
\ttindex{ex}
\ttindex{Ex}
\ttindex{EX}
\ttindex{ex\_intro}
\ttindex{ex2}
\ttindex{Ex2}
\ttindex{ex\_intro2}

\begin{coq_example*}
Definition all := [A:Set][P:A->Prop](x:A)(P x). 
Inductive ex [A:Set;P:A->Prop] : Prop 
    := ex_intro : (x:A)(P x)->(ex A P).
Inductive ex2 [A:Set;P,Q:A->Prop] : Prop
    := ex_intro2 : (x:A)(P x)->(Q x)->(ex2 A P Q).
\end{coq_example*}

The following abbreviations are allowed:
\begin{center}
  \begin{tabular}[h]{|l|r|}
    \hline
    \verb+(ALL x:A | P)+    & \verb+(all A [x:A]P)+ \\
    \verb+(ALL x | P)+      & \verb+(all A [x:A]P)+ \\
    \verb+(EX x:A | P)+     & \verb+(ex A [x:A]P)+ \\
    \verb+(EX x | P)+       & \verb+(ex A [x:A]P)+ \\
    \verb+(EX x:A | P & Q)+ & \verb+(ex2 A [x:A]P [x:A]Q)+ \\
    \verb+(EX x | P & Q)+   & \verb+(ex2 A [x:A]P [x:A]Q)+ \\
    \hline
  \end{tabular}
\end{center}

The type annotation \texttt{:A} can be omitted when \texttt{A} can be
synthesized by the system.

\subsubsection{Equality} \label{Equality}
\index{Equality}

Then, we find equality, defined as an inductive relation. That is,
given a \verb:Set: \verb:A: and an \verb:x: of type \verb:A:, the
predicate \verb:(eq A x): is the smallest which contains \verb:x:.
This definition, due to Christine Paulin-Mohring, is equivalent to
define \verb:eq: as the smallest reflexive relation, and it is also
equivalent to Leibniz' equality.

\ttindex{eq}
\ttindex{refl\_equal}

\begin{coq_example*}
Inductive eq [A:Set;x:A] : A->Prop
    := refl_equal : (eq A x x).
\end{coq_example*}

\subsubsection{Lemmas} \label{PreludeLemmas}
Finally, a few easy lemmas are provided.

\ttindex{absurd}

\begin{coq_example*}
Theorem absurd : (A:Prop)(C:Prop) A -> ~A -> C.
\end{coq_example*}
\begin{coq_eval}
Abort.
\end{coq_eval}
\ttindex{sym\_equal}
\ttindex{trans\_equal}
\ttindex{f\_equal}
\ttindex{sym\_not\_equal}
\begin{coq_example*}
Section equality.
 Variable A,B : Set.
 Variable f   : A->B.
 Variable x,y,z : A.
 Theorem sym_equal : x=y -> y=x.
 Theorem trans_equal : x=y -> y=z -> x=z.
 Theorem f_equal : x=y -> (f x)=(f y).
 Theorem sym_not_equal : ~(x=y) -> ~(y=x).
\end{coq_example*}
\begin{coq_eval}
Abort.
Abort.
Abort.
Abort.
\end{coq_eval}
\ttindex{eq\_ind\_r}
\ttindex{eq\_rec\_r}
\begin{coq_example*}
End equality.
Definition eq_ind_r : (A:Set)(x:A)(P:A->Prop)(P x)->(y:A)y=x->(P y).
Definition eq_rec_r : (A:Set)(x:A)(P:A->Set)(P x)->(y:A)y=x->(P y).
\end{coq_example*}
\begin{coq_eval}
Abort.
Abort.
\end{coq_eval}
\begin{coq_example*}
Immediate sym_equal sym_not_equal.
\end{coq_example*}

\subsection{Datatypes} \label{Datatypes}

In the basic library, we find the definition\footnote{They are in {\tt
Datatypes.v}} 
of the basic data-types of programming, 
again defined as inductive constructions over the sort
\verb:Set:. Some of them come with a special syntax shown on figure
\ref{specif-syntax}.

\subsubsection{Programming} \label{Programming}
\index{Programming}
\index{Datatypes}

\ttindex{unit}
\ttindex{tt}
\ttindex{bool}
\ttindex{true}
\ttindex{false}
\ttindex{nat}
\ttindex{O}
\ttindex{S}

\begin{coq_example*}
Inductive unit : Set := tt : unit.
Inductive bool : Set := true : bool 
                      | false : bool.
Inductive nat : Set := O : nat 
                     | S : nat->nat.
\end{coq_example*}

Note that zero is the letter \verb:O:, and {\sl not} the numeral
\verb:0:.

We then define the disjoint sum of \verb:A+B: of two sets \verb:A: and
\verb:B:, and their product \verb:A*B:.
\ttindex{sum}
\ttindex{A+B}
\ttindex{+}
\ttindex{inl}
\ttindex{inr}
\ttindex{prod}
\ttindex{A*B}
\ttindex{*}
\ttindex{pair}
\ttindex{fst}
\ttindex{snd}
\ttindex{Fst}
\ttindex{Snd}

\begin{coq_example*}
Inductive sum [A,B:Set] : Set
    := inl : A -> A+B 
     | inr : B -> A+B.
Inductive prod [A,B:Set] : Set := pair : A -> B -> A*B.
Section projections.
   Variables A,B:Set.
   Definition fst := [H:A*B] Cases H of (x,y) => x end.
   Definition snd := [H:A*B] Cases H of (x,y) => y end.
End projections.
Syntactic Definition Fst := (fst ? ?).
Syntactic Definition Snd := (snd ? ?).
\end{coq_example*}

\subsection{Specification}

The following notions\footnote{They are defined in module {\tt
Specif.v}} allows to build new datatypes and specifications. 
They are available with the syntax shown on
figure \ref{specif-syntax}\footnote{This syntax can be found in the module
{\tt SpecifSyntax.v}}.

For instance, given \verb|A:Set| and \verb|P:A->Prop|, the construct
\verb+{x:A | (P x)}+ (in abstract syntax \verb+(sig A P)+) is a
\verb:Set:. We may build elements of this set as \verb:(exist x p):
whenever we have a witness \verb|x:A| with its justification
\verb|p:(P x)|.

From such a \verb:(exist x p): we may in turn extract its witness
\verb|x:A| (using an elimination construct such as \verb:Cases:) but
{\sl not} its justification, which stays hidden, like in an abstract
data type. In technical terms, one says that \verb:sig: is a ``weak
(dependent) sum''.  A variant \verb:sig2: with two predicates is also
provided.

\index{\{x:A "| (P x)\}}
\index{"|}
\ttindex{sig}
\ttindex{exist}
\ttindex{sig2}
\ttindex{exist2}

\begin{coq_example*}
Inductive  sig [A:Set;P:A->Prop] : Set
    := exist : (x:A)(P x) -> (sig A P).
Inductive sig2 [A:Set;P,Q:A->Prop] : Set
    := exist2 : (x:A)(P x) -> (Q x) -> (sig2 A P Q).
\end{coq_example*}

A ``strong (dependent) sum'' \verb+{x:A & (P x)}+ may be also defined,
when the predicate \verb:P: is now defined as a \verb:Set:
constructor.

\ttindex{\{x:A \& (P x)\}}
\ttindex{\&}
\ttindex{sigS}
\ttindex{existS}
\ttindex{projS1}
\ttindex{projS2}
\ttindex{sigS2}
\ttindex{existS2}

\begin{coq_example*}
Inductive sigS [A:Set;P:A->Set] : Set
    := existS : (x:A)(P x) -> (sigS A P).
Section projections.
   Variable A:Set.
   Variable P:A->Set.
   Definition projS1 := [H:(sigS A P)] let (x,h) = H in x.
   Definition projS2 := [H:(sigS A P)]<[H:(sigS A P)](P (projS1 H))>
                            let (x,h) = H in h.
End projections. 
Inductive sigS2 [A:Set;P,Q:A->Set] : Set
    := existS2 : (x:A)(P x) -> (Q x) -> (sigS2 A P Q).
\end{coq_example*}

A related non-dependent construct is the constructive sum
\verb"{A}+{B}" of two propositions \verb:A: and \verb:B:.
\label{sumbool}
\ttindex{sumbool}
\ttindex{left}
\ttindex{right}
\ttindex{\{A\}+\{B\}}

\begin{coq_example*}
Inductive sumbool [A,B:Prop] : Set
    := left  : A -> ({A}+{B}) 
     | right : B -> ({A}+{B}).
\end{coq_example*}

This \verb"sumbool" construct may be used as a kind of indexed boolean
data type. An intermediate between \verb"sumbool" and \verb"sum" is
the mixed \verb"sumor" which combines \verb"A:Set" and \verb"B:Prop"
in the \verb"Set" \verb"A+{B}".
\ttindex{sumor}
\ttindex{inleft}
\ttindex{inright}
\ttindex{A+\{B\}}

\begin{coq_example*}
Inductive sumor [A:Set;B:Prop] : Set
    := inleft  : A -> (A+{B}) 
     | inright : B -> (A+{B}).
\end{coq_example*}

\begin{figure}
\label{specif-syntax}
\begin{center}
\begin{tabular}{|lclr|}
\hline
{\specif} & ::= & {\specif} {\tt *} {\specif} & ({\tt prod})\\
  & $|$ & {\specif} {\tt +} {\specif} & ({\tt sum})\\
  & $|$ & {\specif} {\tt + \{} {\specif} {\tt \}} & ({\tt sumor})\\
  & $|$ & {\tt \{} {\specif} {\tt \} + \{} {\specif} {\tt \}} &
  ({\tt sumbool})\\  
  & $|$ & {\tt \{} {\ident} {\tt :} {\specif} {\tt |} {\form} {\tt \}}
  & ({\tt sig})\\
  & $|$ & {\tt \{} {\ident} {\tt :} {\specif} {\tt |} {\form}  {\tt \&}
  {\form} {\tt \}} & ({\tt sig2})\\
  & $|$ & {\tt \{} {\ident} {\tt :} {\specif} {\tt \&} {\specif} {\tt
    \}} & ({\tt sigS})\\
  & $|$ & {\tt \{} {\ident} {\tt :} {\specif} {\tt \&} {\specif} {\tt
    \&} {\specif} {\tt \}} & ({\tt sigS2})\\
  &  & & \\
{\term} & ::= & {\tt (} {\term} {\tt ,} {\term} {\tt )} & ({\tt pair})\\
\hline
\end{tabular}
\caption{Syntax of datatypes and specifications}
\end{center}
\end{figure}

We may define variants of the axiom of choice, like in Martin-L�f's
Intuitionistic Type Theory.
\ttindex{Choice}
\ttindex{Choice2}
\ttindex{bool\_choice}

\begin{coq_example*}
Lemma Choice : (S,S':Set)(R:S->S'->Prop)((x:S){y:S'|(R x y)})
                       -> {f:S->S'|(z:S)(R z (f z))}.
Lemma Choice2 : (S,S':Set)(R:S->S'->Set)((x:S){y:S' & (R x y)})
                       -> {f:S->S' & (z:S)(R z (f z))}.
Lemma bool_choice : (S:Set)(R1,R2:S->Prop)((x:S){(R1 x)}+{(R2 x)}) ->
 {f:S->bool | (x:S)( ((f x)=true  /\ (R1 x)) 
                  \/ ((f x)=false /\ (R2 x)))}.
\end{coq_example*}
\begin{coq_eval}
Abort.
Abort.
Abort.
\end{coq_eval}

The next construct builds a sum between a data type \verb|A:Set| and
an exceptional value encoding errors:

\ttindex{Exc}
\ttindex{value}
\ttindex{error}

\begin{coq_example*}
Inductive Exc [A:Set] : Set := value : A->(Exc A) 
                             | error : (Exc A).
\end{coq_example*}


This module ends with one axiom and theorems, 
relating the sorts \verb:Set: and
\verb:Prop: in a way which is consistent with the realizability
interpretation.
\ttindex{False\_rec}
\ttindex{eq\_rec}
\ttindex{Except}
\ttindex{absurd\_set}
\ttindex{and\_rec}

\begin{coq_example*}
Axiom False_rec : (P:Set)False->P.
Definition except := False_rec.
Syntactic Definition Except := (except ?).
Theorem absurd_set : (A:Prop)(C:Set)A->(~A)->C.
Theorem and_rec : (A,B:Prop)(C:Set)(A->B->C)->(A/\B)->C.
\end{coq_example*}
\begin{coq_eval}
Abort.
Abort.
\end{coq_eval}

\subsection{Basic Arithmetics}

The basic library includes a few elementary properties of natural numbers,
together with the definitions of predecessor, addition and
multiplication\footnote{This is in module {\tt Peano.v}}.
\ttindex{eq\_S}
\ttindex{pred}
\ttindex{pred\_Sn}
\ttindex{eq\_add\_S}
\ttindex{not\_eq\_S}
\ttindex{IsSucc}
\ttindex{O\_S}
\ttindex{n\_Sn}
\ttindex{plus}
\ttindex{plus\_n\_O}
\ttindex{plus\_n\_Sm}
\ttindex{mult}
\ttindex{mult\_n\_O}
\ttindex{mult\_n\_Sm}

\begin{coq_example*}
Theorem eq_S : (n,m:nat) n=m -> (S n)=(S m).
\end{coq_example*}
\begin{coq_eval}
Abort.
\end{coq_eval}
\begin{coq_example*}
Definition pred : nat->nat 
     := [n:nat](<nat>Cases n of O => O 
                          | (S u) => u end).
Theorem pred_Sn : (m:nat) m=(pred (S m)).
Theorem eq_add_S : (n,m:nat) (S n)=(S m) -> n=m.
Immediate eq_add_S.
Theorem not_eq_S : (n,m:nat) ~(n=m) -> ~((S n)=(S m)).
\end{coq_example*}
\begin{coq_eval}
Abort.
Abort.
Abort.
\end{coq_eval}
\begin{coq_example*}
Definition IsSucc : nat->Prop
  := [n:nat](<Prop>Cases n of O => False
                        | (S p) => True end).
Theorem O_S : (n:nat) ~(O=(S n)).
Theorem n_Sn : (n:nat) ~(n=(S n)).
\end{coq_example*}
\begin{coq_eval}
Abort.
Abort.
\end{coq_eval}
\begin{coq_example*}
Fixpoint plus [n:nat] : nat -> nat := 
   [m:nat](<nat>Cases n of 
       O => m 
    | (S p) => (S (plus p m)) end).
Lemma plus_n_O : (n:nat) n=(plus n O).
Lemma plus_n_Sm : (n,m:nat) (S (plus n m))=(plus n (S m)).
\end{coq_example*}
\begin{coq_eval}
Abort.
Abort.
\end{coq_eval}
\begin{coq_example*}
Fixpoint mult [n:nat] : nat -> nat := 
   [m:nat](<nat> Cases n of O => O 
                         | (S p) => (plus m (mult p m)) end).
Lemma mult_n_O : (n:nat) O=(mult n O).
Lemma mult_n_Sm : (n,m:nat) (plus (mult n m) n)=(mult n (S m)).
\end{coq_example*}
\begin{coq_eval}
Abort.
Abort.
\end{coq_eval}

Finally, it gives the definition of the usual orderings \verb:le:,
\verb:lt:, \verb:ge:, and \verb:gt:.
\ttindex{le}
\ttindex{le\_n}
\ttindex{le\_S}
\ttindex{lt}
\ttindex{ge}
\ttindex{gt}

\begin{coq_example*}
Inductive le [n:nat] : nat -> Prop
    := le_n : (le n n)
     | le_S : (m:nat)(le n m)->(le n (S m)).
Definition lt := [n,m:nat](le (S n) m).
Definition ge := [n,m:nat](le m n).
Definition gt := [n,m:nat](lt m n).
\end{coq_example*}

Properties of these relations are not initially known, but may be
required by the user from modules \verb:Le: and \verb:Lt:.  Finally,
\verb:Peano: gives some lemmas allowing pattern-matching, and a double
induction principle.

\ttindex{nat\_case}
\ttindex{nat\_double\_ind}

\begin{coq_example*}
Theorem nat_case : (n:nat)(P:nat->Prop)(P O)->((m:nat)(P (S m)))->(P n).
\end{coq_example*}
\begin{coq_eval}
Abort.
\end{coq_eval}
\begin{coq_example*}
Theorem nat_double_ind : (R:nat->nat->Prop)
     ((n:nat)(R O n)) -> ((n:nat)(R (S n) O))
     -> ((n,m:nat)(R n m)->(R (S n) (S m)))
     -> (n,m:nat)(R n m).
\end{coq_example*}
\begin{coq_eval}
Abort.
\end{coq_eval}

\subsection{Well-founded recursion}

The basic library contains the basics of well-founded recursion and 
well-founded induction\footnote{This is defined in module {\tt Wf.v}}.
\index{Well foundedness}
\index{Recursion}
\index{Well founded induction}
\ttindex{Acc}
\ttindex{Acc\_inv}
\ttindex{Acc\_rec}
\ttindex{well\_founded}

\begin{coq_example*}
Chapter Well_founded.
Variable A : Set.
Variable R : A -> A -> Prop.
Inductive Acc : A -> Prop 
   := Acc_intro : (x:A)((y:A)(R y x)->(Acc y))->(Acc x).
Lemma Acc_inv : (x:A)(Acc x) -> (y:A)(R y x) -> (Acc y).
\end{coq_example*}
\begin{coq_eval}
Destruct 1; Trivial.
Defined.
\end{coq_eval}
\begin{coq_example*}
Section AccRec.
Variable P : A -> Set.
Variable F : (x:A)((y:A)(R y x)->(Acc y))->((y:A)(R y x)->(P y))->(P x).
Fixpoint Acc_rec [x:A;a:(Acc x)] : (P x)
   := (F x (Acc_inv x a) [y:A][h:(R y x)](Acc_rec y (Acc_inv x a y h))).
End AccRec.
Definition well_founded := (a:A)(Acc a).
Theorem well_founded_induction : 
    well_founded ->
        (P:A->Set)((x:A)((y:A)(R y x)->(P y))->(P x))->(a:A)(P a).
\end{coq_example*}
\begin{coq_eval}
Abort.
\end{coq_eval}
\begin{coq_example*}
End Well_founded. 
\end{coq_example*}
\begin{coq_example*}
  Section Wf_inductor.
Variable A:Set.
Variable R:A->A->Prop.
Theorem well_founded_ind : 
    (well_founded A R) ->
        (P:A->Prop)((x:A)((y:A)(R y x)->(P y))->(P x))->(a:A)(P a).
\end{coq_example*}
\begin{coq_eval}
Abort.
\end{coq_eval}
\begin{coq_example*}
End Wf_inductor.
\end{coq_example*}


\subsection{Accessing the {\Type} level}

The basic library includes the definitions\footnote{This is in module
{\tt Logic\_Type.v}} of logical quantifiers axiomatized at the
\verb:Type: level. 

\ttindex{allT}
\ttindex{AllT}
\ttindex{inst}
\ttindex{gen}
\ttindex{exT}
\ttindex{ExT}
\ttindex{EXT}
\ttindex{exT\_intro}
\ttindex{ExT2}
\ttindex{exT2}
\ttindex{EmptyT}
\ttindex{UnitT}
\ttindex{notT}

\begin{coq_example*}
Definition allT := [A:Type][P:A->Prop](x:A)(P x). 

Section universal_quantification.
Variable A : Type.
Variable P : A->Prop.
Theorem inst :  (x:A)(ALLT x | (P x))->(P x).
Theorem gen : (B:Prop)(f:(y:A)B->(P y))B->(allT ? P).
\end{coq_example*}
\begin{coq_eval}
Abort.
Abort.
\end{coq_eval}
\begin{coq_example*}
End universal_quantification.
Inductive  exT [A:Type;P:A->Prop] : Prop
    := exT_intro : (x:A)(P x)->(exT A P).

Inductive exT2 [A:Type;P,Q:A->Prop] : Prop
    := exT_intro2 : (x:A)(P x)->(Q x)->(exT2 A P Q).
\end{coq_example*}

It defines also Leibniz equality \verb:x==y: when \verb:x: and
\verb:y: belong to \verb+A:Type+.
\ttindex{eqT}
\ttindex{refl\_eqT}
\ttindex{sum\_eqT}
\ttindex{sym\_not\_eqT}
\ttindex{trans\_eqT}
\ttindex{congr\_eqT}
\ttindex{eqT\_ind\_r}
\ttindex{eqT\_rec\_r}

\begin{coq_example*}
Inductive eqT [A:Type;x:A] : A -> Prop
                       := refl_eqT : (eqT A x x).
Section Equality_is_a_congruence.
Variables A,B : Type.
Variable  f : A->B.
 Variable x,y,z : A.
 Lemma sym_eqT : (x==y) -> (y==x).
 Lemma trans_eqT : (x==y) -> (y==z) -> (x==z).
 Lemma congr_eqT : (x==y)->((f x)==(f y)).
\end{coq_example*}
\begin{coq_eval}
Abort.
Abort.
Abort.
\end{coq_eval}
\begin{coq_example*}
End Equality_is_a_congruence.
Immediate sym_eqT sym_not_eqT.
Definition eqT_ind_r: (A:Type)(x:A)(P:A->Prop)(P x)->(y:A)y==x -> (P y).
\end{coq_example*}
\begin{coq_eval}
Abort.
\end{coq_eval}

The figure \ref{formulas-syntax-type} presents the syntactic notations 
corresponding to the main definitions
\footnote{This syntax is defined in module {\tt Logic\_TypeSyntax}}

\begin{figure}
\label{formulas-syntax-type}
\begin{center}
\begin{tabular}{|lclr|}
\hline
{\form} & ::= & {\tt ( ALLT} {\ident} \zeroone{{\tt :} {\specif}} {\tt |}
  {\form} {\tt )} & ({\tt allT})\\
  & $|$ & {\tt ( EXT} {\ident} \zeroone{{\tt :} {\specif}} {\tt
  |} {\form} {\tt )}  & ({\tt exT})\\
  & $|$ & {\tt ( EXT} {\ident} \zeroone{{\tt :} {\specif}} {\tt
  |} {\form}  {\tt \&} {\form} {\tt )} & ({\tt exT2})\\
  & $|$ & {\term} {\tt ==} {\term} & ({\tt eqT})\\
\hline
\end{tabular}
\end{center}
\caption{Syntax of first-order formulas in the type universe}
\end{figure}

At the end, it defines datatypes at the {\Type} level.

\begin{coq_example*}
Inductive EmptyT: Type :=.
Inductive UnitT : Type := IT : UnitT.
Definition notT := [A:Type] A->EmptyT.

Inductive identityT [A:Type; a:A] : A->Type :=
     refl_identityT : (identityT A a a).
\end{coq_example*}


\section{The standard library}

\subsection{Survey}

The rest of the standard library is structured into the following 
subdirectories:

\begin{tabular}{lp{12cm}}
  {\bf LOGIC}   & Classical logic and dependent equality \\
  {\bf ARITH}   & Basic Peano arithmetic \\
  {\bf ZARITH}  & Basic integer arithmetic \\
  {\bf BOOL}    &  Booleans (basic functions and results) \\
  {\bf LISTS}   & Monomorphic and polymorphic lists (basic functions and
            results), Streams (infinite sequences defined with co-inductive
            types) \\
  {\bf SETS}    & Sets (classical, constructive, finite, infinite, power set,
            etc.) \\
 {\bf RELATIONS} & Relations (definitions and basic results). There is
             a subdirectory about well-founded relations ({\bf WELLFOUNDED}) \\
 {\bf SORTING} & Various sorting algorithms \\
 {\bf REALS}   & Axiomatization of Real Numbers (classical, basic functions 
                 and results, integer part and fractional part,
                 requires the \textbf{ZARITH} library)
\end{tabular}
\medskip

These directories belong to the initial load path of the system, and
the modules they provide are compiled at installation time. So they
are directly accessible with the command \verb!Require! (see
chapter~\ref{Other-commands}). 

The different modules of the \Coq\ standard library are described in the
additional document \verb!Library.dvi!. They are also accessible on the WWW
through the \Coq\ homepage
\footnote{\texttt{http://coq.inria.fr}}.

\subsection{Notations for integer arithmetics}
\index{Arithmetical notations}

On figure \ref{zarith-syntax} is described the syntax of expressions
for integer arithmetics. It is provided by requiring the module {\tt ZArith}.

\ttindex{+}
\ttindex{*}
\ttindex{-}

The {\tt +} and {\tt -} binary operators bind less than the {\tt *} operator
which binds less than the {\tt |}~...~{\tt |} and {\tt -} unary
operators which bind less than the others constructions.
All the binary operators are left associative. The {\tt [}~...~{\tt ]}
allows to escape the {\zarith} grammar.

\begin{figure}
\begin{center}
\begin{tabular}{|lcl|}
\hline
{\form} & ::= & {\tt `} {\zarithformula} {\tt `}\\
  & & \\
{\term} & ::= & {\tt `} {\zarith} {\tt `}\\
  & & \\
{\zarithformula} & ::= & {\zarith} {\tt =} {\zarith} \\
   & $|$ & {\zarith} {\tt <=} {\zarith} \\
   & $|$ & {\zarith} {\tt <} {\zarith} \\
   & $|$ & {\zarith} {\tt >=} {\zarith} \\
   & $|$ & {\zarith} {\tt >} {\zarith}  \\
   & $|$ & {\zarith} {\tt =} {\zarith} {\tt =} {\zarith} \\
   & $|$ & {\zarith} {\tt <=} {\zarith} {\tt <=} {\zarith} \\
   & $|$ & {\zarith} {\tt <=} {\zarith} {\tt <} {\zarith} \\
   & $|$ & {\zarith} {\tt <} {\zarith} {\tt <=} {\zarith} \\
   & $|$ & {\zarith} {\tt <} {\zarith} {\tt <} {\zarith} \\
   & $|$ & {\zarith} {\tt <>} {\zarith} \\
   & $|$ & {\zarith} {\tt ?} {\tt =} {\zarith} \\
  & & \\
{\zarith}  & ::= & {\zarith} {\tt +} {\zarith} \\
   & $|$ & {\zarith} {\tt -} {\zarith} \\
   & $|$ & {\zarith} {\tt *} {\zarith} \\
   & $|$ & {\tt |} {\zarith} {\tt |} \\
   & $|$ & {\tt -} {\zarith} \\
   & $|$ & {\ident} \\
   & $|$ & {\tt [} {\term} {\tt ]} \\
   & $|$ & {\tt (} \nelist{\zarith}{} {\tt )} \\
   & $|$ & {\tt (} \nelist{\zarith}{,} {\tt )} \\
   & $|$ & {\integer} \\
\hline
\end{tabular}
\end{center}
\label{zarith-syntax}
\caption{Syntax of expressions in integer arithmetics}
\end{figure}

\subsection{Notations for Peano's arithmetic (\texttt{nat})}
\index{Peano's arithmetic notations}

After having required the module \texttt{Arith}, the user can type the
naturals using decimal notation. That is he can write \texttt{(3)}
for \texttt{(S (S (S O)))}. The number must be between parentheses.
This works also in the left hand side of a \texttt{Cases} expression
(see for example section \ref{Refine-example}).

\section{Users' contributions}
\index{Contributions}
\label{Contributions}

Numerous users' contributions have been collected and are available on
the WWW at the following address: \verb!pauillac.inria.fr/coq/contribs!.
On this web page, you have a list of all contributions with
informations (author, institution, quick description, etc.) and the
possibility to download them one by one.
There is a small search engine to look for keywords in all contributions.
You will also find informations on how to submit a new contribution. 

The users' contributions may also be obtained by anonymous FTP from site
\verb:ftp.inria.fr:, in directory \verb:INRIA/coq/: and
searchable on-line at 

\begin{quotation}
  \texttt{http://coq.inria.fr/contribs-eng.html}
\end{quotation}

% $Id$ 


%%% Local Variables: 
%%% mode: latex
%%% TeX-master: t
%%% End: 
