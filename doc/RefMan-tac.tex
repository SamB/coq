\chapter{Tactics}
\index{Tactics}
\label{Tactics}

A deduction rule is a link between some (unique) formula, that we call
the {\em conclusion} and (several) formul{\ae} that we call the {\em
premises}. Indeed, a deduction rule can be read in two ways. The first
one has the shape: {\it ``if I know this and this then I can deduce
this''}. For instance, if I have a proof of $A$ and a proof of $B$
then I have a proof of $A \wedge B$. This is forward reasoning from
premises to conclusion. The other way says: {\it ``to prove this I
have to prove this and this''}. For instance, to prove $A \wedge B$, I
have to prove $A$ and I have to prove $B$. This is backward reasoning
which proceeds from conclusion to premises. We say that the conclusion
is {\em the goal}\index{goal} to prove and premises are {\em the
subgoals}\index{subgoal}.  The tactics implement {\em backward
reasoning}. When applied to a goal, a tactic replaces this goal with
the subgoals it generates. We say that a tactic reduces a goal to its
subgoal(s).

Each (sub)goal is denoted with a number. The current goal is numbered
1. By default, a tactic is applied to the current goal, but one can
address a particular goal in the list by writing {\sl n:\tac} which
means {\it ``apply tactic {\tac} to goal number {\sl n}''}.
We can show the list of subgoals by typing {\tt Show} (see section
\ref{Show}).

Since not every rule applies to a given statement, every tactic cannot be
used to reduce any goal. In other words, before applying a tactic to a
given goal, the system checks that some {\em preconditions} are
satisfied. If it is not the case, the tactic raises an error message.

Tactics are build from tacticals and atomic tactics.  There are, at
least, three levels of atomic tactics. The simplest one implements
basic rules of the logical framework. The second level is the one of
{\em derived rules} which are built by combination of other
tactics. The third one implements heuristics or decision procedures to
build a complete proof of a goal.  
%Here is a table of all existing atomic tactics in \Coq:
%\index{atomic tactics}
%\label{atomic-tactics-table}

\section{Syntax of tactics and tacticals}
\label{tactic-syntax}
\index{tactic@{\tac}}

A tactic is
applied as an ordinary command. If the tactic does not
address the first subgoal, the command may be preceded by the
wished subgoal number. See figure~\ref{InvokeTactic} for the syntax of
tactic invocation and tacticals.

\medskip

\begin{figure}
\begin{center}
\begin{tabular}{|lcl|}
\hline
{\tac} & ::= & \atomictac\\
  & $|$ & {\tt (} {\tac} {\tt )} \\
  & $|$ & {\tac} {\tt Orelse} {\tac}\\
  & $|$ & {\tt Repeat} \tac \\
  & $|$ & {\tt Do} {\num} {\tac} \\
  & $|$ & {\tt Info} \tac \\
  & $|$ & {\tt Try} \tac \\
  & $|$ & {\tt First [} {\tac}{\tt\ | \dots\ | }{\tac} {\tt ]} \\
  & $|$ & {\tt Solve [} {\tac}{\tt\ | \dots\ | }{\tac} {\tt ]} \\
  & $|$ & {\tt Abstract} {\tac} \\
  & $|$ & {\tt Abstract} {\tac} {\tt using} {\ident}\\
  & $|$ & {\tac} {\tt ;} {\tac}\\
  & $|$ & {\tac} {\tt ;[} {\tac} \tt \verb=|=
          \dots\ \verb=|= {\tac} {\tt ]} \\
{\commandtac} & ::= & {\num} {\tt :} {\tac} {\tt .}\\
 & $|$ & {\tac} {\tt .}\\
\hline
\end{tabular}
\end{center}
\caption{Invocation of tactics and tacticals}\label{InvokeTactic}
\end{figure}

\begin{Remarks}
\item The infix tacticals {\tt Orelse} and ``\dots\ {\tt ;} \dots'' are
associative. 
The tactical {\tt Orelse} binds more than the prefix tacticals
{\tt Try}, {\tt Repeat}, {\tt Do}, {\tt Info} and {\tt Abstract} which
themselves bind more than  
the postfix tactical ``{\tt \dots\ ;[ \dots\ ]}'' which 
binds more than ``\dots\ {\tt ;} \dots''.

\noindent For instance

\noindent {\tt Try Repeat \tac$_1$ Orelse
  \tac$_2$;\tac$_3$;[\tac$_{31}$|\dots|\tac$_{3n}$];\tac$_4$.}

\noindent is understood as 

\noindent {\tt (Try (Repeat (\tac$_1$ Orelse \tac$_2$)));
  ((\tac$_3$;[\tac$_{31}$|\dots|\tac$_{3n}$]);\tac$_4$)}.

\item An {\atomictac} is any of the tactics listed below.
\end{Remarks}

\section{Explicit proof as a term}

\subsection{\tt Exact \term}
\tacindex{Exact}
\label{Exact}
This tactic applies to any goal. It gives directly the exact proof
term of the goal. Let {\T} be our goal, let {\tt p} be a term of type
{\tt U} then {\tt Exact p} succeeds iff {\tt T} and {\tt U} are
convertible (see section \ref{conv-rules}).

\begin{ErrMsgs}
\item \errindex{Not an exact proof}
\end{ErrMsgs}


\subsection{\tt Refine \term}
\tacindex{Refine}
\label{Refine}
\index{?@{\texttt{?}}}

This tactic allows to give an exact proof but still with some
holes. The holes are noted ``\texttt{?}''.

\begin{ErrMsgs}
\item \errindex{invalid argument}: 
  the tactic \texttt{Refine} doesn't know what to do
  with the term you gave.
\item \texttt{Refine passed ill-formed term}: the term you gave is not
  a valid proof (not easy to debug in general).
  This message may also occur in higher-level tactics, which call 
  \texttt{Refine} internally.
\item \errindex{There is an unknown subterm I cannot solve}: 
  there is a hole in the term you gave
  which type cannot be inferred. Put a cast around it.
\end{ErrMsgs}

This tactic is currently given as an experiment. An example of use is given
in section \ref{Refine-example}.

\section{Basics}
\index{Typing rules}
Tactics presented in this section implement the basic typing rules of
{\sc Cic} given in chapter \ref{Cic}.

\subsection{{\tt Assumption}}
\tacindex{Assumption}
This tactic applies to any goal. It implements the
``Var''\index{Typing rules!Var} rule given in section
\ref{Typed-terms}. It looks in the local context for an hypothesis
which type is equal to the goal.  If it is the case, the subgoal is
proved. Otherwise, it fails.

\begin{ErrMsgs}
\item  \errindex{No such assumption}
\end{ErrMsgs}

\subsection{\tt Clear {\ident}.}\tacindex{Clear}\label{Clear}
This tactic erases the hypothesis named {\ident} in the local context
of the current goal. Then {\ident} is no more displayed and no more
usable in the proof development.

\begin{ErrMsgs}
\item \errindex{No such assumption}
\end{ErrMsgs}

\subsection{\tt Move {\ident$_1$} after {\ident$_2$}.}\tacindex{Move}
This moves the hypothesis named {\ident$_1$} in the local context
after the hypothesis named {\ident$_2$}.

If {\ident$_1$} comes before {\ident$_2$} in the order of dependences,
then all hypotheses between {\ident$_1$} and {\ident$_2$} which
(possibly indirectly) depend on {\ident$_1$} are moved also.

If {\ident$_1$} comes after {\ident$_2$} in the order of dependences,
then all hypotheses between {\ident$_1$} and {\ident$_2$} which 
(possibly indirectly) occur in {\ident$_1$} are moved also.

\begin{ErrMsgs}
\item \errindex{No such assumption: {\ident$_i$}}

\item \errindex{Cannot move {\ident$_1$} after {\ident$_2$}:
                   it occurs in {\ident$_2$}}
\item \errindex{Cannot move {\ident$_1$} after {\ident$_2$}:
                   it depends on {\ident$_2$}}
\end{ErrMsgs}

\subsection{\tt Intro}
\tacindex{Intro}
\label{Intro}
This tactic applies to a goal which is a product. It implements the
``Lam''\index{Typing rules!Lam} rule given in section
\ref{Typed-terms}.  Actually, only one subgoal will be generated since the
other one can be automatically checked.

If the current goal is a dependent product {\tt (x:T)U} and
{\tt x} is a name that does not exist in the current context, then
{\tt Intro} puts {\tt x:T} in the local context. Otherwise, it puts
{\tt x}{\it n}{\tt :T} where {\it n} is such that {\tt x}{\it n} is a
fresh name. The new subgoal is {\tt U}. If the {\tt x} has been
renamed {\tt x}{\it n} then it is replaced by {\tt x}{\it n} in {\tt
U}.

If the goal is a non dependent product T -> U, then it
puts in the local context either {\tt H}{\it n}{\tt :T} 
(if {\tt T} is {\tt Set} or {\tt Prop}) or {\tt X}{\it n}{\tt :T} (if
the type of {\tt T} is {\tt Type}) or {\tt l}{\it n}{\tt :T} with {\it l}
the first letter of the type of x. {\it n} is such that {\tt H}{\it
  n} or {\tt X}{\it n} {\tt l}{\it n} or are fresh identifiers.
In both cases the new subgoal is {\tt U}. 

If the goal is not a product, the tactic {\tt Intro} applies the tactic
{\tt Red} until the tactic {\tt Intro} can be applied or the goal is not
reducible.

\begin{ErrMsgs}
\item \errindex{No product even after head-reduction}
\end{ErrMsgs}

\Warning: \texttt{{\ident}$_1$ is already used; changed to {\ident}$_2$}

\begin{Variants}
\item {\tt Intros}\tacindex{Intros}\\
  Repeats {\tt Intro} until it meets the head-constant. It never reduces
  head-constants and it never fails.

\item {\tt Intro {\ident}}\\
  Applies {\tt Intro} but forces {\ident} to be the name of the
  introduced hypothesis.

  \ErrMsg \errindex{name {\ident} is already bound }

  \Rem {\tt Intro} doesn't check the whole current context. Actually,
  identifiers declared or defined in required modules can be used as
  {\ident} and, in this case, the old {\ident} of the module is no
  more reachable.

\item {\tt Intros \ident$_1$ \dots\ \ident$_n$} \\ 
  Is equivalent to the composed tactic {\tt Intro \ident$_1$; \dots\ ; Intro
    \ident$_n$}.\\
More generally, the \texttt{Intros} tactic takes a pattern as argument
  in order to introduce names for components of an inductive
  definition or to clear introduced hypotheses; 
  it will be explained in~\ref{Intros-pattern}.

\item {\tt Intros until {\ident}}
  \tacindex{Intros until}\\ 
  Repeats {\tt Intro} until it meets a premise of the goal having form
  {\tt (} {\ident}~{\tt :}~{\term} {\tt )}
  and discharges the variable named {\ident} of
  the current goal.

  \ErrMsg \errindex{No such hypothesis in current goal}\\
  
\item {\tt Intros until {\num}}
  \tacindex{Intros until}\\ 
  Repeats {\tt Intro} until the {\num}-th non-dependant premise. For instance,
  on the subgoal \verb+(x,y:nat)x=y->(z:nat)h=x->z=y+ the tactic
  \texttt{Intros until 2} is equivalent to \texttt{Intros x y H z H0} (assuming
  \texttt{x, y, H, z} and \texttt{H0} do not already occur in context).

  \ErrMsg \errindex{No such hypothesis in current goal}\\
  Happens when {\num} is 0 or is greater than the number of non-dependant
  products of the goal.

\item {\tt Intro after \ident}
  \tacindex{Intro after}\\
  Applies {\tt Intro} but puts the introduced
  hypothesis after the hypothesis \ident{} in the hypotheses.

\begin{ErrMsgs}
\item \errindex{No product even after head-reduction}
\item \errindex{No such hypothesis} : {\ident}
\end{ErrMsgs}

\item {\tt Intro \ident$_1$ after \ident$_2$}
  \tacindex{Intro ... after}\\
  Behaves as previously but \ident$_1$ is the name of the introduced
  hypothesis.
  It is equivalent to {\tt Intro \ident$_1$; Move \ident$_1$ after \ident$_2$}.

\begin{ErrMsgs}
\item \errindex{No product even after head-reduction}
\item \errindex{No such hypothesis} : {\ident}
\end{ErrMsgs}
\end{Variants}

\subsection{\tt Apply \term}
\tacindex{Apply}\label{Apply}
This tactic applies to any goal. 
The argument {\term} is a term well-formed in the local context.
The tactic {\tt Apply} tries to match the
current goal against the conclusion of the type of {\term}. If it
succeeds, then the tactic returns as many subgoals as the
instantiations of the premises of the type of
{\term}.

\begin{ErrMsgs}
\item \errindex{Impossible to unify \dots\ with \dots} \\
  Since higher order unification is undecidable, the {\tt Apply}
  tactic may fail when you think it should work.  In this case, if you
  know that the conclusion of {\term} and the current goal are
  unifiable, you can help the {\tt Apply} tactic by transforming your
  goal with the {\tt Change} or {\tt Pattern} tactics (see sections
  \ref{Pattern}, \ref{Change}).

\item \errindex{Cannot refine to conclusions with meta-variables}\\
  This occurs when some instantiations of premises of {\term} are not
  deducible from the unification. This is the case, for instance, when
  you want to apply a transitivity property. In this case, you have to
  use one of the variants below:
\end{ErrMsgs}

\begin{Variants}
\item{\tt Apply {\term} with {\term$_1$} \dots\ {\term$_n$}} 
  \tacindex{Apply \dots\ with}\\
  Provides {\tt Apply} with explicit instantiations for all dependent
  premises of the type of {\term} which do not occur in the
  conclusion and consequently cannot be found by unification. Notice
  that {\term$_1$} \dots\ {\term$_n$} must be given according to the order
  of these dependent premises of the type of {\term}.

  \ErrMsg \errindex{Not the right number of missing arguments}

\item{\tt Apply {\term} with {\vref$_1$} := {\term$_1$} \dots\ {\vref$_n$}
    := {\term$_n$}} \\ 
  This also provides {\tt Apply} with values for instantiating
    premises. But variables are referred by names and non dependent
    products by order (see syntax in the section~\ref{Binding-list}).

\item{\tt EApply \term}\tacindex{EApply}\label{EApply}\\
  The tactic {\tt EApply} behaves as {\tt Apply} but does not fail when
  no instantiation are deducible for some variables in the premises.
  Rather, it turns these variables into so-called existential variables
  which are variables still to instantiate. An existential variable is
  identified by a name of the form {\tt ?$n$} where $n$ is a number.
  The instantiation is intended to be found later in the proof.

  An example of use of {\tt EApply} is given in section
  \ref{EApply-example}.

\item{\tt LApply {\term}} \tacindex{LApply} \\

  This tactic applies to any goal, say {\tt G}.  The argument {\term}
  has to be well-formed in the current context, its type being
  reducible to a non-dependent product {\tt A -> B} with {\tt B}
  possibly containing products. Then it generates two subgoals {\tt
  B->G} and {\tt A}. Applying {\tt LApply H} (where {\tt H} has type
  {\tt A->B} and {\tt B} does not start with a product) does the same
  as giving the sequence {\tt Cut B. 2:Apply H.} where {\tt Cut} is
  described below.

  \Warning Be careful, when {\term} contains more than one non
  dependent product the tactic {\tt LApply} only takes into account the
  first product.

\end{Variants}

\subsection{\tt LetTac {\ident} {\tt :=} {\term}}\tacindex{LetTac}

This replaces {\term} by {\ident} in the goal and add the
equality {\ident {\tt =} \term} in the local context.

\begin{Variants}
\item {\tt LetTac {\ident} {\tt :=} {\term} {\tt in} {\tt Goal}}

This is equivalent to the above form.

\item {\tt LetTac {\ident$_0$} {\tt :=} {\term} {\tt in} {\ident$_1$}}

This behaves the same but substitutes {\term} not in the goal but in
the hypothesis named {\ident$_1$}.

\item {\tt LetTac {\ident$_0$} {\tt :=} {\term} {\tt in} {\num$_1$} \dots\
{\num$_n$} {\ident$_1$}}

This notation allows to specify which occurrences of the hypothesis
named {\ident$_1$} (or the goal if {\ident$_1$} is
the word {\tt Goal}) should be substituted. The occurrences are numbered
from left to right. A negative occurrence number means an occurrence
which should not be substituted. 

\item {\tt LetTac {\ident$_0$} {\tt :=} {\term} {\tt in} {\num$_1^1$} \dots\
{\num$_{n_1}^1$} {\ident$_1$} \dots {\num$_1^m$} \dots\
{\num$_{n_m}^m$} {\ident$_m$}}

This is the general form. It substitutes {\term} at occurrences
{\num$_1^i$} \dots\ {\num$_{n_i}^i$} of hypothesis {\ident$_i$}. One
of the {\ident}'s may be the word {\tt Goal}.
\end{Variants}

\subsection{\tt Cut {\form}}\tacindex{Cut}
This tactic applies to any goal. It implements the
``App''\index{Typing rules!App} rule given in section
\ref{Typed-terms}. {\tt Cut U} transforms the current goal \texttt{T} 
into the two following subgoals: {\tt U -> T} and \texttt{U}.

\begin{ErrMsgs}
\item \errindex{Not a proposition or a type}\\
  Arises when the argument {\form} is neither of type {\tt Prop}, {\tt
    Set} nor {\tt Type}.
\end{ErrMsgs}

%
% PAS CLAIR;
% DEVRAIT AU MOINS FAIRE UN INTRO;
% DEVRAIT ETRE REMPLACE PAR UN LET;
% MESSAGE D'ERREUR STUPIDE
% POURQUOI Specialize trans_equal ECHOUE ?
%\begin{Variants}
%\item {\tt Specialize \term} 
%   \tacindex{Specialize} \\
%   The argument {\tt t} should be a well-typed
%   term of type {\tt T}.  This tactics is to make a cut of a
%   proposition when you have already the proof of this proposition
%   (for example it is a theorem applied to variables of local
%   context). It is equivalent to {\tt Cut T. 2:Exact t}.
%
%\item {\tt Specialize {\term} with \vref$_1$ := {\term$_1$} \dots
%    \vref$_n$ := \term$_n$}
%  \tacindex{Specialize \dots\ with} \\ 
%  It is to provide the tactic with some explicit values to instantiate
%  premises of {\term} (see section \ref{Binding-list}).
%  Some other premises are inferred using type information and
%  unification. The resulting well-formed 
%  term being {\tt (\term~\term'$_1$\dots\term'$_k$)}
%  this tactic behaves as is used as 
%  {\tt Specialize (\term~\term'$_1$\dots\term'$_k$)} \\ 
%
%  \ErrMsg {\tt Metavariable wasn't in the metamap} \\ 
%  Arises when the information provided in the bindings list is not
%  sufficient.
%\item {\tt Specialize {\num} {\term} with \vref$_1$ := {\term$_1$} \dots\ 
% \vref$_n$:= \term$_n$}\\ 
%  The behavior is the same as before but only \num\ premises of 
%  \term\ will be kept.
%\end{Variants}

\subsection{\tt Generalize \term}
\tacindex{Generalize}\label{Generalize}
This tactic applies to any goal. It generalizes the conclusion w.r.t. one
subterm of it. For example:

\begin{coq_eval}
Goal (x,y:nat) (le O (plus (plus x y) y)).
Intros.
\end{coq_eval}
\begin{coq_example}
Show.
Generalize (plus (plus x y) y).
\end{coq_example}

\begin{coq_eval}
Abort.
\end{coq_eval}

If the goal is $G$ and $t$ is a subterm of type $T$
in the goal, then {\tt Generalize} \textit{t} replaces the goal by {\tt
(x:$T$)$G'$} where $G'$ is obtained from $G$ by replacing all
occurrences of $t$ by {\tt x}. The name of the variable (here {\tt n})
is chosen accordingly to $T$.

\begin{Variants}
\item {\tt Generalize \term$_1$ \dots\ \term$_n$} \\
  Is equivalent to {\tt Generalize \term$_n$; \dots\ ; Generalize
    \term$_1$}. Note that the sequence of \term$_i$'s are processed from 
    $n$ to $1$.

\item {\tt Generalize Dependent \term} \\
\tacindex{Generalize Dependent}
  This generalizes {\term} but
  also {\em all} hypotheses which depend on {\term}.

\end{Variants}

\subsection{\tt Change \term}
\tacindex{Change}\label{Change}
This tactic applies to any goal. It implements the rule
``Conv''\index{Typing rules!Conv} given in section \ref{Conv}.
{\tt Change U} replaces the current goal \T\ with a \U\ providing
that \U\ is well-formed and that \T\ and \U\ are
convertible.

\begin{ErrMsgs}
\item \errindex{convert-concl rule passed non-converting term}
\end{ErrMsgs}

\begin{Variants}
\item {\tt Change {\term} in {\ident}}\\ 
  \tacindex{Change \dots\ in}
  This applies the {\tt Change} tactic not to the goal but to the
  hypothesis {\ident}.
\end{Variants}

\SeeAlso \ref{Conversion-tactics}

\subsection{Bindings list}
\index{Binding list}\label{Binding-list}
\index[tactic]{Binding list}
A bindings list is generally used after the keyword {\tt with} in
tactics.  The general shape of a bindings list is {\tt \vref$_1$ :=
  \term$_1$ \dots\ \vref$_n$ := \term$_n$} where {\vref} is either an
{\ident} or a {\num}. It is used to provide a tactic with a list of
values (\term$_1$, \dots, \term$_n$) that have to be substituted
respectively to \vref$_1$, \dots, \vref$_n$. For all $i \in [1\dots\ n]$, if
\vref$_i$ is \ident$_i$ then it references the dependent product {\tt
  \ident$_i$:T} (for some type \T); if \vref$_i$ is \num$_i$ then it
references the \num$_i$-th non dependent premise.

A bindings list can also be a simple list of terms {\tt \term$_1$
\term$_2$ \dots\term$_n$}. In that case the references to which these
terms correspond are determined by the tactic. In case of {\tt Elim
\term} (see section \ref{Elim}) the terms should correspond to all the dependent products in
the type of \term\ while in the case of {\tt Apply \term} only the
dependent products which are not bound in the conclusion of the type
are given.


\section{Negation and contradiction}

\subsection{\tt Absurd \term}
\tacindex{Absurd}\label{Absurd}

This tactic applies to any goal. The argument {\term} is any
proposition {\tt P} of type {\tt Prop}. This tactic applies {\tt
False} elimination, that is it deduces the current goal from {\tt False},
and generates as
subgoals {\tt $\sim$P} and {\tt P}. It is very useful in proofs by
cases, where some cases are impossible. In most cases, 
\texttt{P} or $\sim$\texttt{P} is one of the hypotheses of the local
context.

\subsection{\tt Contradiction}
\label{Contradiction}
\tacindex{Contradiction}

This tactic applies to any goal. The {\tt Contradiction} tactic
attempts to find in the current context (after all {\tt Intros}) one
which is equivalent to {\tt False}. It permits to prune irrelevant
cases.  This tactic is a macro for the tactics sequence {\tt Intros;
  ElimType False; Assumption}.

\begin{ErrMsgs}
\item \errindex{No such assumption}
\end{ErrMsgs}


\section{Conversion tactics}
\index{Conversion tactics}
\index[tactic]{Conversion tactics}
\label{Conversion-tactics}

This set of tactics implements different specialized usages of the
tactic \texttt{Change}.

%%%%%%%%%%%%%%%%%%%%%%%%%%%%%%%%%%
%voir reduction__conv_x : histoires d'univers.
%%%%%%%%%%%%%%%%%%%%%%%%%%%%%%%%%%

\subsection{{\tt Cbv} \flag$_1$ \dots\ \flag$_n$, {\tt Lazy} \flag$_1$
\dots\ \flag$_n$ and {\tt Compute}}
\tacindex{Cbv}\tacindex{Lazy}

These parameterized reduction tactics apply to any goal and perform
the normalization of the goal according to the specified flags. Since
the reduction considered in \Coq\ include $\beta$ (reduction of
functional application), $\delta$ (unfolding
of transparent constants, see \ref{Transparent}) 
and $\iota$ (reduction of {\tt Cases}, {\tt Fix} and {\tt
CoFix} expressions), every flag is one of {\tt Beta}, {\tt Delta},
{\tt Iota}, {\tt [\ident$_1$\ldots\ident$_k$]} and 
{\tt -[\ident$_1$\ldots\ident$_k$]}. 
The last two flags give the list of
constants to unfold, or the list of constants not to unfold. These two
flags can occur only after the {\tt Delta} flag. The goal may be
normalized with two strategies: {\em lazy} ({\tt Lazy} tactic), or
{\em call-by-value} ({\tt Cbv} tactic).

The lazy strategy is a call-by-need strategy, with sharing of
reductions: the arguments of a function call are partially evaluated
only when necessary, but if an argument is used several times, it is
computed only once. This reduction is efficient for reducing
expressions with dead code. For instance, the proofs of a proposition
$\exists_T ~x. P(x)$ reduce to a pair of a witness $t$, and a proof
that $t$ verifies the predicate $P$. Most of the time, $t$ may be
computed without computing the proof of $P(t)$, thanks to the lazy
strategy.

The call-by-value strategy is the one used in ML languages: the
arguments of a function call are evaluated first, using a weak
reduction (no reduction under the $\lambda$-abstractions). Despite the
lazy strategy always performs fewer reductions than the call-by-value
strategy, the latter should be preferred for evaluating purely
computational expressions (i.e. with few dead code).

\begin{Variants}
\item {\tt Compute}\\
  \tacindex{Compute}
  This tactic is an alias for {\tt Cbv Beta Delta Iota}.
\end{Variants}

\begin{ErrMsgs}
\item \errindex{Delta must be specified before}\\
  A list of constants appeared before the {\tt Delta} flag.
\end{ErrMsgs}


\subsection{{\tt Red}}
\tacindex{Red}
This tactic applies to a goal which have form {\tt (x:T1)\dots(xk:Tk)(c
  t1 \dots\ tn)} where {\tt c} is a constant.  If {\tt c} is transparent
then it replaces {\tt c} with its definition (say {\tt t}) and then
reduces {\tt (t t1 \dots\ tn)} according to $\beta\iota$-reduction rules.

\begin{ErrMsgs}
\item \errindex{Not reducible}
\end{ErrMsgs}

\subsection{{\tt Hnf}}
\tacindex{Hnf}
This tactic applies to any goal. It replaces the current goal with its
head normal form according to the $\beta\delta\iota$-reduction
rules. {\tt Hnf} does not produce a real head normal form but either a
product or an applicative term in head normal form or a variable.

\Example
The term \verb+(n:nat)(plus (S n) (S n))+ is not reduced by {\tt Hnf}.

\Rem The $\delta$ rule will only be applied to transparent constants
(i.e. which have not been frozen with an {\tt Opaque} command; see
section \ref{Opaque}).

\subsection{\tt Simpl}
\tacindex{Simpl}
This tactic applies to any goal. The
tactic {\tt Simpl} first applies $\beta\iota$-reduction rule.
Then it expands transparent constants and tries to reduce {\tt T'}
according, once more, to 
$\beta\iota$ rules. But when the $\iota$ rule is not applicable then
possible $\delta$-reductions are not applied.  For instance trying to
use {\tt Simpl} on {\tt (plus n O)=n} will change nothing.

\subsection{\tt Unfold \ident}
\tacindex{Unfold}\label{Unfold}
This tactic applies to any goal. The argument {\ident} must be the
name of a defined transparent constant (see section
\ref{Simpl-definitions} and \ref{Transparent}). 
The tactic {\tt Unfold} applies the
$\delta$ rule to each occurrence of {\ident} in the current goal and
then replaces it with its $\beta\iota$-normal form.

\Warning If the constant is opaque, nothing will happen and no warning
is printed.

\begin{ErrMsgs}
\item {\ident} \errindex{does not denote an evaluable constant}
\end{ErrMsgs}

\begin{Variants}
\item {\tt Unfold {\ident}$_1$ \dots\ \ident$_n$}\\ 
  \tacindex{Unfold \dots\ in}
  Replaces {\em simultaneously} {\ident}$_1$, \dots, {\ident}$_n$ with
  their definitions and replaces the current goal with its
  $\beta\iota$ normal form.
\item {\tt Unfold \num$_1^1$ \dots\ \num$_i^1$ {\ident}$_1$ \dots\  \num$_1^n$
    \dots\ \num$_j^n$ \ident$_n$}\\ 
  The lists \num$_1^1$, \dots, \num$_i^1$ and \num$_1^n$, \dots, \num$_j^n$
  are to specify the occurrences of {\ident}$_1$, \dots, \ident$_n$ to be
  unfolded. Occurrences are located from left to right in the linear
  notation of terms.\\ 
  \ErrMsg {\tt bad occurrence numbers of {\ident}$_i$}
\end{Variants}

\subsection{{\tt Fold} \term}
\tacindex{Fold}

This tactic applies to any goal. \term\ is reduced using the {\tt Red}
tactic. Every occurrence of the resulting term in the goal is then
substituted for \term.

\begin{Variants}
\item {\tt Fold} \term$_1$ \dots\ \term$_n$ \\
  Equivalent to {\tt Fold} \term$_1${\tt;}\ldots{\tt; Fold} \term$_n$.
\end{Variants}

\subsection{{\tt Pattern {\term}}}
\tacindex{Pattern}\label{Pattern}
This command applies to any goal. The argument {\term} must be a free
subterm of the current goal.  The command {\tt Pattern} performs
$\beta$-expansion (the inverse of $\bt$-reduction) 
of the current goal (say \T) by
\begin{enumerate}
\item replacing all occurrences of {\term} in {\T} with a fresh variable
\item abstracting this variable
\item applying the abstracted goal to {\term}
\end{enumerate}
For instance, if the current goal {\T} is {\tt (P t)} when {\tt t} does not occur in
{\tt P} then {\tt Pattern t} transforms it into {\tt ([x:A](P x) t)}. This
command has to be used, for instance, when an {\tt Apply} command
fails on matching.

\begin{Variants}
\item {\tt Pattern {\num$_1$} \dots\ {\num$_n$} {\term}}\\
  Only the occurrences {\num$_1$} \dots\ {\num$_n$} of {\term} will be
  considered for $\beta$-expansion. Occurrences are located from left
  to right.
\item {\tt Pattern {\num$_1^1$} \dots\ {\num$_{n_1}^1$} {\term$_1$} \dots
    {\num$_1^m$} \dots\ {\num$_{n_m}^m$} {\term$_m$}}\\ 
  Will process occurrences \num$_1^1$, \dots, \num$_i^1$ of \term$_1$,
  \dots, \num$_1^m$, \dots, \num$_j^m$ of \term$_m$ starting from \term$_m$.
  Starting from a goal {\tt (P t$_1$\dots\ t$_m$)} with the {\tt
    t$_i$} which do not occur in $P$, the tactic {\tt Pattern
    t$_1$\dots\ t$_m$} generates the equivalent goal {\tt
    ([x$_1$:A$_1$]\dots\ [x$_m$:A$_m$](P x$_1$\dots\ x$_m$)
    t$_1$\dots\ t$_m$)}.\\ 
  If $t_i$ occurs in one of the generated types A$_j$ these
  occurrences will also be considered and possibly abstracted.
\end{Variants}

\subsection{Conversion tactics applied to hypotheses}

{\convtactic} {\tt in} \ident$_1$ \dots\ \ident$_n$ \\
Applies the conversion tactic {\convtactic} to the
hypotheses \ident$_1$, \ldots, \ident$_n$. The tactic {\convtactic} is
any of the conversion tactics listed in this section. 

\begin{ErrMsgs}
\item \errindex{No such hypothesis} : {\ident}.
\end{ErrMsgs}


\section{Introductions}
Introduction tactics address goals which are inductive constants.
They are used when one guesses that the goal can be obtained with one
of its constructors' type.

\subsection{\tt Constructor \num}
\label{Constructor}
\tacindex{Constructor}
This tactic applies to a goal such
that the head of its conclusion is an inductive constant (say {\tt
  I}).  The argument {\num} must be less or equal to the numbers of
constructor(s) of {\tt I}. Let {\tt ci} be the {\tt i}-th constructor
of {\tt I}, then {\tt Constructor i} is equivalent to {\tt Intros;
  Apply ci}.

\begin{ErrMsgs}
\item \errindex{Not an inductive product}
\item \errindex{Not enough Constructors}
\end{ErrMsgs}

\begin{Variants}
\item \texttt{Constructor} \\
  This tries \texttt{Constructor 1} then
  \texttt{Constructor 2}, \dots\ , then \texttt{Constructor} \textit{n}
  where \textit{n} if the number of constructors of the head of the
  goal.
\item {\tt Constructor \num~with} {\bindinglist}
  \tacindex{Constructor \dots\  with}\\ 
  Let {\tt ci} be the {\tt i}-th constructor of {\tt I}, then {\tt
    Constructor i with \bindinglist} is equivalent to {\tt Intros; Apply ci
    with \bindinglist}.

  \Warning the terms in the \bindinglist\ are checked
  in the context where {\tt Constructor} is executed and not in the
  context where {\tt Apply} is executed (the introductions are not
  taken into account).
\item {\tt Split}\tacindex{Split}\\
  Applies if {\tt I} has only one constructor, typically in the case
  of conjunction $A\wedge B$. It is equivalent to {\tt Constructor 1}.
\item {\tt Exists {\bindinglist}}\tacindex{Exists} \\
  Applies if {\tt I} has only one constructor, for instance in the
  case of existential quantification $\exists x\cdot P(x)$. 
  It is equivalent to {\tt Intros; Constructor 1 with \bindinglist}.
\item {\tt Left}\tacindex{Left},  {\tt Right}\tacindex{Right}\\
  Apply if {\tt I} has two constructors, for instance in the case of
  disjunction $A\vee B$. They are respectively equivalent to {\tt
    Constructor 1} and {\tt Constructor 2}.
\item {\tt Left \bindinglist}, {\tt Right \bindinglist}, 
      {\tt Split \bindinglist} \\
  Are equivalent to the corresponding {\tt Constructor $i$ with \bindinglist}.
\end{Variants}

\section{Eliminations (Induction and Case Analysis)}
Elimination tactics are useful to prove statements by induction or
case analysis.
Indeed, they make use of the elimination (or induction) principles
generated with inductive definitions (see section
\ref{Cic-inductive-definitions}).

\subsection{\tt Elim \term}
\tacindex{Elim}\label{Elim}
This tactic applies to any goal. The type of the argument
{\term} must be an inductive constant. Then according to the type of
the goal, the tactic {\tt Elim} chooses the right destructor and
applies it (as in the case of the {\tt Apply} tactic). For instance,
assume that our proof context contains {\tt n:nat}, assume that our
current goal is {\tt T} of type {\tt Prop}, then
{\tt Elim n} is equivalent to {\tt Apply nat\_ind with n:=n}.

\begin{ErrMsgs}
\item \errindex{Not an inductive product}
\item \errindex{Cannot refine to conclusions with meta-variables}\\ As {\tt
    Elim} uses {\tt Apply}, see section \ref{Apply} and the variant
  {\tt Elim \dots\ with \dots} below.
\end{ErrMsgs}

\begin{Variants}
\item {\tt Elim \term} also works when the type of {\term} starts with
   products and the head symbol is an inductive definition. In that
  case the tactic tries both to find an object in the inductive
  definition and to use this inductive definition for elimination.  In
  case of non-dependent products in the type, subgoals are generated
  corresponding to the hypotheses. In the case of dependent products,
  the tactic will try to find an instance for which the elimination
  lemma applies.

\item {\tt Elim {\term} with \term$_1$ \dots\ \term$_n$}
  \tacindex{Elim \dots\ with} \\ 
  Allows the user to give explicitly the values for dependent
  premises of the elimination schema. All arguments must be given.\\ 
  \ErrMsg \errindex{Not the right number of dependent arguments}
\item{\tt Elim {\term} with {\vref$_1$} := {\term$_1$} \dots\ {\vref$_n$}
    := {\term$_n$}} \\ 
  Provides also {\tt Elim} with values for instantiating premises by
  associating explicitly variables (or non dependent products) with
  their intended instance.
\item{\tt Elim {\term$_1$} using {\term$_2$}}
\tacindex{Elim \dots\ using} \\
  Allows the user to give explicitly an elimination predicate
  {\term$_2$} which is not the standard one for the underlying
  inductive type of {\term$_1$}. Each of the {\term$_1$} and {\term$_2$} is
  either a simple term or a term with a bindings list (see
  \ref{Binding-list}). 
\item {\tt ElimType \form}\tacindex{ElimType}\\
  The argument {\form} must be inductively defined. {\tt ElimType I}
  is equivalent to {\tt Cut I. Intro H{\rm\sl n}; Elim H{\rm\sl n}; 
    Clear H{\rm\sl n}}
  Therefore the hypothesis {\tt H{\rm\sl n}} will not appear in the
  context(s) of the subgoal(s).\\ 
  Conversely, if {\tt t} is a term of (inductive) type {\tt I} and
  which does not occur in the goal then
  {\tt Elim t} is equivalent to {\tt ElimType I; 2: Exact t.}

  \ErrMsg \errindex{Impossible to unify \dots\ with \dots} \\ Arises when
  {\form} needs to be applied to parameters.

\item {\tt Induction \ident}\tacindex{Induction}\\
  When {\ident} is a quantified variable of the goal, this is
    equivalent to {\tt Intros until {\ident}; Pattern {\ident}; Elim
    {\ident}}

  Otherwise, it behaves as a ``user-friendly'' version of {\tt Elim
  \ident}: it does not duplicate {\ident} after induction and it
  automatically generalizes the hypotheses dependent on {\ident} or
  dependent on some atomic arguments of the inductive type of {\ident}.

\item {\tt Induction {\num}}\\ 
  Is analogous to {\tt Induction {\ident}} but for the {\num}-th
  non-dependent premise of the goal.
\end{Variants}

\subsection{\tt Case \term}\label{Case}\tacindex{Case}
The tactic {\tt Case} is used to perform case
analysis without recursion. The type of {\term} must be inductively defined. 

\begin{Variants}
\item {\tt Case {\term} with \term$_1$ \dots\ \term$_n$}
  \tacindex{Case \dots\ with}\\ 
  Analogous to {\tt Elim \dots\ with} above.
\item {\tt Destruct \ident}\tacindex{Destruct}\\
  Is equivalent to the tactical {\tt Intros Until \ident; Case \ident}.
\item {\tt Destruct {\num}}\\
  Is equivalent to {\tt Destruct {\ident}} but for the {\num}-th non
  dependent premises of the goal.
\end{Variants}

\subsection{\tt Intros \pattern}\label{Intros-pattern}
\tacindex{Intros \pattern}

The tactic {\tt Intros} applied to a pattern performs both
introduction of variables and case analysis in order to give names to
components of an hypothesis.

A pattern is either:
\begin{itemize}
\item the wildcard: {\tt \_}
\item a variable
\item a list of patterns: $p_1~\ldots~p_n$
\item a disjunction of patterns: {\tt [}$p_1$ {\tt |} {\ldots} {\tt
|} $p_n$ {\tt ]}
\item a conjunction of patterns: {\tt (} $p_1$ {\tt ,} {\ldots} {\tt
,} $p_n$ {\tt )}
\end{itemize}

The behavior of \texttt{Intros} is defined inductively over the
structure of the pattern given as argument:
\begin{itemize}
\item introduction on the wildcard do the introduction and then
  immediately clear (cf~\ref{Clear}) the corresponding hypothesis;
\item introduction on a variable behaves like described in~\ref{Intro}; 
\item introduction over a
list of patterns $p_1~\ldots~p_n$ is equivalent to the sequence of
introductions over the patterns namely:
\texttt{Intros $p_1$;\ldots; Intros $p_n$}, the goal should start with
at least $n$ products;
\item introduction over a
disjunction of patterns $[p_1~|~~\ldots~|~p_n]$, it 
introduces a new variable $X$, its type should be an inductive
definition with $n$
constructors, then it performs a case analysis over $X$ 
(which generates $n$ subgoals), it 
clears $X$ and performs on each generated subgoals the corresponding
\texttt{Intros}~$p_i$ tactic;
\item introduction over a 
conjunction  of patterns $(p_1,\ldots,p_n)$, it
introduces a new variable $X$, its type should be an inductive 
definition with $1$
constructor with (at least) $n$ arguments, then it performs a case 
analysis over $X$ 
(which generates $1$ subgoal with at least $n$ products), it 
clears $X$ and performs an introduction over the list of patterns $p_1~\ldots~p_n$.
\end{itemize}
\begin{coq_example}
Lemma intros_test : (A,B,C:Prop)(A\/(B/\C))->(A->C)->C.
Intros A B C [a|(_,c)] f.
Apply (f a).
Proof c.
\end{coq_example}

%\subsection{\tt FixPoint \dots}\tacindex{Fixpoint}
%Not yet documented.

\subsection {\tt Double Induction \num$_1$ \num$_2$}
\tacindex{Double Induction}
This tactic applies to any goal. If the \num$_1$th and  \num$_2$th
premises of the goal have an inductive type, then this tactic
performs double induction on these premises.
For instance, if the current goal is \verb+(n,m:nat)(P n m)+ then,
{\tt Double Induction 1 2} yields the four cases with their respective
inductive hypothesis. In particular the case for
\verb+(P (S n) (S m))+
with the inductive hypothesis about both \verb+n+ and \verb+m+.

\subsection{\tt Decompose [ {\ident$_1$} \dots\ {\ident$_n$} ] \term}
\label{Decompose}
\tacindex{Decompose}
This tactic allows to recursively decompose a
complex proposition in order to obtain atomic ones.
Example: 

\begin{coq_eval}
Reset Initial.
\end{coq_eval}
\begin{coq_example}
Lemma ex1: (A,B,C:Prop)(A/\B/\C \/ B/\C \/ C/\A) -> C.
Intros A B C H; Decompose [and or] H; Assumption.
\end{coq_example}
\begin{coq_example*}
Qed.
\end{coq_example*}

{\tt Decompose} does not work on right-hand sides of implications or products.

\begin{Variants}
  
\item {\tt Decompose Sum \term}\tacindex{Decompose Sum}
  This decomposes sum types (like \texttt{or}).
\item {\tt Decompose Record \term}\tacindex{Decompose Record}
  This decomposes record types (inductive types with one constructor,
  like \texttt{and} and \texttt{exists} and those defined with the
  \texttt{Record} macro, see p. \pageref{Record}).
\end{Variants}

\section{Equality}
These tactics use the equality {\tt
eq:(A:Set)A->A->Prop} defined in file {\tt Logic.v} and the equality
{\tt eqT:(A:Type)A->A->Prop} defined in file {\tt
Logic\_Type.v} (see section \ref{Equality}). They
are simply written {\tt t=u} and {\tt t==u},
respectively.  In the following, the notation {\tt
t=u} will represent either one of these two
equalities.

\subsection{\tt Rewrite \term}
\label{Rewrite}
\tacindex{Rewrite}
This tactic applies to any goal. The type of {\term}
must have the form

\texttt{(x$_1$:A$_1$) \dots\ (x$_n$:A$_n$)}\term$_1${\tt =}\term$_2$. 

\noindent Then {\tt Rewrite \term} replaces every occurrence of 
\term$_1$ by \term$_2$ in the goal. Some of the variables x$_1$ are
solved by unification, and some of the types \texttt{A}$_1$, \dots,
\texttt{A}$_n$ become new subgoals.

\Rem In case the type of  
\term$_1$ contains occurrences of variables bound in the
type of \term, the tactic tries first to find a subterm of the goal
which matches this term in order to find a closed instance \term$'_1$
of \term$_1$, and then all instances of \term$'_1$ will be replaced.

\begin{ErrMsgs}
\item \errindex{The term provided does not end with an equation}

\item \errindex{Tactic generated a subgoal identical to the original goal}\\
This happens if \term$_1$ does not occur in the goal.
\end{ErrMsgs}

\begin{Variants}
\item {\tt Rewrite -> {\term}}\tacindex{Rewrite ->}\\
  Is equivalent to {\tt Rewrite \term}

\item {\tt Rewrite <- {\term}}\tacindex{Rewrite <-}\\
  Uses the equality \term$_1${\tt=}\term$_2$ from right to left

\item {\tt Rewrite {\term} in {\ident}}
  \tacindex{Rewrite \dots\ in}\\
  Analogous to  {\tt Rewrite {\term}} but rewriting is done in the
  hypothesis named {\ident}.

\item {\tt Rewrite -> {\term} in {\ident}}
  \tacindex{Rewrite -> \dots\ in}\\
  Behaves as {\tt Rewrite {\term} in {\ident}}.
 
\item {\tt Rewrite <- {\term} in {\ident}}\\
  \tacindex{Rewrite <- \dots\ in}
  Uses the equality \term$_1${\tt=}\term$_2$ from right to left to
  rewrite in the hypothesis named {\ident}.
\end{Variants}


\subsection{\tt CutRewrite -> \term$_1$ = \term$_2$}
\label{CutRewrite}
\tacindex{CutRewrite}

This tactic acts like {\tt Replace {\term$_1$} with {\term$_2$}}
(see below).

\subsection{\tt Replace {\term$_1$} with {\term$_2$}}
\tacindex{Replace \dots\ with}
This tactic applies to any goal. It replaces all free occurrences of
{\term$_1$} in the current goal with {\term$_2$} and generates the
equality {\term$_2$}{\tt =}{\term$_1$} as a subgoal. It is equivalent
to {\tt Cut \term$_1$=\term$_2$; Intro H{\sl n}; Rewrite H{\sl n}; 
  Clear H{\sl n}}.

%N'existe pas...
%\begin{Variants}
%  
%\item {\tt Replace {\term$_1$} with {\term$_2$} in \ident}
%  This replaces {\term$_1$} with {\term$_2$} in the hypothesis named
%  \ident, and generates the subgoal {\term$_2$}{\tt =}{\term$_1$}. 
%  \begin{ErrMsgs}
%  \item \errindex{No such hypothesis}
%  \end{ErrMsgs}
%
%\end{Variants}

\subsection{\tt Reflexivity}
\label{Reflexivity}
\tacindex{Reflexivity}
This tactic applies to a goal which has the form {\tt t=u}. It checks
that {\tt t} and {\tt u} are convertible and then solves the goal.
It is equivalent to {\tt Apply refl\_equal} (or {\tt Apply
  refl\_equalT} for an equality in the \Type\ universe).

\begin{ErrMsgs}
\item \errindex{The conclusion is not a substitutive equation}
\item \errindex{Impossible to unify \dots\ with ..}
\end{ErrMsgs}

\subsection{\tt Symmetry}\tacindex{Symmetry}
This tactic applies to a goal which have form {\tt t=u}
(resp. \texttt{t==u}) and changes it into {\tt u=t} (resp. \texttt{u==t}).

\subsection{\tt Transitivity \term}\tacindex{Transitivity}
This tactic applies to a goal which have form {\tt t=u}
and transforms it into the two subgoals 
{\tt t={\term}} and {\tt {\term}=u}.

\section{Equality and inductive sets}
We describe in this section some special purpose
tactics dealing with equality and inductive sets or
types.  These tactics use the equalities {\tt
eq:(A:Set)A->A->Prop} defined in file {\tt Logic.v}
and {\tt eqT:(A:Type)A->A->Prop} defined in file
{\tt Logic\_Type.v} (see section \ref{Equality}).
They are written {\tt t=u} and {\tt t==u},
respectively. In the following, unless it is stated
otherwise, the notation {\tt t=u} will represent
either one of these two equalities.

\subsection{\tt Decide Equality}
\label{DecideEquality}
\tacindex{Decide Equality}
This tactic solves a goal of the form
$(x,y:R)\{x=y\}+\{\verb|~|x=y\}$, where $R$ is an inductive type
such that its constructors do not take proofs or functions as
arguments, nor objects in dependent types.

\begin{Variants}
\item {\tt Decide Equality {\term}$_1$ {\term}$_2$ }.\\
 Solves a goal of the form {\tt \{}\term$_1${\tt =}\term$_2${\tt
\}+\{\verb|~|}\term$_1${\tt =}\term$_2${\tt \}}.
\end{Variants}

\subsection{\tt Compare \term$_1$ \term$_2$}
\tacindex{Compare}
This tactic compares two given objects \term$_1$ and \term$_2$ 
of an inductive datatype. If $G$ is the current goal, it leaves the sub-goals
\term$_1${\tt =}\term$_2$ {\tt ->} $G$ and \verb|~|\term$_1${\tt =}\term$_2$
{\tt ->} $G$. The type
of \term$_1$ and \term$_2$ must satisfy the same restrictions as in the tactic
\texttt{Decide Equality}.

\subsection {\tt Discriminate {\ident}}
\label{Discriminate}
\tacindex{Discriminate}
This tactic proves any goal from an absurd
hypothesis stating that two structurally different terms of an
inductive set are equal. For example, from the hypothesis {\tt (S (S
  O))=(S O)} we can derive by absurdity any proposition.  Let {\ident}
be a hypothesis of type {\tt{\term$_1$} = {\term$_2$}} in the local
context, {\term$_1$} and {\term$_2$} being elements of an inductive set.
To build the proof, the tactic traverses the normal
forms\footnote{Recall: opaque constants will not be expanded by
  $\delta$ reductions} of {\term$_1$} and {\term$_2$} looking for a
couple of subterms {\tt u} and {\tt w} ({\tt u} subterm of the normal
form of {\term$_1$} and {\tt w} subterm of the normal form of
{\term$_2$}), placed at the same positions and whose
head symbols are two different constructors. If such a couple of subterms
exists, then the proof of the current goal is completed,
otherwise the tactic fails.

\begin{ErrMsgs}
\item {\ident} \errindex{Not a discriminable equality} \\
  occurs when the type of the specified hypothesis is not an equation.
\end{ErrMsgs}  


\begin{Variants}
\item {\tt Discriminate}\tacindex{Discriminate} \\
  It applies to a goal of the form {\tt
    \verb=~={\term$_1$}={\term$_2$}} and it is equivalent to: 
  {\tt Unfold not; Intro {\ident}} ; {\tt Discriminate
    {\ident}}.

  \begin{ErrMsgs}
  \item \errindex{No discriminable equalities} \\
  occurs when the goal does not verify the expected preconditions.
  \end{ErrMsgs}
\end{Variants}

\subsection{\tt Injection {\ident}}
\label{Injection}
\tacindex{Injection}
The {\tt Injection} tactic is based on the fact that constructors of
inductive sets are injections. That means that if $c$ is a constructor
of an inductive set, and if $(c~\vec{t_1})$ and $(c~\vec{t_2})$ are two
terms that are equal then $~\vec{t_1}$ and $~\vec{t_2}$ are equal
too.

If {\ident} is an hypothesis of type {\tt {\term$_1$} = {\term$_2$}},
then {\tt Injection} behaves as applying injection as deep as possible to
derive the equality of all the subterms of {\term$_1$} and {\term$_2$}
placed in the same positions. For example, from the hypothesis {\tt (S
  (S n))=(S (S (S m))} we may derive {\tt n=(S m)}.  To use this
tactic {\term$_1$} and {\term$_2$} should be elements of an inductive
set and they should be neither explicitly equal, nor structurally
different. We mean by this that, if {\tt n$_1$} and {\tt n$_2$} are
their respective normal forms, then:
\begin{itemize}
\item {\tt n$_1$} and {\tt n$_2$} should not be syntactically equal,
\item there must not exist any couple of subterms {\tt u} and {\tt w},
  {\tt u} subterm of {\tt n$_1$} and {\tt w} subterm of {\tt n$_2$} ,
  placed in the same positions and having different constructors as
  head symbols.
\end{itemize}
If these conditions are satisfied, then, the tactic derives the
equality of all the subterms of {\term$_1$} and {\term$_2$} placed in
the same positions and puts them as antecedents of the current goal.

\Example Consider the following goal:

\begin{coq_example*}
Inductive list : Set  := 
        nil: list | cons: nat-> list -> list.
Variable  P : list -> Prop.
\end{coq_example*}
\begin{coq_eval}
Lemma ex: (l:list)(n:nat)(P nil)->(cons n l)=(cons O nil)->(P l).
Intros l n H H0.
\end{coq_eval}
\begin{coq_example}
Show.
Injection H0.
\end{coq_example}
\begin{coq_eval}
Abort.
\end{coq_eval}

Beware that \texttt{Injection} yields always an equality in a sigma type
whenever the injected object has a dependent type.

\begin{ErrMsgs}
\item {\ident} \errindex{is not a projectable equality} 
  occurs when the type of
  the hypothesis $id$ does not verify the preconditions.
\item \errindex{Not an equation} occurs when the type of the
  hypothesis $id$ is not an equation.
\end{ErrMsgs}


\begin{Variants}
\item{\tt Injection}\tacindex{Injection} \\
  If the current goal is of the form {\tt \verb=~={\term$_1$}={\term$_2$}}, 
  the tactic computes the head normal form
  of the goal and then behaves as the sequence: {\tt Unfold not; Intro
    {\ident}; Injection {\ident}}. \\
  
  \ErrMsg \errindex{goal does not satisfy the expected preconditions}
\end{Variants}

\subsection{\tt Simplify\_eq {\ident}}
\tacindex{Simplify\_eq}
\label{Simplify-eq}
Let {\ident} be the name of an hypothesis of type {\tt
  {\term$_1$}={\term$_2$}} in the local context. If {\term$_1$} and
{\term$_2$} are structurally different (in the sense described for the
tactic {\tt Discriminate}), then the tactic {\tt Simplify\_eq} behaves as {\tt
  Discriminate {\ident}} otherwise it behaves as {\tt Injection
  {\ident}}.

\begin{Variants}
\item{\tt Simplify\_eq}
If the current goal has form $\verb=~=t_1=t_2$, then this tactic does 
\texttt{Hnf; Intro {\ident}; Simplify\_eq {\ident}}.
\end{Variants}

\subsection{\tt Dependent Rewrite -> {\ident}}
\tacindex{Dependent Rewrite ->}
\label{Dependent-Rewrite}
This tactic applies to any goal.  If \ident\ has type 
\verb+(existS A B a b)=(existS A B a' b')+ 
in the local context (i.e. each term of the
equality has a sigma type $\{ a:A~ \&~(B~a)\}$) this tactic rewrites
\verb+a+ into \verb+a'+ and \verb+b+ into \verb+b'+ in the current
goal. This tactic works even if $B$ is also a sigma type.  This kind
of equalities between dependent pairs may be derived by the injection
and inversion tactics.

\begin{Variants}
\item{\tt Dependent Rewrite <- {\ident}}
\tacindex{Dependent Rewrite <-} \\
Analogous to {\tt Dependent Rewrite ->} but uses the equality from
right to left.
\end{Variants}

\section{Inversion}
\label{Inversion}

\subsection{\tt Inversion {\ident}}\tacindex{Inversion}

Let the type of \ident~ in the local context be $(I~\vec{t})$,
where $I$ is a (co)inductive predicate. Then,
\texttt{Inversion} applied to \ident~ derives for each possible
constructor $c_i$ of $(I~\vec{t})$, {\bf all} the necessary
conditions that should hold for the instance $(I~\vec{t})$ to be
proved by $c_i$.

\begin{Variants}
\item \texttt{Inversion\_clear} \ident\\
  \tacindex{Inversion\_clear}
  That does \texttt{Inversion} and then erases \ident~ from the
  context.
\item \texttt{Inversion } \ident~ \texttt{in} \ident$_1$ \dots\ \ident$_n$\\
  \tacindex{Inversion \dots\ in}
  Let \ident$_1$ \dots\ \ident$_n$, be identifiers in the local context. This
  tactic behaves as generalizing \ident$_1$ \dots\ \ident$_n$, and
  then performing \texttt{Inversion}.
\item \texttt{Inversion\_clear} \ident~ \texttt{in} \ident$_1$ \ldots
  \ident$_n$\\ 
  \tacindex{Inversion\_clear \dots\ in}
  Let \ident$_1$ \dots\ \ident$_n$, be identifiers in the local context. This
  tactic behaves as generalizing \ident$_1$ \dots\ \ident$_n$, and
  then performing {\tt Inversion\_clear}.
\item \texttt{Dependent Inversion} \ident~\\
  \tacindex{Dependent Inversion}
  That must be used when \ident\ appears in the current goal. 
  It acts like \texttt{Inversion} and then substitutes \ident\ for the
  corresponding term in the goal.
\item \texttt{Dependent Inversion\_clear} \ident~\\
  \tacindex{Dependent Inversion\_clear}
  Like \texttt{Dependant Inversion}, except that \ident~ is cleared
  from the local context.
\item \texttt{Dependent Inversion } \ident~ \texttt{ with } \term \\
  \tacindex{Dependent Inversion \dots\ with}
  This variant allow to give the good generalization of the goal. It
  is useful when the system fails to generalize the goal automatically. If
  \ident~ has type $(I~\vec{t})$ and $I$ has type
  $(\vec{x}:\vec{T})s$,   then \term~  must be of type
  $I:(\vec{x}:\vec{T})(I~\vec{x})\rightarrow s'$ where $s'$ is the
  type of the goal.
\item \texttt{Dependent Inversion\_clear } \ident~ \texttt{ with } \term\\
  \tacindex{Dependent Inversion\_clear \dots\ with}
  Like \texttt{Dependant Inversion \dots\ with} but clears \ident from
  the local context.
\item \texttt{Inversion} \ident \texttt{ using} \ident$'$ \\
  \tacindex{Inversion \dots\ using}
  Let \ident~ have type $(I~\vec{t})$ ($I$ an inductive
  predicate) in the local context, and \ident$'$ be a (dependent) inversion
  lemma. Then, this tactic refines the current goal with the specified
  lemma.
\item \texttt{Inversion} \ident~ \texttt{using} \ident$'$ 
  \texttt{in} \ident$_1$\dots\ \ident$_n$\\
  \tacindex{Inversion \dots\ using \dots\ in}
  This tactic behaves as generalizing \ident$_1$\dots\ \ident$_n$,
  then doing \texttt{Inversion}\ident~\texttt{using} \ident$'$.
\item \texttt{Simple Inversion} \ident~\\
  \tacindex{Simple Inversion}
  It is a very primitive inversion tactic that derives all the necessary
  equalities  but it does not simplify the  constraints as
  \texttt{Inversion} do.
\end{Variants}

\SeeAlso \ref{Inversion-examples} for detailed examples

\subsection{\tt Derive Inversion \ident~ with
  $(\vec{x}:\vec{T})(I~\vec{t})$ Sort \sort}
\label{Derive-Inversion}
\comindex{Derive Inversion}
\index[tactic]{Derive Inversion@{\tt Derive Inversion}}

This command generates an inversion principle for the
\texttt{Inversion \dots\ using} tactic.
Let $I$ be an inductive predicate and $\vec{x}$ the variables
occurring in $\vec{t}$. This command generates and stocks the
inversion lemma for the sort \sort~ corresponding to the instance
$(\vec{x}:\vec{T})(I~\vec{t})$ with the name \ident~ in the {\bf
global} environment. When applied it is equivalent to have inverted
the instance with the tactic {\tt Inversion}.

\begin{Variants}
\item \texttt{Derive Inversion\_clear} \ident~ \texttt{with}
  \comindex{Derive Inversion\_clear}
  $(\vec{x}:\vec{T})(I~\vec{t})$ \texttt{Sort} \sort~ \\ 
  \index{Derive Inversion\_clear \dots\ with}
  When applied it is equivalent to having
  inverted the instance with the tactic \texttt{Inversion}
  replaced by the tactic \texttt{Inversion\_clear}.
\item \texttt{Derive Dependent Inversion} \ident~ \texttt{with}
  $(\vec{x}:\vec{T})(I~\vec{t})$ \texttt{Sort} \sort~\\
  \comindex{Derive Dependent Inversion}
  When applied it is equivalent to having
  inverted the instance with the tactic \texttt{Dependent Inversion}.
\item \texttt{Derive Dependent Inversion\_clear} \ident~ \texttt{with}
  $(\vec{x}:\vec{T})(I~\vec{t})$ \texttt{Sort} \sort~\\
  \comindex{Derive Dependent Inversion\_clear}
  When applied it is equivalent to having
  inverted the instance with the tactic \texttt{Dependent Inversion\_clear}.
\end{Variants}

\SeeAlso \ref{Inversion-examples} for examples

\subsection{\texttt{Quote} \ident}\tacindex{Quote}
\index[default]2-level approach

This kind of inversion has nothing to do with the tactic
\texttt{Inversion} above. This tactic does \texttt{Change (\ident\
  t)}, where \texttt{t} is a term build in order to ensure the
convertibility. In other words, it does inversion of the function
\ident. This function must be a fixpoint on a simple recursive
datatype: see \ref{Quote-examples} for the full details.

\begin{ErrMsgs}
\item \errindex{Quote: not a simple fixpoint}\\
  Happens when \texttt{Quote} is not able to perform inversion properly.
\end{ErrMsgs}

\begin{Variants}
\item \texttt{Quote {\ident} [ \ident$_1$ \dots \ident$_n$ ]}\\
  All terms that are build only with \ident$_1$ \dots \ident$_n$ will be
  considered by \texttt{Quote} as constants rather than variables.
\end{Variants}

\SeeAlso file \texttt{theories/DEMOS/DemoQuote.v} in the distribution

\section{Automatizing}
\label{Automatizing}

\subsection{\tt Auto}
\tacindex{Auto}
This tactic implements a Prolog-like resolution procedure to solve the
current goal. It first tries to solve the goal using the {\tt
  Assumption} tactic, then it reduces the goal to an atomic one using
{\tt Intros} and introducing the newly generated hypotheses as hints.
Then it looks at the list of tactics associated to the head symbol of
the goal and tries to apply one of them (starting from the tactics
with lower cost). This process is recursively applied to the generated
subgoals. 

By default, Auto only uses the hypotheses of the current goal and the
hints of the database named "core". 

\begin{Variants}
\item  {\tt Auto \num}\\
  Forces the search depth to be \num. The maximal search depth is 5 by default.
\item {\tt Auto with \ident$_1$ \dots\ \ident$_n$}\\
  Uses the hint databases $\ident_1$ \dots\ $\ident_n$ in addition to
  the database "core". See section \ref{Hints-databases} for the list
  of pre-defined databases and the way to create or extend a database.
  This option can be combined with the previous one.
\item {\tt Auto with *}\\
  Uses all existing hint databases, minus the special database
  "v62". See section \ref{Hints-databases}
\item {\tt Trivial}\tacindex{Trivial}\\
  This tactic is a restriction of {\tt Auto} that is not recursive and 
  tries only hints which cost is 0. Typically it solves trivial
  equalities like $X=X$.
\item \texttt{Trivial with \ident$_1$ \dots\ \ident$_n$}\\
\item \texttt{Trivial with *}\\
\end{Variants}

\Rem {\tt Auto} either solves completely the goal or else leave it
intact. \texttt{Auto} and \texttt{Trivial} never fail.

\SeeAlso section \ref{Hints-databases}

\subsection{\tt EAuto}\tacindex{EAuto}\label{EAuto}

This tactic generalizes {\tt Auto}. In contrast with 
the latter, {\tt EAuto} uses unification of the goal
against the hints rather than pattern-matching
(in other words, it uses {\tt EApply} instead of
{\tt Apply}).
As a consequence, {\tt EAuto} can solve such a goal:

\begin{coq_example}
Hints Resolve ex_intro.
Goal (P:nat->Prop)(P O)->(EX n | (P n)).
EAuto.
\end{coq_example}
\begin{coq_eval}
Abort.
\end{coq_eval}

Note that {\tt ex\_intro} should be declared as an
hint.

\SeeAlso section \ref{Hints-databases}

\subsection{\tt Prolog [ \term$_1$ \dots\ \term$_n$ ] \num}
\tacindex{Prolog}\label{Prolog}
This tactic, implemented by Chet Murthy, is based upon the concept of
existential variables of Gilles Dowek, stating that resolution is a
kind of unification. It tries to solve the current goal using the {\tt
  Assumption} tactic, the {\tt Intro} tactic, and applying hypotheses
of the local context and terms of the given list {\tt [ \term$_1$
  \dots\ \term$_n$\ ]}.  It is more powerful than {\tt Auto} since it
may apply to any theorem, even those of the form {\tt (x:A)(P x) -> Q}
where {\tt x} does not appear free in {\tt Q}.  The maximal search
depth is {\tt \num}.

\begin{ErrMsgs}
\item \errindex{Prolog failed}\\
  The Prolog tactic was not able to prove the subgoal.
\end{ErrMsgs}

\subsection{\tt Tauto}
\tacindex{Tauto}\label{Tauto}

This tactic implements a decision procedure for intuitionistic propositional
calculus based on the contraction-free sequent calculi LJT* of Roy Dyckhoff
\cite{Dyc92}. Note that {\tt Tauto} succeeds on any instance of an
intuitionistic tautological proposition. For instance, it succeeds on:

\begin{verbatim}
(x:nat)(P:nat->Prop)x=O\/(P x)->~x=O->(P x)
\end{verbatim}

\noindent{}while {\tt Auto} fails.

\subsection{\tt Intuition}
\tacindex{Intuition}\label{Intuition}

The tactic \verb1Intuition1 takes advantage of the search-tree builded
by the decision procedure involved in the tactic {\tt Tauto}. It uses
this information to generate a set of subgoals equivalent to the
original one (but simpler than it) and applies the tactic 
{\tt Auto  with *} to them \cite{Mun94}. At the end, {\tt Intuition}
performs {\tt Intros}.

For instance, the tactic {\tt Intuition} applied to the goal
\begin{verbatim}
((x:nat)(P x))/\B->((y:nat)(P y))/\(P O)\/B/\(P O)
\end{verbatim}
internally replaces it by the equivalent one:
\begin{verbatim}
((x:nat)(P x) -> B -> (P O))
\end{verbatim}
and then uses {\tt Auto with *} which completes the proof.

Originally due to C�sar~Mu�oz, these tactics ({\tt Tauto} and {\tt Intuition})
have been completely reenginered by David~Delahaye using mainly the tactic
language (see chapter~\ref{TacticLanguage}). The code is now quite shorter and
a significant increase in performances has been noticed. The general behavior
with respect to dependent types has slightly changed to get clearer semantics.
This may lead to some incompatibilities.

\SeeAlso file \texttt{contrib/Rocq/DEMOS/Demo\_tauto.v}

% \subsection{\tt Linear}\tacindex{Linear}\label{Linear}
% The tactic \texttt{Linear}, due to Jean-Christophe Filli�atre
% \cite{Fil94}, implements a decision procedure for {\em Direct
%   Predicate Calculus}, that is first-order Gentzen's Sequent Calculus
% without contraction rules \cite{KeWe84,BeKe92}.  Intuitively, a
% first-order goal is provable in Direct Predicate Calculus if it can be
% proved using each hypothesis at most once.

% Unlike the previous tactics, the \texttt{Linear} tactic does not belong
% to the initial state of the system, and it must be loaded explicitly
% with the command

% \begin{coq_example*}
% Require Linear.
% \end{coq_example*}

% For instance, assuming that \texttt{even} and \texttt{odd} are two
% predicates on natural numbers, and \texttt{a} of type \texttt{nat}, the
% tactic \texttt{Linear} solves the following goal

% \begin{coq_eval}
% Variables even,odd : nat -> Prop.
% Variable a:nat.
% \end{coq_eval}

% \begin{coq_example*}
% Lemma example : (even a) 
%               -> ((x:nat)((even x)->(odd (S x))))
%               -> (EX y | (odd y)).
% \end{coq_example*}

% You can find examples of the use of \texttt{Linear} in
% \texttt{theories/DEMOS/DemoLinear.v}.
% \begin{coq_eval}
% Abort.
% \end{coq_eval}

% \begin{Variants}
% \item {\tt Linear with \ident$_1$ \dots\ \ident$_n$}\\
%   \tacindex{Linear with} 
%   Is equivalent to apply first {\tt Generalize \ident$_1$ \dots
%     \ident$_n$} (see section \ref{Generalize}) then the \texttt{Linear}
%   tactic.  So one can use axioms, lemmas or hypotheses of the local
%   context with \texttt{Linear} in this way.
% \end{Variants}

% \begin{ErrMsgs}
% \item \errindex{Not provable in Direct Predicate Calculus}
% \item \errindex{Found $n$ classical proof(s) but no intuitionistic one}\\ 
%   The decision procedure looks actually for classical proofs of the
%   goals, and then checks that they are intuitionistic.  In that case,
%   classical proofs have been found, which do not correspond to
%   intuitionistic ones.
% \end{ErrMsgs}


\subsection{\tt Omega}
\tacindex{Omega}
\label{Omega}

The tactic \texttt{Omega}, due to Pierre Cr�gut,
is an automatic decision procedure for Prestburger
arithmetic. It solves quantifier-free 
formulae build with \verb|~|, \verb|\/|, \verb|/\|,
\verb|->| on top of equations and inequations on
both the type \texttt{nat} of natural numbers and \texttt{Z} of binary
integers. This tactic must be loaded by the command \texttt{Require
  Omega}. See the additional documentation about \texttt{Omega}
(chapter~\ref{OmegaChapter}).

\subsection{\tt Ring \term$_1$ \dots\ \term$_n$}
\tacindex{Ring}
\comindex{Add Ring}
\comindex{Add Semi Ring}

This tactic, written by Samuel Boutin and Patrick Loiseleur, 
does AC rewriting on every
ring. The tactic must be loaded by \texttt{Require Ring} under
\texttt{coqtop} or \texttt{coqtop -full}.
The ring must be declared in the \texttt{Add Ring}
command (see \ref{Ring}). The ring of booleans is predefined; if one
wants to use the tactic on \texttt{nat} one must do \texttt{Require
  ArithRing}; for \texttt{Z}, do \texttt{Require ZArithRing}.

\term$_1$, \dots, \term$_n$ must be subterms of the goal
conclusion. \texttt{Ring} normalize these terms
w.r.t. associativity and commutativity and replace them by their
normal form.

\begin{Variants}
\item \texttt{Ring} When the goal is an equality $t_1=t_2$, it
  acts like \texttt{Ring} $t_1$ $t_2$ and then simplifies or solves
  the equality.

\item \texttt{NatRing} is a tactic macro for \texttt{Repeat Rewrite
    S\_to\_plus\_one; Ring}. The theorem \texttt{S\_to\_plus\_one} is a
  proof that \texttt{(n:nat)(S n)=(plus (S O) n)}.

\end{Variants}

\Example
\begin{coq_example*}
Require ZArithRing.
Goal (a,b,c:Z)`(a+b+c)*(a+b+c) 
               = a*a + b*b + c*c + 2*a*b + 2*a*c + 2*b*c`.
\end{coq_example*}
\begin{coq_example}
Intros; Ring.
\end{coq_example}
\begin{coq_eval}
Reset Initial.  
\end{coq_eval}

You can have a look at the files \texttt{Ring.v},
\texttt{ArithRing.v}, \texttt{ZArithRing.v} to see examples of the
\texttt{Add Ring} command.

\SeeAlso Chapter~\ref{Ring} for more detailed explanations about this tactic.

\subsection{\tt Field}
\tacindex{Field}

This tactic written by David~Delahaye and Micaela~Mayero solves equalities
using commutative field theory. Denominators have to be non equal to zero and,
as this is not decidable in general, this tactic may generate side conditions
requiring some expressions to be non equal to zero. This tactic must be loaded
by {\tt Require Field}. Field theories are declared (as for {\tt Ring}) with
the {\tt Add Field} command.

\Example
\begin{coq_example*}
Require Reals.
Goal (x,y:R)``x*y>0`` -> ``x*((1/x)+x/(x+y)) == -(1/y)*y*(-(x*x/(x+y))-1)``.
\end{coq_example*}

\begin{coq_example}
Intros; Field.
\end{coq_example}

\begin{coq_eval}
Reset Initial.  
\end{coq_eval}

\subsection{\tt Add Field}
\comindex{Add Field}

This vernacular command adds a commutative field theory to the database for the
tactic {\tt Field}. You must provide this theory as follows:\\

{\tt Add Field {\it A} {\it Aplus} {\it Amult} {\it Aone} {\it Azero} {\it
Aopp} {\it Aeq} {\it Ainv} {\it Rth} {\it Tinvl}}\\

\noindent where {\tt {\it A}} is a term of type {\tt Type}, {\tt {\it Aplus}}
is a term of type {\tt A->A->A}, {\tt {\it Amult}} is a term of type {\tt
A->A->A}, {\tt {\it Aone}} is a term of type {\tt A}, {\tt {\it Azero}} is a
term of type {\tt A}, {\tt {\it Aopp}} is a term of type {\tt A->A}, {\tt {\it
Aeq}} is a term of type {\tt A->bool}, {\tt {\it Ainv}} is a term of type {\tt
A->A}, {\tt {\it Rth}} is a term of type {\tt (Ring\_Theory {\it A Aplus Amult
Aone Azero Ainv Aeq})}, and {\tt {\it Tinvl}} is a term of type {\tt
(n:{\it A})\~{}(n=={\it Azero})->({\it Amult} ({\it Ainv} n) n)=={\it Aone}}.
To build a ring theory, refer to chapter~\ref{Ring} for more details.

This command adds also an entry in the ring theory table if this theory is not
already declared. So, it is useless to keep, for a given type, the {\tt Add
Ring} command if you declare a theory with {\tt Add Field}, except if you plan
to use specific features of {\tt Ring} (see chapter~\ref{Ring}). However, the
module {\tt Ring} is not loaded by {\tt Add Field} and you have to make a {\tt
Require Ring} if you want to call the {\tt Ring} tactic.

\begin{Variants}
\item {\tt Add Field {\it A} {\it Aplus} {\it Amult} {\it Aone} {\it Azero}
{\it Aopp} {\it Aeq} {\it Ainv} {\it Rth} {\it Tinvl}}\\
{\tt \phantom{Add Field }with minus:={\it Aminus}}\\
Adds also the term {\it Aminus} which must be a constant expressed by means of
{\it Aopp}.

\item {\tt Add Field {\it A} {\it Aplus} {\it Amult} {\it Aone} {\it Azero}
{\it Aopp} {\it Aeq} {\it Ainv} {\it Rth} {\it Tinvl}}\\
{\tt \phantom{Add Field }with div:={\it Adiv}}\\
Adds also the term {\it Adiv} which must be a constant expressed by means of
{\it Ainv}.
\end{Variants}

\SeeAlso file {\tt theories/Reals/Rbase.v} for an example of instantiation,\\
\phantom{\SeeAlso}theory {\tt theories/Reals} for many examples of use of {\tt
Field}.

\SeeAlso \cite{DelMay01} for more details regarding the implementation of {\tt
Field}.

\subsection{\tt AutoRewrite  [ \ident$_1$ \dots \ident$_n$ ]}
\tacindex{AutoRewrite}

This tactic \footnote{The behavior of this tactic has much changed compared to
the versions available in the previous distributions (V6). This may cause
significant changes in your theories to obtain the same result. As a drawback
of the reenginering of the code, this tactic has also been completely revised
to get a very compact and readable version.} carries out rewritings according
the rewriting rule bases {\tt \ident$_1$ \dots \ident$_n$}.

Each rewriting rule of a base \ident$_i$ is applied to the main subgoal until
it fails. Once all the rules have been processed, if the main subgoal has
progressed (e.g., if it is distinct from the initial main goal) then the rules
of this base are processed again. If the main subgoal has not progressed then
the next base is processed. For the bases, the behavior is exactly similar to
the processing of the rewriting rules.

The rewriting rule bases are built with the {\tt Hint~Rewrite} vernacular
command.

\Warning{} This tactic may loop if you build non terminating rewriting systems.

\begin{Variant}
\item {\tt AutoRewrite  [ \ident$_1$ \dots \ident$_n$ ] using \tac}\\
Performs, in the same way, all the rewritings of the bases {\tt $ident_1$ $...$
$ident_n$} applying {\tt \tac} to the main subgoal after each rewriting step.
\end{Variant}

\subsection{\tt HintRewrite [ \term$_1$ \dots \term$_n$ ] in \ident}
\comindex{HintRewrite}

This vernacular command adds the terms {\tt \term$_1$ \dots \term$_n$} (their
types must be equalities) in the rewriting base {\tt \ident} with the default
orientation (left to right).

This command is synchronous with the section mechanism (see \ref{Section}):
when closing a section, all aliases created by \texttt{HintRewrite} in that
section are lost. Conversely, when loading a module, all \texttt{HintRewrite}
declarations at the global level of that module are loaded.

\begin{Variants}
\item {\tt HintRewrite -> [ \term$_1$ \dots \term$_n$ ] in \ident}\\
This is strictly equivalent to the command above (we only precise the
orientation which is the default one).

\item {\tt HintRewrite <- [ \term$_1$ \dots \term$_n$ ] in \ident}\\
Adds the rewriting rules {\tt \term$_1$ \dots \term$_n$} with a right-to-left
orientation in the base {\tt \ident}.

\item {\tt HintRewrite [ \term$_1$ \dots \term$_n$ ] in {\ident} using \tac}\\
When the rewriting rules {\tt \term$_1$ \dots \term$_n$} in {\tt \ident} will
be used, the tactic {\tt \tac} will be applied to the generated subgoals, the
main subgoal excluded.
\end{Variants}

\SeeAlso \ref{AutoRewrite-example} for examples showing the use of this tactic.

\SeeAlso file \texttt{contrib/Rocq/DEMOS/Demo\_AutoRewrite.v}

\section{The hints databases for Auto and EAuto}
\index{Hints databases}\label{Hints-databases}
The hints for Auto and EAuto have been reorganized since \Coq{}
6.2.3. They are stored in several databases. Each databases maps head
symbols to list of hints. One can use the command \texttt{Print Hint \ident}
to display the hints associated to the head symbol \ident{}
(see \ref{PrintHint}). Each hint has a name,
a cost that is an nonnegative integer, and a pattern. The hint is
tried by \texttt{Auto} if the conclusion of current goal matches its
pattern, and after hints with a lower cost. The general command to add
a hint to a database is:

\comindex{Hint}
\begin{quotation}
  \texttt{Hint \textsl{name} : \textsl{database} := \textsl{hint\_definition}}
\end{quotation}

\noindent where {\sl hint\_definition} is one of the following expressions:

\begin{itemize}
\item \texttt{Resolve} {\term} \index[command]{Hints!Resolve}\\
  This command adds {\tt Apply {\term}} to the hint list
  with the head symbol of the type of \term. The cost of that hint is
  the number of subgoals generated by {\tt Apply {\term}}.
  
  In case the inferred type of \term\ does not start with a product the
  tactic added in the hint list is {\tt Exact {\term}}. In case this
  type can be reduced to a type starting with a product, the tactic {\tt
    Apply {\term}} is also stored in the hints list.
  
  If the inferred type of \term\ does contain a dependent
  quantification on a predicate, it is added to the hint list of {\tt
    EApply} instead of the hint list of {\tt Apply}. In this case, a
  warning is printed since the hint is only used by the tactic {\tt
    EAuto} (see \ref{EAuto}). A typical example of hint that is used
  only by \texttt{EAuto} is a transitivity lemma.

  \begin{ErrMsgs}
  \item \errindex{Bound head variable} \\
    The head symbol of the type of {\term} is a bound variable such
    that this tactic cannot be associated to a constant.
  \item \term\ \errindex{cannot be used as a hint} \\
    The type of \term\ contains products over variables which do not
    appear in the conclusion. A typical example is a transitivity axiom.
    In that case the {\tt Apply} tactic fails, and thus is useless.
  \end{ErrMsgs}

\item \texttt{Immediate {\term}} \index[command]{Hints!Immediate}\\
  
  This command adds {\tt Apply {\term}; Trivial} to the hint list
  associated with the head symbol of the type of \ident in the given
  database. This tactic will fail if all the subgoals generated by
  {\tt Apply {\term}} are 
  not solved immediately by the {\tt Trivial} tactic which only tries
  tactics with cost $0$.
  
  This command is useful for theorems such that the symmetry of equality
  or $n+1=m+1 \rightarrow n=m$ that we may like to introduce with a
  limited use in order to avoid useless proof-search.
  
  The cost of this tactic (which never generates subgoals) is always 1,
  so that it is not used by {\tt Trivial} itself.

  \begin{ErrMsgs}
  \item \errindex{Bound head variable}\\
  \item \term\ \errindex{cannot be used as a hint} \\
  \end{ErrMsgs}

\item \texttt{Constructors} {\ident}\index[command]{Hint!Constructors}\\
  
  If {\ident} is an inductive type, this command adds all its
  constructors as hints of type \texttt{Resolve}. Then, when the
  conclusion of current goal has the form \texttt{({\ident} \dots)},
  \texttt{Auto} will try to apply each constructor.

  \begin{ErrMsgs}
    \item {\ident} \errindex{is not an inductive type}
    \item {\ident} \errindex{not declared}
  \end{ErrMsgs}

\item \texttt{Unfold} {\ident}\index[command]{Hint!Unfold}\\
  This adds the tactic {\tt Unfold {\ident}} to the hint list
  that will only be used when the head constant of the goal is \ident.
  Its cost is 4.

\item \texttt{Extern \num\ \pattern\ }\textsl{tactic}\index[command]{Hints!Extern}\\
  This hint type is to extend Auto with tactics other than
  \texttt{Apply} and \texttt{Unfold}. For that, we must specify a
  cost, a pattern and a tactic to execute. Here is an example:

\begin{quotation}
\begin{verbatim}
Hint discr : core := Extern 4 ~(?=?) Discriminate.
\end{verbatim}
\end{quotation}

  Now, when the head of the goal is a disequality, \texttt{Auto} will
  try \texttt{Discriminate} if it does not succeed to solve the goal
  with hints with a cost less than 4.

  One can even use some sub-patterns of the pattern in the tactic
  script. A sub-pattern is a question mark followed by a number like
  \texttt{?1} or \texttt{?2}. Here is an example:

\begin{coq_example*}
Require EqDecide.
Require PolyList.
\end{coq_example*}
\begin{coq_example}
Hint eqdec1 : eqdec := Extern 5 {?1=?2}+{~ (?1=?2)} 
                                Generalize ?1 ?2; Decide Equality.

Goal (a,b:(list nat*nat)){a=b}+{~a=b}.
Info Auto with eqdec.
\end{coq_example}
\begin{coq_eval}
Abort.
\end{coq_eval}

\end{itemize}

\Rem There is currently (in the \coqversion\ release) no way to do
pattern-matching on hypotheses.

\begin{Variants}
\item \texttt{Hint \ident\ : \ident$_1$ \dots\ \ident$_n$ :=
    \textsl{hint\_expression}}\\
  This syntax allows to put the same hint in several databases.

  \Rem The current implementation of \texttt{Auto} has no
  optimization about hint duplication: 
  if the same hint is present in two databases 
  given as arguments to \texttt{Auto}, it will be tried twice. We
  recommend to put the same hint in two different databases only if you
  never use those databases together.

\item\texttt{Hint \ident\ := \textsl{hint\_expression}}\\
    If no database name is given, the hint is registered in the "core" 
    database. 

    \Rem We do not recommend to put hints in this database in your
    developpements, except when the \texttt{Hint} command
    is inside a section. In this case the hint will be thrown when
    closing the section (see \ref{Hint-and-Section})
    
\end{Variants}

There are shortcuts that allow to define several goal at once:

\begin{itemize}
\item \comindex{Hints Resolve}\texttt{Hints Resolve \ident$_1$ \dots\ \ident$_n$ : \ident.}\\
  This command is a shortcut for the following ones:
  \begin{quotation}
   \noindent\texttt{Hint \ident$_1$ : \ident\ := Resolve \ident$_1$}\\
   \dots\\
   \texttt{Hint \ident$_1$ : \ident := Resolve \ident$_1$}
  \end{quotation}
  Notice that the hint name is the same that the theorem given as
  hint.
\item \comindex{Hints Immediate}\texttt{Hints Immediate \ident$_1$ \dots\ \ident$_n$ : \ident.}\\
\item \comindex{Hints Unfold}\texttt{Hints Unfold \ident$_1$ \dots\ \ident$_n$ : \ident.}\\
\end{itemize}

%\begin{Warnings}
%  \item \texttt{Overriding hint named \dots\ in database \dots}
%\end{Warnings}

\subsection{Hint databases defined in the \Coq\ standard library}

Several hint databases are defined in the \Coq\ standard
library. There is no systematic relation between the directories of the
library and the databases.

\begin{description}
\item[core] This special database is automatically used by
  \texttt{Auto}. It contains only basic lemmas about negation,
  conjunction, and so on from. Most of the hints in this database come 
  from the \texttt{INIT} and \texttt{LOGIC} directories.

\item[arith] This databases contains all lemmas about Peano's
  arithmetic proven in the directories \texttt{INIT} and
  \texttt{ARITH}

\item[zarith] contains lemmas about binary signed integers from the
  directories \texttt{theories/ZARITH} and
  \texttt{tactics/contrib/Omega}. It contains also a hint with a high
  cost that calls Omega.

\item[bool] contains lemmas about booleans, mostly from directory
  \texttt{theories/BOOL}.

\item[datatypes] is for lemmas about about lists, trees, streams and so on that 
  are proven in \texttt{LISTS}, \texttt{TREES} subdirectories.

\item[sets] contains lemmas about sets and relations from the 
  directory \texttt{SETS} and \texttt{RELATIONS}.
\end{description}

There is also a special database called "v62". It contains all things that are
currently hinted in the 6.2.x releases. It will not be extended later. It is
not included in the hint databases list used in the "Auto with *" tactic.

The only purpose of the database "v62" is to ensure compatibility for
old developpements with further versions of Coq. 
If you have a developpement that used to compile with 6.2.2 and that not
compiles with 6.2.4, try to replace "Auto" with "Auto with v62" using the
script documented below. This will ensure your developpement will compile
will further releases of Coq.

To write a new developpement, or to update a developpement not finished yet, 
you are strongly advised NOT to use this database, but the pre-defined
databases. Furthermore, you are advised not to put your own Hints in the
"core" database, but use one or several databases specific to your
developpement.

\subsection{\tt Print Hint}
\label{PrintHint}
\comindex{Print Hint}
\index[tactic]{Hints!\texttt{Print Hint}}
This command displays all hints that apply to the current goal. It
fails if no proof is being edited, while the two variants can be used at
every moment.

\begin{Variants}
\item {\tt  Print Hint {\ident} }\\
 This command displays only tactics associated with \ident\ in the
 hints list. This is independent of the goal being edited, to this
 command will not fail if no goal is being edited.

\item {\tt Print Hint *}\\
  This command displays all declared hints. 
\end{Variants}


\subsection{Hints and sections}
\label{Hint-and-Section}

Like grammar rules and structures for the \texttt{Ring} tactic, things 
added by the \texttt{Hint} command will be erased when closing a
section.

Conversely, when the user does \texttt{Require A.}, all hints 
of the module \texttt{A} that are not defined inside a section are
loaded.

\section{Tacticals}
\index[tactic]{Tacticals}\index{Tacticals}\label{Tacticals}
We describe in this section how to combine the tactics provided by the
system to write synthetic proof scripts called {\em tacticals}. The
tacticals are built using tactic operators we present below.

\subsection{\tt Idtac}
\tacindex{Idtac}
\index{Tacticals!Idtac@{\tt Idtac}} The constant {\tt Idtac} is the
identity tactic: it leaves any goal unchanged.

\subsection{\tt Fail}
\tacindex{Fail}
\index{Tacticals!Fail@{\tt Fail}}

The tactic {\tt Fail} is the always-failing tactic: it does not solve
any goal. It is useful for defining other tacticals.

\subsection{\tt Do {\num} {\tac}}
\tacindex{Do}
\index{Tacticals!Do@{\tt Do}}
This tactic operator repeats {\num} times the tactic {\tac}. It fails
when it is not possible to repeat {\num} times the tactic.

\subsection{\tt \tac$_1$ {\tt Orelse} \tac$_2$}
\tacindex{Orelse}
\index{Tacticals!Orelse@{\tt Orelse}}
The tactical \tac$_1$ {\tt Orelse} \tac$_2$ tries to apply \tac$_1$
and, in case of a failure, applies \tac$_2$. It associates to the
left.

\subsection{\tt Repeat {\tac}}
\index[tactic]{Repeat@{\tt Repeat}}
\index{Tacticals!Repeat@{\tt Repeat}}

This tactic operator repeats {\tac} as long as it does not fail.

\subsection{\tt {\tac}$_1$ {\tt ;} \tac$_2$}
\index{;@{\tt ;}}
\index[tactic]{;@{\tt ;}}
\index{Tacticals!yy@{\tt {\tac$_1$};\tac$_2$}}
This tactic operator is a generalized composition for sequencing.  The
tactical {\tac}$_1$ {\tt ;} \tac$_2$ first applies \tac$_1$ and
then applies \tac$_2$ to any
subgoal generated by \tac$_1$. {\tt ;} associates to the left.

\subsection{\tt \tac$_0$; [ \tac$_1$ | \dots\ | \tac$_n$ ]}
\index[tactic]{;[]@{\tt ;[\ldots$\mid$\ldots$\mid$\ldots]}}
\index{;[]@{\tt ;[\ldots$\mid$\ldots$\mid$\ldots]}}
\index{Tacticals!zz@{\tt {\tac$_0$};[{\tac$_1$}$\mid$\ldots$\mid$\tac$_n$]}}

This tactic operator is a generalization of the precedent tactics
operator. The tactical {\tt \tac$_0$ ; [ \tac$_1$ | \dots\ | \tac$_n$ ]}
first applies \tac$_0$ and then
applies \tac$_i$ to the i-th subgoal generated by \tac$_0$. It  fails if
$n$ is not the exact number of remaining subgoals.

\subsection{\tt Try {\tac}}
\tacindex{Try}
\index{Tacticals!Try@{\tt Try}}
This tactic operator applies tactic \tac, and catches the possible
failure of \tac. It never fails.

\subsection{\tt First [ \tac$_0$ | \dots\ | \tac$_n$ ]}
\tacindex{First}
\index{Tacticals!First@{\tt First}}

This tactic operator tries to apply the tactics \tac$_i$ with $i=0\ldots{}n$,
starting from $i=0$, until one of them does not fail. It fails if all the
tactics fail.

\begin{ErrMsgs}
\item \errindex{No applicable tactic.}
\end{ErrMsgs}

\subsection{\tt Solve [ \tac$_0$ | \dots\ | \tac$_n$ ]}
\tacindex{Solve}
\index{Tacticals!First@{\tt Solve}}

This tactic operator tries to solve the current goal with the tactics \tac$_i$
with $i=0\ldots{}n$, starting from $i=0$, until one of them solves. It fails if
no tactic can solve.

\begin{ErrMsgs}
\item \errindex{Cannot solve the goal.}
\end{ErrMsgs}

\subsection{\tt Info {\tac}}
\tacindex{Info}
\index{Tacticals!Info@{\tt Info}}
This is not really a tactical. For elementary tactics, this is
equivalent to \tac. For complex tactic like \texttt{Auto}, it displays
the operations performed by the tactic.

\subsection{\tt Abstract {\tac}}
\tacindex{Abstract}
\index{Tacticals!Abstract@{\tt Abstract}}
From outside, typing \texttt{Abstract \tac} is the same that
typing \tac. Internally it saves an auxiliary lemma called 
{\ident}\texttt{\_subproof}\textit{n} where {\ident} is the name of the
current goal and \textit{n} is chosen so that this is a fresh name.

This tactical is useful with tactics such \texttt{Omega} or
\texttt{Discriminate} that generate big proof terms. With that tool
the user can avoid the explosion at time of the \texttt{Save} command
without having to cut ``by hand'' the proof in smaller lemmas.

\begin{Variants}
\item \texttt{Abstract {\tac} using {\ident}}.\\
  Give explicitly the name of the auxiliary lemma.
\end{Variants}

\section{Generation of induction principles with {\tt Scheme}}
\label{Scheme}
\comindex{Scheme}

The {\tt Scheme} command is a high-level tool for generating
automatically (possibly mutual) induction principles for given types
and sorts.  Its syntax follows the schema:

\noindent
{\tt Scheme {\ident$_1$} := Induction for \ident'$_1$ Sort {\sort$_1$} \\
  with\\
  \mbox{}\hspace{0.1cm} \dots\ \\
        with {\ident$_m$} := Induction for {\ident'$_m$} Sort
        {\sort$_m$}}\\

\ident'$_1$ \dots\ \ident'$_m$ are different inductive type
identifiers belonging to
the same package of mutual inductive definitions. This command
generates {\ident$_1$}\dots{} {\ident$_m$} to be mutually recursive
definitions. Each term {\ident$_i$} proves a general principle 
of mutual induction for objects in type {\term$_i$}. 


\begin{Variants}
\item {\tt Scheme {\ident$_1$} := Minimality for \ident'$_1$ Sort {\sort$_1$} \\
    with\\
    \mbox{}\hspace{0.1cm} \dots\ \\
    with {\ident$_m$} := Minimality for {\ident'$_m$} Sort
    {\sort$_m$}}\\
  Same as before but defines a non-dependent elimination principle more
  natural in case of inductively defined relations. 
\end{Variants}

\SeeAlso \ref{Scheme-examples}

\section{Simple tactic macros}
\index[tactic]{tactic macros}
\index{tactic macros}
\comindex{Tactic Definition}
\label{TacticDefinition}

A simple example has more value than a long explanation: 

\begin{coq_example}
Tactic Definition Solve := Simpl; Intros; Auto.
Tactic Definition ElimBoolRewrite b H1 H2 := 
  Elim b; 
  [Intros; Rewrite H1; EAuto | Intros; Rewrite H2; EAuto ].
\end{coq_example}

The tactics macros are synchronous with the \Coq\ section mechanism:
a \texttt{Tactic Definition} is deleted from the current environment
when you close the section (see also \ref{Section}) 
where it was defined. If you want that a
tactic macro defined in a module is usable in the modules that
require it, you should put it outside of any section.

The chapter~\ref{TacticLanguage} gives examples of more complex
user-defined tactics.


% $Id$ 

%%% Local Variables: 
%%% mode: latex
%%% TeX-master: "Reference-Manual"
%%% TeX-master: "Reference-Manual"
%%% End: 



